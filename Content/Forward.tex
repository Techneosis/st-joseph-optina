\addcontentsline{toc}{chapter}{FORWARD}
\chapter*{Forward\markboth{Forward}{}}

\begin{adjustwidth}{1.5cm}{1.5cm}
\centering
In the Name of the Father, and of the Son, and of the Holy Spirit. Amen.
\end{adjustwidth}
\vspace{.25cm}

\begin{adjustwidth}{4cm}{0cm}
	\color{red}
	\centering
	\textit{Upon whom shall I look, but upon him that is humble and quiet and trembleth at my words?}\\
	\hfill --Esaias 66:2
\end{adjustwidth}
\vspace{.25cm}
\hugered{F}OR MANY YEARS NOW, over a decade, we have had an English translation of the biography of the truly humble, meek, and quiet Elder of Optina, Hiero-schemamonk Joseph. The translation was made from the Russian by one of the senior fathers of our monastery, who had earlier translated Seraphim's Seraphim, the life of Pelagia Ivanovna. During these years, we have read the manuscript in the refectory during meals, and pilgrims and visitors to the monastery have read it in the guest quarters. Members of the community, pilgrims, visitors -- all have with one voice pleaded with us to print the biography. As in the instance of \textit{Seraphim's Seraphim}, once one begins to read this delightful and soul-profiting account, one is not able to put it down. So endearing is the image of the meek and joyful Elder that arises from the biography, that one is immediately drawn to him in one's heart.

Since a biography of the great Elder Amvrosy by John Dunlop has already appeared in English,\footnote{John B. Dunlop, \textit{Staretz Amurosy: Model for Dostoyevsky's Staretz Zossima} (Belmont, Mass.: Nordland Publishing Co., 1972).} with an extensive introduction concerning eldership and the founding and history of Optina, there is no reason that we should tire the pious reader with a repetition of names, dates, and details concerning Optina and her holy Elders. We certainly could never approach the work done by Dr. Dunlop, a work of love for which we are all indebted to him. Suffice it to say that the Elder Joseph was the spiritual son and disciple of the Elder Amvrosy and his successor as Elder of the Optina Hermitage.

The present publication, therefore, is a sequel to the book \textit{Staretz Amvrosy}, another chapter in the lives of the Optina Elders. It would be very edifying and spiritually profitable if others would undertake the translation and publication of the complete biographies of the other great Elders of Optina: Lev, Moses, Makary, and Anatoly. Then we would have in English a more complete series of these wondrous fathers. The biography of the Elder Joseph was the last of such biographies of the Optina Elders to be printed, since he reposed in 1911, only six years before that destructive Communist revolution which closed down Optina and its hermitage, as well as countless other monasteries, convents, and spiritual centers of the Russian land.

The biography of the Elder Joseph first appeared in Russian in 1911. It consisted of a brief account about the Elder Joseph written by a monk of the Optina Hermitage, and also of a much longer and more detailed account compiled soon after by the Shamordino Convent. The life compiled by the Shamordino Convent was reprinted by the Holy Trinity Monastery of Jordanville, New York, in 1962. In 1978, the Saint Herman of Alaska Brotherhood of Platina, California, issued a facsimile reprint of the 1911 edition, to which were added photographs from the collection of I. M. Kontsevich and an introduction by him.

In the present volume, we have placed first the two hundred page life by the Shamordino Convent, which covers the life span of the Elder from birth to repose; then we have appended the shorter account of some thirty pages written by an Optina monk whose name remains unknown to us. Although this anonymous account was the first to be printed, the author himself admits that he did not know the Elder Joseph for very long, and he narrates at greater length those incidents which he himself witnessed, giving detailed accounts especially of the last years, the repose, and the funeral of the holy Elder.

To these two lives we have appended the ``Cell Rule of Five Hundred of the Optina Monastery,'' since it is mentioned in the life and was nowhere to be found in English. We have also translated and added a letter by the Elder Joseph entitled, ``May Orthodox Christians Pray For Non-Orthodox Christians, And If So, How May They Pray For Them?'' since this is a very timely subject and of interest to many of the faithful, both native and converts, living in the midst of non-Orthodox. This is the only printed letter of the Elder which we have been able to find. It first appeared in the religious periodical \textit{Edifying Readings}, Moscow, No. 3, 1901, under the name of Hieromonk Joseph. For some time we were not able to ascertain if the letter was by the Elder Joseph of Optina or some other Hieromonk Joseph. However, upon locating the above-mentioned periodical in the St. Vladimir's Theological Seminary library, we verified that the letter indeed was by ``the Superior of the Optina Skete, Father Joseph.'' In the Life it is mentioned that at the time of the holy Elder's repose, there was every intention to print all his collected letters, but-alas-because of the great upheaval of the revolution, this never came to pass. Therefore, this one letter of the Elder, aside from its timely content, becomes even more precious, since until others are found, it must serve as representative of his other correspondence.

The lives and teachings of holy men and women are a great source of instruction to the faithful for they serve as a living example-something concrete, not abstract. In that they fulfilled the injunctions and commandments of the Lord, the saints are the incarnation of the beatitudes, the ``living Bible,'' ``the perfection of the Gospel'' as we chant in the hymns of the Church. Though being ``subject to like passions as we are,''\footnote{James 5:17.} ``out of weakness they were made strong''\footnote{Heb. 11:34.} because of their love for God and their struggles for righteousness' sake, and are therefore especially able to help us and instruct us. Their example and teaching, like that of Holy Scripture, are relevant for every generation and are always contemporary. So is it with the Elder Joseph. This present volume, like that of \textit{Papa-Nicholas Planas}\footnote{\textit{Papa-Nicholas Planas} (Boston, Mass.: Holy Transfiguration Monastery, 1981).} and \textit{Staretz Amvrosy}, are genuine treasure troves, inexhaustible mines filled with a wealth of Orthodox piety and teaching rarely to be met with elsewhere. Take for example chapter five entitled, ``The Elder Joseph's Instructions.'' It is filled with spiritual knowledge which comes from experience, knowledge which can be applied by all of us, each in his own context. How timely is the observation of the Elder concerning dreams and apparitions given on the occasion when a young woman had reposed and began appearing to a friend. He writes, ``The blessed Diadochus advises us not to trust even true, grace-filled apparitions. The Lord will not rebuke one for not trusting, since he does so not out of disdain for God, but in order not to fall into demonic deception'' (p. 178). This is but one of many such instructive and soul-benefiting passages one encounters in this small book.

The Monastery of Optina, with its skete and dependencies, was officially closed by the God-hating Communists in 1923. The Elder in his foreknowledge appears to have foreseen this, for on one occasion he warned, ``... or else, because of our disobedience, the Lord will take away the Elders and He will leave no one'' (p. \pageref{ch5obedience}). The main church of the monastery proper, dedicated to St. Mary of Egypt, was dynamited --its ruins can be seen to this day (see p. \pageref{saint-mary}). The fate of the other churches and chapels was similar. They were plundered and finally dismantled. Some of the buildings were put to other uses. The small cottages (cells) of the skete, where the Elders and fathers lived out their lives, were given to workers as residences and have thus survived to this day. We have heard from unofficial reports that native visitors (foreign visitors are prohibited from visiting the area) have been shown the cell of the Elders Amvrosy and Joseph by its present occupants as well as other places of interest connected with the Elders and monastics who lived there when it was still a monastery and skete. (It is to be understood that some of these visitors are believers who come as pilgrims to pray, and others are native tourists who come out of curiosity. Of late in the Soviet Union there has been a revival of interest in the cultural and national monuments of the past.) This has annoyed the local authorities and they have threatened the present occupants of the cottages with eviction if they continue this illegal work of serving as guides to visitors.

We have also heard from a visiting American professor, who stayed in the Soviet Union for some time, that he was shown a building in Moscow where, according to a Soviet professor, the archives and personal papers of the Optina Monastery and Hermitage have been stored for many years, having been transferred there at the time of the destruction of the monastery. If this is true, then there is hope, if the world does not end shortly, that in the future many documents and personal papers of the Optina fathers may see the light of day, possibly even the correspondence of the Elder Joseph, if it has been preserved. Let us hope and pray that God grant it.

Some years ago, we had the thought to dedicate this small book to Fr. Vladimir of the Holy Trinity Monastery in Jordanville. Fr. Vladimir has a special love for the Elder Joseph and has been encouraging us throughout the years to print the book. However, knowing his modesty, we are certain that if it were dedicated to Fr. Vladimir, it would be a cause of embarrassment to him. Therefore, we have decided that it would be appropriate to dedicate it to two of our hierarchs of blessed memory, Archbishop Andrew of Spring Valley and Bishop Nektary of Seattle, direct disciples of the last Elder of Optina, Nektary, who died in exile in 1928. We know that this will be pleasing to Fr. Vladimir also, since Vladyka Andrew was our common spiritual father for some years. We knew Vladyka Andrew well and on almost every occasion that we went to him --and they were countless-- he would inevitably tell us some story concerning Optina and his experiences there, especially concerning his beloved Elder Nektary, whom he continued to visit secretly, even after the Elder's exile. The Elder Nektary actually reposed in his arms, literally under the epitrachelion of Vladyka Andrew, who was then the young priest Fr. Adrian. Vladyka used to tell us that of all the clergy of the deanery to which he belonged he was the only one to survive the atheist persecutions. All the other clergy were either shot immediately or perished in the camps.

Bishop Nektary would also continuously reminisce concerning Optina and its Elders. On his one visit to our monastery, after our meal in the refectory he related to us concerning the last Elders Theodosy ( + 1920), Anatoly the Younger ( + 1922), and Nektary ( + 1928). Vladyka said that the Elder Theodosy was the embodiment of wisdom and theology, Anatoly of love, and Nektary of humility. He would also speak concerning his memories of Optina and its Elders to our abbot and one of our senior fathers who visited Seattle yearly. Vladyka's blessed mother, who in the tonsure later took the name Nektaria nun, used to take him to Optina from the time that he was a child. She was actually at Optina when the Soviets began closing the monastery, and was arrested with all the clergy, monastics, and visitors who were there. But seeing that she was a simple, pious pilgrim, the authorities released her after one day of custody. Notwithstanding, she continued to take her son Oleg (that was the name of Vladyka Nektary in the world) to see the Elder Nektary at the place of his exile right up to the end, even though this was quite dangerous. It is only appropriate then that this present volume should be dedicated to the blessed memory of these two hierarchs. We wish to thank Mrs. Irina Hay of Lexington, Massachusetts, for her inestimable help in rendering certain passages from Russian into English and her numerous suggestions concerning the whole manuscript. We wish to make acknowledgment also of the service rendered by Mr. John Bruk of Roslindale, Massachusetts, in reading the manuscript and offering his suggestions. We are also indebted to Mr. Anthony Pasquale of Providence, Rhode Island, for providing us with the photograph of the Elder Joseph on page \pageref{elder-joseph-bed}.

Remember us, dear readers, in your prayers unto the Lord.

{
	\vspace{.5cm}
	\hspace*{\fill} {\color{red}HOLY TRANSFIGURATION MONASTERY}
	\vspace{.5cm}
}\\
\textit{Feast of the Holy Apostles}\\
1984
