\chapter{Preparation For The Struggle of Eldership}
\hugered{I}N THE LAST YEARS of the Elder Amvrosy's life, such a multitude came to him that the whole area of the hut was often packed tight with people. Many wished to speak with Fr. Joseph and many were sent to him by the Elder himself. For these conversations, a small reception cell was set up in the women's section, where the icon of the Mother of God, \textit{It Is Truly Meet}, was located. But since, for the most part, it was either people of high standing or those who were particularly respected who used to sit in this cell--and they were not easily sent about their business--Fr. Joseph, tactful and modest as he was, would go out into the anteroom to do his work. Here it was quite cold during the winter, and though he dressed warmly, still he would often catch a cold because of his poor health.

Thus, in February 1888, he fell seriously ill, but because of his long-suffering, he did not give way for a long time and forced himself to carry on. Finally the illness overcame him and he took to his bed. The Service of Holy Unction was solemnly performed for him in the cell of the Elder, who was very saddened by his illness. The doctor advised him to transfer the patient to the infirmary, but Fr. Joseph did not wish to part from his Abba in this time of trial, for who would alleviate his suffering there? Who would comfort and hearten him if the fear of death should come upon him?\footnote{Cf, Ps. 54:4-5.} Moreover, the Elder would not come there to bless his disciple for the night's rest. The Elder himself did not desire the transfer, but the skete Superior came and persistently demanded that the patient be taken to the monastery. This transfer, effected in the cold weather and in an open sleigh, made his illness all the worse, and they brought him to the infirmary barely alive, almost unconscious.

With every passing day he became worse--there was no hope. Fr. Amvrosy was grieved, but of course in spirit and prayer he was inseparable from his beloved child. On Sunday, February 14, during the late Liturgy, with the blessing of the Elder and the permission of the Superior, they tonsured him into the schema.\footnote{See Glossary, p. \pageref{schema}.} On Monday, toward evening, he became so sick that the Prayers at the Departing of the Soul were read over him. In place of sickly groans, the patient loudly emitted prayerful sighs, so that those present became uneasy, and felt that he was near his end.

The Elder, who did not expect him to recover, permitted his spiritual children to bid him farewell.

The Superior also came to call upon him. “Well, is it bad with you?” asked the Archimandrite.

“The pangs of death have surrounded me,''\footnote{Cf. Ps. 17:4.} Fr. Joseph answered

“Maybe you'll be getting up yet,” added the Superior, who sincerely grieved over him.

After the Prayers at the Departing of the Soul had been read, Fr. Joseph asked the brother who looked after him to go to the Elder and tell him that he wished to be allowed to depart in peace. When the brother relayed this request to the Elder Amvrosy, he ordered him to return and, after going in to the patient, to say to himself, “Holy, Holy, Holy, Lord of Sabaoth.” The brother fulfilled the Elder's orders exactly as instructed; he had just pronounced these words when the patient asked for some tea, and from that moment on he began to feel better.

Afterward, this same novice, while sitting behind a screen, heard Fr. Joseph say in a loud voice, “Lord, I did not want this; Thou seest, Lord!” The novice shuddered and, on glancing through a slit in the screen behind which the patient was lying, saw that Fr. Joseph was looking intently at the icon of the Saviour and was raising his hands to Him.

That same brother later said to Fr. Joseph's spiritual children, “Oh, if only you knew what kind of person your spiritual father is!” This same novice told the Elder Amvrosy the following: “I went into the infirmary ward and heard someone speaking behind the screen: 'Be patient, my beloved one, only a little remains.' Thinking that someone was there, I glanced behind the screen and was astounded. No one was there but Fr. Joseph, lying on his back with his eyes closed. Such fear came over me that my hair stood on end.” Afterward, Elder Amvrosy told several people that Fr. Joseph had been deemed worthy of seeing the Heavenly Queen during his illness.

In the evening, a soft whisper was heard throughout the hut that Batiushka Joseph was better. Soon the Elder came out joyously and said, “Fr. Joseph was all set to die, but he didn't die; now he's better.”

After Fr. Joseph's recovery, Archimandrite Isaaky officially assigned him as the Elder's helper, and from that time on he began to hear confessions regularly, since it was becoming difficult for the Elder. Batiushka Amvrosy was very satisfied and joyously said to many, “Now we have a new spiritual father!” A person close to the Elder asked a blessing of him to go once to Fr. Joseph for confession. “Fortunate woman!” the Elder told her, “And not once only, but always.” At this time, the Elder made arrangements for an extension to be built downstairs, in which Fr. Joseph began to receive visitors.

During the summer of the same year, 1888, Batiushka Amvrosy blessed Fr. Joseph to go to Kiev to worship at the holy places. And thus, after almost thirty years, his cherished desire was fulfilled. One cannot help but imagine how much trembling piety, how many sighs of compunction, how many prayers poured out of this pure, humble soul there.

At that time, the Superior of the Kiev Caves Lavra was Archimandrite Juvenaly (Polovtsev) who had formerly lived in retirement in the Optina Hermitage. After his arrival at the Lavra, Fr. Joseph went to him to give him greetings from the Elder Amvrosy. The Superior did not happen to be there, and the cell-attendants, accustomed to their director's important, distinguished guests, suggested that the unfamiliar monk wait in the foyer. He had to wait for a long time; the cell-attendants paid no attention to him whatsoever. At last, it was time for dinner and, having remembered their guest, the attendants called him inside to eat. There before dinner, according to an established custom, they offered him something to drink. But Fr. Joseph, never having taken any kind of wine into his mouth, absolutely refused. The cell-attendants taunted him and laughed at him, but finally left him in peace. They then began eating and conversing freely, being not in the least reserved in the stranger's presence. In the meantime, the Superior came and a cell-attendant reported that some Optina monk was awaiting him. Seeing Fr. Joseph, the Superior exclaimed, “Whom do I see? Why, this is the future Elder!” and hurried to embrace him, showing him such respect that the cell-attendants--dumb-struck--did not know what to think. Then he took him to his cell and ordered the cell-attendant to transfer his things from the guest house and to prepare a room for him in his quarters. He conversed with his guest, and right up until evening he reminisced about Optina which was so dear to him. When Fr. Joseph came to the room set aside for him, the two cell-attendants were waiting for him there. Throwing themselves at his feet, they asked forgiveness for their rudeness and begged him not to relate their intemperance to the Archimandrite. The meek Fr. Joseph embraced them and reassured them with a smile of love.

One of these cell-attendants, who later became a hieromonk in one of the monasteries in the Kursk Province, related this story and added, “It was as if boiling water had been poured on us. We thought, ‘Now we're done for; he'll complain to the Superior.' But then he amazed us with his humility and meekness and he spoke so kindly to us that we were truly ashamed of our conduct."

In Kursk, he visited the protopresbyter whom he had known in his childhood. The protopresbyter was so happy to see his friend's son that he did not know how to show his affection for Fr. Joseph and he called him at one moment Ivan Evfimovich and at another time Evfimy Emilyanovich (his father's name), thereby showing that in him he revered the father whom he had so loved and respected.

While passing by, Fr. Joseph also visited the Borisovsk Convent where his sister, Mother Leonida, was living. The sickly old woman nearly fell ill from joy. She was lost in admiration of her "little brother.” It seemed to her that she could not speak enough with him and could not gaze at him enough. They both remembered how he came with the knapsack and cleared the snow near her cell; how much anxiety there was in her soul for him then. Now, it had all passed and she saw before her the hieromonk respected by all, the assistant to the great Elder. To complete her joy, she wanted to see him serving, and the Abbess and sisters also pleaded with him to do so. But Fr. Joseph refused, for he still had not completely regained his strength after the illness he had endured, and had been very fatigued by the long and unaccustomed journey, and was not feeling very well. Mother Leonida was so distressed, however, and begged him with so many tears, that he finally acceded to her request.

The Liturgy began. Just before the Gospel, there befell an unexpected silence. It turned out that Fr. Joseph had begun to feel faint and the priest who was concelebrating with him finished the Liturgy in his stead. Mother Leonida, who was in the church and heard the silence and commotion in the altar, surmised that something had surely happened to her brother; she fainted and was barely revived. When he recovered, Fr. Joseph came to her, and with his usual smile, said, “Well now, you've had your consolation!” Then Mother Leonida began requesting that he leave soon for Optina. “Leave soon” she said, “because if you die here, I'll never be able to bear it."

The Elder Amvrosy was immensely gladdened at the return of his helper, since without him it had been very difficult with the visitors. The life of both strugglers took up its usual course again; the Elder went every summer for about three weeks to stay at Shamordino, and his faithful friend, Fr. Joseph, was his constant companion.

The summer of 1890 came. In June the usual preparations for going to Shamordino began, but the Elder said to Fr. Joseph, “I'm not taking you this time; you have to stay here; you’re needed here.” It was the first time in the thirty years that they had been living together that Fr. Amvrosy had gone without him. Moreover, Fr. Amvrosy ordered Fr. Joseph to move into his cell and to transfer the large icon \textit{Surety of Sinners} into the reception room. Fr. Joseph was saddened over all these arrangements; his heart was painfully tortured. “The Elder will never return!” flashed through his mind.

The Elder departed. Summer passed by. Several times Fr. Amvrosy planned to return to Optina, but an invisible hand restrained him every time; then the oncoming autumn put an end to his further attempts to leave Shamordino. Fr. Joseph's foreboding came true: The Elder returned no more to his skete hut.

In the beginning, Fr. Joseph became very lonely without the Elder; he was just like a complete orphan, now all alone. But through his unfailing submission to the will of God and that of the Elder, he reconciled himself to the situation. He went to the Elder once a month, and Fr. Amvrosy in his fatherly way always made sure that there was a closed sleigh sent for him. The Elder's quarters were adjacent to the building of the Superior. When Fr. Joseph came to the Elder, he never considered himself at home and never went in to the Elder until the cell-attendant had announced him. He would often have to wait for a long time since, in his humility, he gave place to everyone else. Even other monks coming to Shamordino to see the Elder were freer and bolder than Fr. Joseph, and they spent much more time with the Elder; however, the Elder saw and valued the deep, limitless, selfless love of Fr. Joseph.

In the last days of his life, the Elder Amvrosy was often worried and exhausted by all the affairs and concerns of the unfinished convent; often the continual conversations about these affairs completely exhausted him. After these conversations, he refused all food and could not sleep. Often those around him heard the Elder's words: “Only after seeing Fr. Joseph can I eat and sleep on schedule.” By this the Elder wished to say that only Fr. Joseph did not vex him. He never even spoke to the Elder concerning his grief over their separation, so as not to upset him. This was discovered accidentally. One of Fr. Joseph's spiritual daughters, seeing him very sad and sorrowful, said to the Elder, “Batiushka, how hard it is to look at Batiushka Joseph! How grieved he is without you."

To this the Elder answered her, sighing heavily, “He says nothing about this to me.”

Meanwhile at Optina, in the Elder Amvrosy's absence, the monks accustomed to the Elder's guidance began to go to Fr. Joseph, and many began confessing to him. The Superior, because of his extreme old age, found it laborious in winter to travel each time to Shamordino to see the Elder; therefore he chose Fr. Joseph for his confessor and treated him with great respect. He went to the skete every Saturday himself, and after confession, stayed for a long time and conversed with him. It was moving to see how the venerable white-haired Superior went to the monk whom he had tonsured to repent before him of his voluntary and involuntary sins, humbly kneeling before the holy icons.

When the Christmas fast came, Fr. Amvrosy, who had become extremely weak, began sending his spiritual children at Shamordino to Optina to confess to Fr. Joseph. At first, this was difficult for the sisters who were used to entrusting their spiritual secrets only to the Elder, and they went reluctantly to their new spiritual father. Sometimes the Elder would confess a sister himself and then send her to Optina to Fr. Joseph for the prayers of absolution to be read. In these actions of the Elder, there was hidden a deeper intention. By sending his own spiritual children to Fr. Joseph,\footnote{The majority of the sisters had the head of the skete, Fr. Anatoly, as their spiritual father, but a small number had confessed to Elder Amvrosy since their entrance into the convent.} the Elder showed that he was giving them over to no one but Fr. Joseph. This action had an important significance later after the repose of the Elder and when the question of his successor arose.

One nun related, “A year before his death, Fr. Amvrosy blessed me to confess to Fr. Joseph, but I did not want to go to him. Fr. Amvrosy said to me, 'And if I die to whom will you go?'

“I began to weep and said, 'I don't know; besides you, there is no one else I can go to.'

“To this he answered, “Well, I entrust you to Fr. Joseph; conduct yourself with him and write down everything for him, as you do with me. So now, go and confess.'

“I went unwillingly. Fr. Joseph was in the lower hut; I received a blessing and said that Fr. Amvrosy had sent me to confess to him, but that I didn't want to. He looked at me and said, 'As you wish; if Batiushka sent you . . .' and with these words he left me. A little later Batiushka Joseph came and began confession. How light, how joyful my soul became! Having finished confession, I went again to Fr. Amvrosy. He received me soon after my arrival; there was no end to my joy and Batiushka, himself joyful, met me with the words, 'Well, did you confess?'

“I said, 'Yes.'

“He said, 'You told everything to Fr. Joseph, you repented of everything?'

“I answered,‘Everything,' and Batiushka said, 'Well now, tell me, confess.'

“I again began to confess while Batiushka, listening attentively, was looking somewhere beyond my head and smiling. I could not understand what this meant. Then I looked around--and what did I see? Batiushka Joseph standing behind me and smiling. Then Fr. Amvrosy, smiling, said to me, ‘Well, go now. May the Lord help you.' I went, and what delight there was in my soul! To describe it is impossible, for only he who has experienced it himself can understand. Fr. Joseph received me the sinner and he was for me a father and guide, the comfort and joy of my life.”

Not long before the Elder's repose, it was rumored that they wanted to move Batiushka Joseph out of the hut and place him in a cell among the brethren of the skete. This report greatly upset many and they began to say to Fr. Joseph, “Why don't you go to Batiushka? If you wait, they'll move you out.”

But Fr. Joseph, with his customary tranquillity, plied, “When they order me to go out, then I'll go to Batiushka. You see, without his blessing I can't leave here anyway; but with his prayers it will be all right for me to live even in the skete.”

Thus because of his humility he never feared any kind of oppression; things go well for the humble man everywhere and he is prepared to meet everything with joy. This rumor troubled the Elder Amvrosy, and he sent someone to find out about it; however, everything calmed down.

In September 1891, the Elder became ill. The illness at first was considered to be trivial, and no one became especially frightened. But Batiushka Joseph at Optina, while awake heard repeated three times the words, “The Elder will die.” With his usual humility, he said nothing of it to anyone and, fearing to give credence to such a revelation, he gave no significance to it and did not even go to Shamordino.

But on October 8, the Elder's condition became so critical that they sent for Fr. Joseph. When he came, the Elder could hardly speak on account of his weakness. He silently pointed to the book for confession, which Fr. Joseph read through with great sorrow. In the evening, the Service of Holy Unction was solemnly performed for the Elder; then in the morning, Fr. Joseph for the last time gave Communion to the Elder, as he had always done at Optina. One should have seen with what pious feeling he fulfilled this last service for his Abba!

On October 10, at 11:30 in the morning, the great Elder, Hiero-schemamonk Amvrosy reposed, leaving behind a great multitude of mourners. But the grief of his closest disciple, who had lived for thirty years by his will and had been nourished by his teachings, was undoubtedly greater than that of everyone else. His courageous and strong soul revealed all its greatness during these unbearably grievous moments. At a time when many very spiritual people were shocked by this apparently premature death, he alone did not for one moment become confused and did not lose heart, but he comforted and strengthened others. From the very beginning, many found him a haven and spiritual support for themselves; they felt that the spirit of the reposed Fr. Amvrosy dwelt in the new Elder.

When the question arose as to where the reposed Elder Amvrosy was to be buried, there was much agitation. The majority said that he should be buried at Shamordino, since this convent had been founded and built by his labors and care, and since the Elder had gone to live there for the sake of its welfare. Even the Superior of the Optina Hermitage agreed. Bishop Vitaly of Kaluga was also of the same opinion. Of course, Fr. Joseph, as a dedicated son of the Optina Monastery, could not but wish that this great teacher be laid to rest among the family of Optina elders; yet because of all his love and devotion for the reposed, he was filled with a special pity for the orphaned Shamordino Convent and therefore, immediately after the Elder's death, he expressed his agreement also to let them have the body of the Elder. Since the Optina Hermitage undoubtedly had lawful rights, Vladyka decided not to take the responsibility upon himself and he sent a telegram to the Holy Synod.

At this time, Elder Joseph ordered all the sisters to pray fervently that the Lord dispose the hearts of the authorities to lay the body of the Elder in their convent. But when a telegram was received from the Synod with the decision that the body be given burial in the Optina Hermitage, Fr. Joseph did not allow anyone to dispute the directions of the higher authority; he strictly charged the weak, the grumbling, and the indignant, saying, “You see, you prayed and it didn't come out as you wanted it; it was not done without the will of God, so that means it must be thus."

The funeral service was conducted in the Shamordino church by the Most Reverend Bishop Vitaly on the thirteenth, and on the next day the body of the Elder was solemnly taken to the Optina Hermitage. Despite the rain and wind, the candles did not go out throughout the journey. It was already getting dark when the procession approached Optina, and the bell began to toll, calling the brotherhood to meet their father and spiritual director, bereft now of breath. Some distance behind the bier, a lonely monk walked hastily. His face was sad, his lips whispered a prayer. It was the beloved disciple of the deceased Elder, his successor, the Elder Joseph. His Abba, who had given him rebirth and guided him spiritually, was returning after a long separation; but alas, he was not returning to the skete or the hut, he was not returning for life and new direction; he was returning to lie at rest beside the precious relics of his great teachers, the Elders Lev and Makary.

On the day of the burial, October 15, the Most Reverend Bishop Vitaly conferred upon the new Elder the \textit{nabedrennik}.\footnote{See Glossary, p. 298.} After dinner, Vladyka, walking by the fresh grave and seeing near it a crowd of nuns, turned to them with a question:

“Shamordino sisters, whom will you now choose for your Elder? Surely Fr. Joseph?”

“Bless, holy Master,” they answered.

“May God bless,” said Vladyka.

The night before, the Bishop had already visited the Elder in the skete hut. He treated Fr. Joseph very benevolently and when leaving said, “We will see each other again.”

After the burial, Fr. Joseph was unexpectedly ordered to go to the Bishop. Seeing him, Vladyka greeted him very kindly. He invited him to take part in the council, and turning to everyone, added, “Fr. Joseph does not hold an official position, but I would like him to take part in our council.” By showing this high honor to that simple hieromonk of the skete, Vladyka wanted to emphasize Fr. Joseph's position as an Elder.

As was known, Bishop Vitaly, because of his sickly condition,\footnote{He reposed in Kiev a year later from cancer of the stomach, in the monastery of Archimandrite Jonah, for whom he had always had a special love.} was sometimes irritable and quick-tempered. And so it happened here. Archimandrite Isaaky brought upon himself the displeasure of the Bishop because of something he said for the benefit of the Shamordino Convent; the Bishop became flushed and stopped the Superior abruptly. The venerable Elder, Archimandrite Isaaky, was obviously taken aback; consternation ensued. Everyone became silent, not daring to contradict the irate Bishop. But behold, the humble Fr. Joseph stood up, and with the unfeigned simplicity characteristic of him, meekly but firmly spoke his just word whereby he fearlessly supported the Archimandrite. Everyone stood stock-still, expecting a storm. But Vladyka immediately mellowed and began speaking in a completely different tone; he recognized the justice of Fr. Joseph's case and consented to the opinion of the Archimandrite and the Elder. Behold the characteristics of evangelical selflessness: when it was necessary, he was ready to suffer for the truth; but if there were no special urgency or need, he never sought to distinguish himself among the brotherhood.

Upon leaving Optina, the Bishop, among other things, said the following significant words to the brotherhood, “. . . Choose for yourselves an elder according to the spirit of your monastery . . .” Indeed, it was possible to find elders both according to years and according to life, but they were of a different spirit.

The grace of eldership is a special gift which not everyone desiring it is granted, even though he be of a good and spiritual life. Elders are specially chosen ones of God, prepared beforehand by the special ways of God's providence. Their main and essential qualities must be: spiritual experience (gained by an active life in obedience under the direction of an Elder), discretion (given only to pure hearts enlightened by the Spirit of God), patience, love, and humility. The last quality is an especially evident sign of a true Elder. Even if he be neither learned nor well-read, even if he cannot carry on a philosophical conversation or answer all theological questions, the Elder who is humble can be wiser with the grace inherited through humility than “the disputers of this age,"\footnote{I Cor. 1:20.} and he can give a word of edification in such strength of the spirit that before it all theoretical knowledge proves to be insignificant. A man may lead a good life and speak persuasively and be experienced in writing, but that power of grace which often abides in the simple Elders marked by God is not in him. Let us take for an example any monastic establishment in which there are such Elders. Perhaps many monks leading a good and even ascetical life live there; perhaps there are monks who are well-read, and perhaps even some with a scientific education, who are developed intellectually and spiritually, and with whom it is pleasant to speak; yet if a visitor is sick in soul, if his conscience is tormenting him, if his heart is troubled and suffering from any cause, he will not find an answer to his burning questions by speaking with such monks. He will not open his soul to them, but he will go seeking for a monk of grace who, with his all-forgiving love, will embrace him like a sheep gone astray, and with his humility will conquer his haughty intelligence, and with his simple, unlearned, yet grace-filled words will dispel his perplexity and will open to him the word of life. The influence of such a person is boundless. One must remember, however, that only those who have received anointing from above possess this gift. Therefore, those who in the course of their life turn for instruction to anyone, without discernment, fall into great error. “Can the blind lead the blind? Shall they not both fall into the ditch?'\footnote{Luke 6:39.}

Little by little, uncompelled by anyone, the Optina brotherhood began to come with their spiritual needs to the direct heir of the spiritual gifts of the Optina Elders, Fr. Joseph. The Shamordino Convent, with its Superior at the head,\footnote{To the question of the Superior as to whom the nuns should now go for spiritual direction, Archimandrite Isaaky answered, “As for us in Optina, there is one common Elder, Fr. Joseph; so must he be for you also.”} also placed itself under his direction, and after them other monasteries and convents and laymen as well did likewise. Optina Hermitage once again shone forth its lamp of eldership upon all Russia, and as before, the door of the Elder's cell was open to all those desiring, seeking, and in need of spiritual guidance, support in temptations, and comfort in sorrows.

After the repose of the Elder Amvrosy, there remained many unfinished and involved matters which at first were very difficult for Fr. Joseph to untangle. To an expression of condolence expressed by one individual regarding this matter, Fr. Joseph firmly answered, “Yes, Batiushka left me in an awkward position. But even if I should have to go to Siberia, yet I would never judge the Elder.” Because of such selfless dedication and love, the Lord helped him to bring all matters into complete order, so that everyone was set at ease and began to trust him all the more.

A laywoman, who had been under the reposed Elder's guidance, after his repose was grieving greatly and was at a loss as to where she should now turn, to Fr. Joseph or to someone else? Sitting and pondering these thoughts, as though half-asleep, she yet clearly heard the voice of the Elder Amvrosy saying, “Follow Fr. Joseph--he will be a great luminary.” This put an end to her hesitation, and she went with full faith to Fr. Joseph and began referring her spiritual needs to him.