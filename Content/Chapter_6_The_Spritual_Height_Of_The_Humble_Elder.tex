\chapter{The Spiritual Height Of The Humble Elder}
\hugered{W}HILE TEACHING others patience, humility, guilelessness, and everything necessary for a Christian disposition of soul, the Elder Joseph himself was the first to give example in fulfilling all these virtues. He bore every grief and unpleasantness with such good spirits and tranquility, that outsiders were not able to guess what trials he had.

Sometimes, certain actions of others which were obviously harmful or disturbing were pointed out to him in an agitated manner. In such cases, the Elder meekly said, ``What can we do? We have to be patient. It will not harm us, but the benefit will be great if we bear it humbly.'' He would often say, ```There is a zeal not according to knowledge'''\footnote{Cf. Rom. 10:2.} or ```Ye know not what manner of spirit ye are of.'''\footnote{Luke 9:55.} He never became angry with those who grieved him. It was as though he did not even notice any evil.

When one had to confess that one had judged persons who were ill-disposed to the Elder, he would usually say, ``You should not judge. After all, it is not they, but rather the enemy that is inciting them, so we should pray for them.'' He was so imbued with humility and so strong in spirit, that he calmly accepted both honor and dishonor equally, reacting to praise with the same profound humility. When people told him of the praise that others had accorded him or expressed their own satisfaction with honors given him, he would always answer, ```Let this mind be in you which was also in Christ Jesus.''' and the rest.\footnote{Phil. 2:5.}

Once he was told, ``Batiushka, you are avoiding honor and it follows right behind you.''

The Elder answered this seriously, deeply sighing, ``Of what use is it? What does one need it for? However,'' he added, ``as one must not seek honor, thus also one must not refuse it from those living in society, in order that others might benefit. Honor that is imposed on one is also from God.''

As was said above, the Elder's inner spiritual life was concealed from everyone. Only one thing is known for certain, that he occupied himself with inner prayer, or the so-called ``noetic activity.'' His closest cell-attendant, with whom he was most at ease, related that when he entered the Elder's cell for some reason or other, he often found him saying the Jesus Prayer. Batiushka would be lying on the bed and would be pronouncing with great piety and heartfelt contrition the words of the prayer in a half-whisper, putting special emphasis on the words, ``Lord Jesus Christ.'' In order to avoid self-deception, he kept track of the number of prayer ropes he had said with olive pits which were always in a small box on a little table near the bed. This prayer was especially moving and compunctionate after he received the Holy Mysteries, when the grace-filled Elder would be completely immersed in prayerful contemplation. Owing to the strong movement of the prayer, he could not even keep it within and loudly called upon the Name of God.

Once he was asked how he had received the gift of prayer and whether he had had it for long. Batiushka answered with his usual simplicity, ``Prayer itself teaches. It is said, `Give prayer unto him that prays, and he who is disposed to it will hear one word about it and keep it'; but I have read the \textit{Philokalia} and...'' The Elder did not finish since the rest was clear.

Since Batiushka himself was a man of unceasing prayer, he encouraged others in the use of the Jesus Prayer also. In his instructions, he spoke especially well about the prayer as being the most important activity for everyone. If he saw a special disposition toward it in someone, he tried to develop and maintain in him the spark of this inclination. This he did with his wise advice from experience which was in accordance with the writings of the holy Fathers as set forth in the \textit{Philokalia}.

The Elder resolutely and strictly prevented the impatient and inexperienced from concerning themselves with high levels of prayer. He taught them to travel upon their way gradually by beginning with the Jesus Prayer pronounced orally and with a set number of prayer ropes. ``This protects one against high-mindedness,'' said the Elder. ``Otherwise, by not keeping count, one might think that one has said many prayers, but the prayer rope might show that not even a hundred have been said. Even if you don't completely attain the fruits and perfection of prayer, it is good at least to die on the way toward this. Do not search for high things (i.e., consolation and spiritual illuminations); they come when God pleases.''

Having become tempered in patience, the Elder led others along this same path. ``In your patience gain ye your souls,\footnote{Luke 21:19.}'' and ``He that endureth to the end shall be saved''\footnote{Matt. 10:22.} were his favorite sayings, just as they had been with the Elder Amvrosy. ``For the inexperienced more harm than good can come from consolation, even that which is spiritual,'' he said. ``From it, the soul unnoticeably becomes lifted up; when the soul has become accustomed to consolation, it weakens, so that when griefs arise it loses spirit and falls. Patience is the mother of consolation, according to the holy Fathers.''

When asked, ``What aim must one have in praying?'' the Elder answered, ``Salvation, so that you are asking for mercy and not consolation. A person should not pray out of vainglory, but should bear with thankfulness everything grievous that comes his way. If we receive some kind of consolation in prayer, then all the more should we consider ourselves guilty and indebted, because we received it for nothing.

``Some say that it is impossible to listen attentively to what is being read and sung in church when praying according to a set number of prayer ropes,'' said the Elder. ``However, these people while listening find time for various other thoughts. Even in ancient times, the holy Fathers said the prayer according to a set number and for this purpose they developed the prayer rope.''

As an example of the profit gained from such a practice the Elder pointed out a certain skete novice L.\footnote{A rich, educated young man, who entered the skete and in two years passed away in blessed repose.} ``He complained to me,'' related Batiushka Joseph, ``that it was boring for him to stand for a long time in church, therefore he always wanted to quietly join in with the chanting of the choir. I told him that instead he should say the Jesus Prayer by prayer rope. He began to do so and found that it became easy for him in church and not boring, and the divine services did not seem so lengthy. During his illness, he said the Jesus Prayer and always wanted to be alone, and thus he died with the prayer.''

By his effectual words and living example he would awaken and raise the spirit and would clearly prove that not only is the practice of the prayer possible now just as in ancient times, but it is even necessary for all; and that all things, including prayer itself, are obtained through prayer. Therefore, one should never abandon it.

His great humility and unceasing prayer of the heart brought the Lord Himself to dwell in him, and the grace-filled Elder transcended all earthly things. That this schemamonk was truly like the fiery seraphim is witnessed by the following story of the venerable priest, Fr. P. Levashov, who was known by the people of Optina and Shamordino.

\begin{longquote}{Fr. P. Levashov said:}
In 1907, I happened to visit Optina Hermitage for the first time, although I hadn't planned to. Earlier, I had heard something about the Elders, but I had never seen them. When I came to the monastery, the first thing I did was to go to bed since I had spent a sleepless night in the rail-car. The bell for Vespers woke me. The pilgrims went to the church for the divine service, but I rushed to the skete in order that I might be able to speak while there were fewer visitors. Inquiring the way to the skete and to the Elder Joseph's cell there, I came at last to the reception room of the hut. This reception room was small with very modest furnishings. The walls were hung with portraits of pious ascetics and with sayings of the holy Fathers. There was only one visitor when I arrived, an official from Petersburg. Soon the Elder's cell-attendant summoned the official to Batiushka and said to me, `He has been waiting a long time'. The official was gone for about three minutes and then returned. Around his head I saw rays of a strange light. With tears in his eyes and emotion in his voice, he told me that he and others were present that morning when they had brought the wonderworking icon of the Kaluga Mother of God from the skete. Batiushka had come out of the hut to pray and they had seen rays of light radiating from him while he prayed. Within a few minutes, I was also called to the Elder. I entered his humble cell, which was dimly lit and had only poor furnishings of wood. At that time I saw the aged Elder, worn out by continual struggles and fasting, barely able to raise himself from his cot. He was sick at the time. We greeted each other; a moment later I saw a strange light around his head, a quarter by one-and-a-half in height,\trans{These dimensions are given according to the \textit{arshin}, which is twenty-seven English inches in length.} and also a broad ray of light falling upon him from above as though the ceiling of the cell had parted. The ray of light fell from Heaven and was exactly like the light around his head. The Elder's face became filled with grace and he smiled. I had expected nothing of the sort, and therefore was so amazed that I simply forgot all the questions which had been cluttering my head and for which I sought to receive answers from an Elder experienced in the spiritual life. With most profound Christian humility and meekness (the distinctive qualities of the Elder), he stood and patiently waited for me to speak. Stunned, I could not tear myself away from this sight which was to me totally incomprehensible. At last, vaguely remembering that I wanted to confess to him I began, “Batiushka, I am a great sinner.” I had barely succeeded in saying this much, when his face became serious and the light which was pouring upon him and surrounding his head disappeared. Before me again stood the familiar Elder, whom I had seen when I entered the room. But it did not remain thus for long. In a moment, again the light began to shine around his head and again the same ray of light appeared, only several times brighter and stronger. He refused to confess me because of his illness. I asked for his advice about the opening of a trusteeship\trans{Parish trusteeships were established in 1864 in order to provide for the upkeep and maintenance of the parish church and clergy, for the construction of elementary schools, and for charitable undertakings in the parish. Each trusteeship consisted of local clergymen and the church warden as permanent members, and parishioners chosen by the parish who served for a set number of years.} in my parish and I requested his holy prayers. I could scarcely tear myself away from such a wondrous vision, and I must have said goodbye to Batiushka at least ten times, and all the while I was gazing at his grace-filled countenance which was radiant with an angelic smile and this unearthly light, which continued as I left him. After three years I again went to Optina Hermitage and was with Batiushka Joseph many times, but I never saw him this way again.

The light which I saw above the Elder cannot be compared to any earthly light, such as light from the sun, from phosphorus, from electricity, from the moon, etc.; that is to say, I had never seen its like in visible nature.

I believe this vision resulted from the Elder's being in an intense state of prayer, with the grace of God manifestly coming down upon His chosen one. But why I was deemed worthy of seeing such a vision I cannot explain, knowing my sinfulness, and I can `only boast in mine infirmities.'\footnote{Cf. II Cor. 12:5.} Perhaps the Lord was calling me, a sinner, to the path of repentance and amendment by showing me in a visible manner what kind of grace God's chosen ones can attain while still in this earthly vale of tears and grief.

My story is true, for after this vision I felt unspeakably joyful with a strong religious zeal, although before I went to the Elder I had had no such feeling.

Four years have passed since then, and now, merely remembering it, I relive the compunction and rapture. My story will be `unto the Jews a stumbling-block, and unto the Greeks foolishness,'\footnote{I Cor. 1:23.} and to those of little belief, to those wavering in faith and doubting, an invention, a fantasy, or, at best, a hallucination. In our times of unbelief, faithlessness, and religious decay, such stories give rise to smiles and sometimes even to bitterness. What should we, the servants of truth, do? Be silent? Certainly not! The ever-memorable Elder Joseph is in truth a fiery and brilliant luminary, and one does not hide luminaries under a bushel, but rather puts them on a lamp-stand to shine upon everyone in the true Church of Christ. I ask all believing Christians to pray for him so that he will pray for us before the Throne of God.

All that I have related above is pure truth, and there is here not a shadow of exaggeration or invention, which I testify in the Name of God and by my conscience as a priest.
\end{longquote}
