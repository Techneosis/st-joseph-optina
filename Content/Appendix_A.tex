\addcontentsline{toc}{chapter}{APPENDIX A\\The Cell Rule Of Five Hundred Of The Optina Monastery}
\chapter*{The Cell Rule Of Five Hundred\\Of The Optina Monastery\footnote{This Cell Rule is taken from \textit{The Monastic Cell Rule} (in Russian), reprinted by the Convent of the Vladimir Mother of God in San Francisco (San Francisco, 1977), pp. 23-27.}\markboth{Appendix A}{}}

\label{cell-rule}\hugered{I}n addition to the church services: the Liturgy, Matins, and Vespers with Compline, which all of the brethren of the monastery are obliged to attend, many of them daily read in their cells: One chapter from the Gospel in order, beginning first with the Gospel of Matthew, to the last chapter of the Gospel of John, and two chapters from the Epistle, likewise in order, beginning with the Acts of the holy Apostles and ending with the last chapter of the Apocalypse of Saint John the Theologian. The last seven chapters of the Apocalypse are read one a day. In this way the last chapter is read on exactly the same day as the last chapter of the Gospel of John.\footnote{There are 89 chapters in the 4 Gospels, 171 chapters in the remaining books of the New Testament. With the 7 concluding chapters of the Apocalypse being read one a day, there would be 89 readings, corresponding to the chapters of the Gospels. One would thus begin a new cycle every 90th day. -TRANS.} Then, after the completion of the reading of the whole New Testament, in this manner they begin again from the first chapters a new cycle of reading in precisely the same order. From the Psalter they read one \textit{kathisma} a day, beginning with the first and ending with the last.\footnote{There are 20 kathismata in the Psalter. -TRANS.} In addition to this, they perform the so-called Cell Rule of Five Hundred in the following order:

After the customary three prostrations performed at the commencement of every rule of prayer, both in church and in one's cell, with the prayers: (1) \textbf{God, be merciful to me a sinner.} (2) \textbf{God be gracious\footnote{The Slavonic word \textit{ochisti} is often translated ``cleanse,'' because the word in Slavonic has the fundamental meaning of ``cleanse'' or ``purify.'' An extended meaning is the cleansing of sins, which includes meanings of ``to forgive,'' ``to show mercy,'' ``to be gracious.'' This last rendering accords with the Greek \textit{hilastheti}. –TRANS.} unto my sins and have mercy on me.} (3) \textbf{O Thou Who hast fashioned me, Lord, have mercy. I have sinned beyond measure, O Lord, forgive me,} in one's cell a fourth prostration is added together with the prayer: \textbf{My Lady, Most Holy Theotokos, save me a sinner.} Then the following prayers are said:

\begin{center}
Through the prayers of our holy Fathers, Lord Jesus\\Christ our God, have mercy on us. Amen.

Glory to Thee, our God, glory to Thee.
\end{center}

Heavenly King, O Comforter, the Spirit of truth, Who art everywhere present and fillest all things, O Treasury of every good and Bestower of life: Come and dwell in us, and cleanse us from every stain, and save, O Good One, our souls.

\begin{hangparas}{.25in}{1}
Holy God, Holy Mighty, Holy Immortal, have mercy on us (3).

Glory to the Father, and to the Son, and to the Holy Spirit, both now and ever, and unto the ages of ages. Amen.
\end{hangparas}

All Holy Trinity, have mercy on us. Lord, be gracious unto our sins. Master, pardon our iniquities. Holy One, visit and heal our infirmities for Thy name's sake.

\begin{center}
Lord, have mercy (3).
\end{center}

\begin{hangparas}{.25in}{1}
Glory to the Father, and to the Son, and to the Holy Spirit, both now and ever, and unto the ages of ages. Amen.
\end{hangparas}

Our Father, which art in the Heavens, hallowed be Thy name. Thy Kingdom come. Thy will be done, on earth as it is in Heaven. Give us this day our daily bread. And forgive us our debts, as we forgive our debtors. And lead us not into temptation, but deliver us from the evil one.

\begin{hangparas}{.25in}{1}
Through the prayers of our holy Fathers, Lord Jesus Christ our God, have mercy on us. Amen.

Lord, have mercy (12). Glory to the Father, and to the Son, and to the Holy Spirit, both now and ever, and unto the ages of ages. Amen.

O come, let us worship God our King.

O come, let us worship and fall down before Christ our King and God.

O come, let us worship and fall down before Him, Christ the King and our God.
\end{hangparas}

\begin{center}
\vspace{1cm}
Psalm 50
\end{center}

\begin{hangparas}{.25in}{1}
Have mercy on me, O God, according to Thy great mercy; and according to the multitude of Thy compassions blot out my transgression.

Wash me thoroughly from mine iniquity, and cleanse me from my sin.

For I know mine iniquity, and my sin is ever before me. Against Thee only have I sinned and done this evil before Thee, that Thou mightest be justified in Thy

words, and prevail when Thou art judged.

For behold, I was conceived in iniquities, and in sins did my mother bear me.

For behold, Thou hast loved truth; the hidden and secret things of Thy wisdom hast Thou made manifest unto me.

Thou shalt sprinkle me with hyssop, and I shall be made clean; Thou shalt wash me, and I shall be made whiter than snow.

Thou shalt make me to hear joy and gladness; the bones that be humbled, they shall rejoice.

Turn Thy face away from my sins, and blot out all mine iniquities.

Create in me a clean heart, O God, and renew a right spirit within me.

Cast me not away from Thy presence, and take not Thy Holy Spirit from me.

Restore unto me the joy of Thy salvation, and with Thy governing Spirit establish me.

I shall teach transgressors Thy ways, and the ungodly shall turn back unto Thee.

Deliver me from blood-guiltiness, O God, Thou God of my salvation; my tongue shall rejoice in Thy righteousness.

O Lord, Thou shalt open my lips, and my mouth shall declare Thy praise.

For if Thou hadst desired sacrifice, I had given it; with whole-burnt offerings Thou shalt not be pleased.

A sacrifice unto God is a broken spirit; a heart that is broken and humbled God will not despise.

Do good, O Lord, in Thy good pleasure unto Sion, and let the walls of Jerusalem be builded.

Then shalt Thou be pleased with a sacrifice of righteousness, with oblation and whole-burnt offerings. Then shall they offer bullocks upon Thine altar.
\end{hangparas}

\begin{center}
	\vspace{1cm}
	\textit{The Symbol of Faith}
\end{center}

\begin{hangparas}{.25in}{1}
I believe in one God, the Father Almighty, Maker of heaven and earth, and of all things visible and invisible.

And in one Lord Jesus Christ, the Son of God, the Only-begotten, begotten of the Father before all ages; Light of Light, true God of true God; begotten, not made; being of one essence with the Father; by Whom all things were made;

Who for us men, and for our salvation, came down from the Heavens, and was incarnate of the Holy Spirit and the Virgin Mary, and became man;

And was crucified for us under Pontius Pilate, suffered and was buried;

And arose again on the third day according to the Scriptures;

And ascended into the Heavens, and sitteth at the right hand of the Father;

And shall come again, with glory, to judge both the living and the dead; Whose kingdom shall have no end.

And in the Holy Spirit, the Lord, the Giver of life; Who proceedeth from the Father; Who with the Father and the Son together is worshipped and glorified; Who spake by the prophets.

In One, Holy, Catholic, and Apostolic Church. I confess one baptism for the remission of sins. I look for the resurrection of the dead, And the life of the ages to come. Amen.
\end{hangparas}

Then one hundred prayers: \textbf{Lord Jesus Christ, Son of God, have mercy on me a sinner}, with full prostrations for the first ten prayers, full bows for the next twenty prayers, and on the last, that is, the hundredth prayer, again a full prostration.

After this, the following Prayer to the Most Holy Theotokos:\footnote{This prayer is found at the end of the Morning Prayers.}

\begin{adjustwidth}{1cm}{1cm}
My Most Holy Lady Theotokos, by thy holy and all-powerful entreaties dispel from me, thy humble, wretched servant, despondency, forgetfulness, folly, carelessness, and all impure, evil, and blasphemous thoughts out of my wretched heart and my darkened mind. And quench the flame of my passions, for I am poor and wretched, and deliver me from my many cruel memories and deeds, and free me from all evil actions: for blessed art thou by all generations, and glorified is thy most honourable name unto the ages of ages. Amen.
\end{adjustwidth}

At the end of this prayer a full prostration.

Then again one hundred Jesus Prayers in the same order as before with ten full prostrations and twenty full bows. On the last Jesus Prayer a full prostration and again the same prayer: \textbf{My Most Holy Lady Theotokos}, with a full prostration.

The third group of one hundred likewise as the first and second.

The fourth group of one hundred consists of prayers to the Most Holy Theotokos: \textbf{My Most Holy Lady Theotokos, save me a sinner.} In this group of one hundred the first ten prayers are likewise made with full prostrations and the following twenty with full bows, the remaining without bows. The last and hundredth prayer is made with a full prostration, after which with a full prostration the prayer: \textbf{My Most Holy Lady Theotokos.}

Then fifty prayers: \textbf{O holy Angel of God, my guardian, pray to God for me a sinner.} On the first five prayers, full prostrations; on the following ten, full bows; the remaining thirty-four, without bows. Only on the last prayer a full prostration and again the prayer: \textbf{My Most Holy Lady Theotokos}, with a full prostration.

Then fifty prayers: \textbf{All Saints, pray to God for me a sinner.} On the first five prayers, full prostrations; on the following ten, full bows; the remaining thirty-four, without bows. Again the last prayer with a full prostration, after which is said the prayer: \textbf{My Most Holy Lady Theotokos}, with a full prostration. Then:

\begin{adjustwidth}{1cm}{1cm}
It is truly meet to call thee blessed, the Theotokos, the ever-blessed and all-immaculate and Mother of our God. More honourable than the Cherubim, and beyond compare more glorious than the Seraphim, thee who without corruption gavest birth to God the Word, the very Theotokos, thee do we magnify.
\end{adjustwidth}

At the end of this prayer a full prostration.

After this: \textbf{Glory to Thee, Christ God, our Hope, glory be to Thee. Glory. Both now. Lord, have mercy} (3) and \textbf{Through the prayers of our holy Fathers, Lord Jesus Christ our God, have mercy on us. Amen.}

On weekdays all of the above-mentioned bows and prostrations are performed. On the days of Pentecost,\trans{The ``days of Pentecost'' extend from Pascha to the beginning of the Fast of the Holy Apostles.} on days when there is the Polyeleos, on Forefeasts and for the duration of the Feasts, on days when the Great Doxology is chanted at Matins and in the church services full prostrations are dispensed with, in like manner in one's cell the full prostrations are replaced with full bows, as is also the case on all days throughout the year when there is a Vigil. On the last two days of Passion Week, for all of Bright Week, and from the twenty-fourth of December until the seventh of January, this cell rule is completely dispensed with, as is likewise the case on all Sundays throughout the year, even if the all-night Vigil has not been performed, but only Vespers and Matins, as is done in winter.

Any change in the composition of this cell rule, as well as deduction from it or addition to it, is left to the will and blessing of the Elder or Spiritual Father of the individual.
