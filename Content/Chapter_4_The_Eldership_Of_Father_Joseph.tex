\chapter{The Eldership Of Father Joseph}
\hugered{W}E HAVE SEEN how the Elder Amvrosy himself instructed and helped his beloved son and disciple, the heir of his spiritual gifts. So too, after the repose of the great Elder Amvrosy, his cell, or, as it was called, his hut--the witness of so many prayers and struggles, where so many had received spiritual regeneration, where so many tears of grief, joy, and repentance had been shed--was not deserted; the flow of living water stopped a short time, only to gush forth anew as a strong spring. The spirit of the Elders Amvrosy and Makary arose in the person of Fr. Joseph. Although the latter, of course, had his own individual traits, he was so permeated with the spirit of his great teacher in all his views, acts, and decisions, that he became as it were his reflection. Especially at first, it was precisely this quality which was valued in him. Because of the love, faith, and devotion which everyone had for the Elder Amvrosy, it was difficult for them to surrender themselves to another guide. The awareness and assurance that Fr. Joseph would say exactly what Fr. Amvrosy would have said, that he would decide a question without fail just as his deceased Elder would have resolved it, that one would hear from him directions which he himself had previously received from the Elder--this spiritual unity, this, so to speak, visible succession of the great gift of eldership, attracted people and gradually drew them toward him. Even Fr. Joseph's external features began to resemble Fr. Amvrosy, and this purely external aspect moved the souls of the deceased Elder's spiritual children. His mystical, spiritual presence was felt by everyone; when the Elder Joseph went out for a general blessing, often one could hear such exclamations as, ``Yes, this is none other than Batiushka Amvrosy himself... how he resembles Batiushka!''

Fr. Joseph received guests in the same cell where the deceased Elder had; he confessed sitting in the same place, that is, on the bed where the deceased Batiushka, because of his ill health, had spent his time in a semi-reclining position. Now at the head of the bed there stood a large, very faithful portrait of the Elder Amvrosy. This whole setting spoke intimately to the soul. Aside from this, the Elder Joseph, on account of his humility and love for the Elder Amvrosy, never said anything of himself, but always referred to an example from the life of his guide. He himself was very lonely without the Elder and loved to recall memories of him; often these cherished reminiscences would suddenly stop short, half way through a word ``... all is passed, all is passing,'' concluded Fr. Joseph with a deep sigh and a quiet sorrow shone in his meek eyes. Very characteristic is his reply to the question of one spiritual daughter about how many sorrows had fallen to his lot. ``I didn't feel the sorrows I had to bear during the Elder's lifetime, but now...'' and from emotion he could not finish.

The foremost and heaviest burden that lay on the Elder Joseph was, of course, the orphaned Shamordino Convent. Unfinished and unprovided for, it experienced hard times. Since there was much agitation and confusion at that time, Fr. Joseph saw and experienced much grief. But by his patience and humility, he always came through victorious. The enemy of the human race did his work, and many spiritual people demonstrated their human frailty, but Fr. Joseph alone remained invulnerable and imperturbable. With the help of God he was able to maintain good, sincere relations with everyone who directly or indirectly was unfriendly to him, and never did he speak a bad word about anyone. In the end, these qualities caused everyone to recognize his spiritual superiority, and inspired great respect for him.

During the Elder Amvrosy's life, Fr. Joseph took no part in Shamordino's affairs; after the Elder's repose, however, as was said above, he felt such pity for the convent, more than he might ever have expected as he said subsequently. He took this offspring, as it were, of Fr. Amvrosy into his own hands. The Superior of the convent, a faithful and devoted disciple of Fr. Amvrosy, began now with deep faith to confer with the Elder Joseph in all matters concerning the convent, and as before, nothing was ever done in the convent contrary to his will or without his blessing. Deprived of her sight, the Abbess Evfrosynia found her only support and comfort in the Elder Joseph, and to him alone did she entrust everything that lay heavily upon her soul. This Superior demonstrated a surprising example of a good relationship with the Elder: being herself an experienced and wise spiritual Eldress, she greatly humbled herself before him, as before an Elder shown to her by God, even though she was the same age as he and had entered the monastic life eight years before Fr. Joseph. She went to him often, and wrote still more often,\footnote{Despite her blindness, she always wrote letters herself, following a ruler with a pencil.} and constantly called upon his prayerful help in every matter, even as she had called upon the precious name of Fr. Amvrosy.

Soon Fr. Joseph was officially confirmed as the spiritual father of the Shamordino sisters also, on a par with the head of the skete, the Elder Anatoly. He would go to Shamordino twice a year--during the Apostles' Fast and the Dormition Fast\footnote{During the winter, the sisters came to him.}--to hear the confessions of the Elder Amvrosy's spiritual children who had now come over to him.

For the grieving sisters, these arrivals were a great joy. They met him and accompanied him as they had done with Batiushka Amvrosy. They crowded around him and accompanied him during his inspection of the convent, and with the same gentleness and zeal did they show concern that he who with such love had taken upon himself all their burdensome worries should be free from anxiety. Batiushka Joseph became a second father to them; their needs, their griefs were painful for him, their spiritual salvation was precious.

He almost never stayed overnight at Shamordino, no matter how much they asked him. ``No,'' he said, ``although it's late, somehow it is pleasant to travel that road along which I travelled with Batiushka.'' How much love for the Elder and how much deep sorrow for him were concealed in those words!

But soon these trips stopped altogether; his weak constitution could not bear the prolonged fatigue. Having once fallen ill at Shamordino, the Elder stopped going there and did not even see the newly-built church.

His life at Optina was also toilsome; in the morning, he would go forth to his labor and, like a faithful servant of God, would remain at it until evening. From eight o'clock in the morning, he began seeing visitors. After refectory, where he went without fail when he was healthy, he would rest for a while. From two o'clock on, he again began receiving people until eight o'clock, and sometimes even later; afterward, he always listened to the evening rule read by the cell-attendants. In the summer on warm days, he walked in the forest after two o'clock, and those wishing to see him were permitted to accompany him. The people usually walked a little behind while the Elder was occupied with someone ahead. When he sat down for a rest, he would narrate something edifying. These spiritual walks with their guide in the wild forest brought to mind the ancient wilderness with its anchorites. The same cell-attendants who were with him from the beginning stayed on; soon, however, one was taken to the monastery. The Elder was left with only one for a long time until the Superior sent him another under obedience to stay there.

Generally, he was very strict with himself in his external life. Never, in spite of his weak health and the labors beyond his strength, did he allow himself any indulgence. Only in his last years did he stop going to refectory since, in general, he could not go out into the air; previously, however, he would not even light the fire in his cell in the evening. Later they used to prepare watery kissel or rice porridge for his supper. He never drank wine, not even in sickness when it was necessary for him and the doctors insistently demanded it. He would never agree to it, and those near him who begged him to fortify himself would be strictly silenced by him, who told them that it was unnecessary, and that they should say nothing more about it.

When he served in the main church of the monastery on feast days, while he still had his strength, he never allowed himself to ride, but would always go by foot, both in winter and in autumn, even in bad weather. He was also strict with himself about his clothing. For a long time he wore a shabby fur cassock which no longer kept him warm, but only burdened him. Although he was weak and sensitive to cold, in no way could he be persuaded to change it. Finally, several benefactors bought some fur and sewed a new cassock without asking him. Batiushka accepted it with love, not wishing to grieve their zeal with a refusal, but later he had the collar changed so that it would not differ from that of a simple monk.

He never refused to serve when asked. Even though some pleaded with him to refuse in bad weather or ill health, he would answer strictly by way of admonition, ``It is by this that we live; how then can I refuse?''

The Elder slept very little, and during the day he did not always lie down, but occupied himself with reading or prayer. Once, when it was brought to his attention that he was wearing himself out and depriving himself of even a little rest, Batiushka said, ``Does rest consist only of sleeping? The night is for sleeping, but it is enough during the day to be by oneself.'' When they were grieved on account of his exhaustion from labors, he usually brought up the example of Saint Theodore the Studite who, though worn out during his imprisonment, comforted and encouraged the grieving brothers and wrote directions to his flock, despite his sufferings from festering wounds. ``Such was his love,'' added the Elder humbly. He never lost time in idleness; everything was regulated. It is worthy of note that, when he had become an Elder and almost no time was left to him for rest, upon finding a free moment, he either took up a book or occupied himself with regluing old envelopes on the inside, which he then used for sending letters to neighboring monasteries. He was punctual and had moderation in everything; perhaps this is why everything went smoothly and why everyone was satisfied.

In his relations with others, he treated everyone the same to a remarkable degree. Always friendly and compassionate, he did not curry favor with anyone, did not purposely try to attract anyone, did not defer to anyone; even for high-ranking persons, he did not change his discipline. He would receive relatives who visited him the same way that he received the rest and in the same reception room; he would speak with them, but he did not offer any tea or special comforts to relatives. In general, he neither asked anyone to come to him nor did he refuse anyone.

A woman wrote to him requesting that he take her under his guidance, and she explained that much was troubling and worrying her. Not having received an answer to this letter, she came to Optina and became so well disposed toward the Elder that she opened her soul to him. When she asked why Batiushka had not answered her letter, he said, ``I was waiting for you to write me what was troubling you.''

``I had decided not to do this without your permission and thought that you would not accept me,'' objected the lady.

``Really, when do Elders ever refuse those who turn to them?'' he said to her.

Accepting everyone without distinction, Fr. Joseph always answered the questions asked him, but he would never carry the conversation. Once one woman thought, ``Why doesn't Batiushka himself ever say anything?''

The Elder suddenly answered her thoughts, ``The one questioned need speak only to answer the one who asks the question.''\trans{See \textit{Apophthegmata Patrum}, Poemen, no. 45.}

One of the monks close to the Elder related that, at first, he even used to grumble at the Elder because he was so sparing with words and never said anything without being questioned. When one went to see him, he would only say, ``Well, what?'' and one would have to ask the questions oneself.

``Why is it,'' thought the monk, ``that the Elder who is so well read in the teachings of the holy Fathers and so filled with spiritual wisdom that he could say much, always has to be plied with questions?'' Later, however, the Elder explained to him the words of Saint Peter of Damascus, who writes, ``One should not say anything profitable without the inquiry of the brethren, in order that the good be effected by free will, as the Apostle teaches, `Neither as being lords over God's heritage, but being examples to the flock.'\footnote{1 Peter 5:3} Without being asked, the ancient Fathers did not speak even of that which served for salvation, considering this to be idle talking.''

``Then,'' said the monk, ``I understood the deep wisdom of the Elder and stopped judging him. I received great profit from his short, but powerful answers; by experience, I became convinced that although another might say much, his words do not remain in the heart.''

The Elder said to another monk, probably because he also was troubled by the same thought, ``Many people are displeased with me because I say little. It is not necessary to say much in order to comfort the grieving soul; it is necessary only that the person himself be allowed to speak freely without interruptions, and when he has expressed all his sorrows, then his sorrow will be lightened by this very thing. All that is left is for one to add a few warm words with love and to make clear any perplexity; then the person becomes visibly strengthened in faith, renewed in soul, and is again ready to endure everything.''

Everyone that came to the Elder found that his short answers and concise directions were stronger and more effective than the most thorough and prolonged discussions. He knew in two or three words how to say so much, that immediately everything became clear and comprehensible. The most assured justifications of self-love and the proud excusing of one's actions were torn to shreds at one of his words, ``Well then, you have to be patient.'' By his humility, he humbled the most tempestuous hearts; such heavenly peace always wafted from him that in his presence the most obstinate, proud, and stubborn people were completely changed.

It will not be superfluous here to recall that the Elder Amvrosy would say at times, ``I give you wine mingled with water to drink, but Fr. Joseph will give you to drink undiluted wine.'' Of course, one should see in this, first of all, the great humility of that wondrous man among the elect, the Elder Amvrosy; and then it is clearly evident that Fr. Joseph, who was a reflection of Fr. Amvrosy both in life and in teaching, would differ from his teacher in the manner of his direction. Fr. Amvrosy was an educated person, and possessed a well-rounded intellectual development, he was sociable by nature, and his speech, aside from being strong in grace, was captivating by its brilliance of thought, picturesqueness of expression, lightness, liveliness, wit with joy, in which was hidden profound wisdom, both worldly and spiritual. Fr. Joseph, on the other hand, was all concentration, and his speech was restrained and breathed of the teachings of the holy Fathers alone.

He was never one for pampering and concessions. As a true monk, he was never outwardly affectionate, although he was condescending and tender. His only expression of attention and affection toward his spiritual children consisted in striking some of them lightly on the head on infrequent occasions. With those closer and more dedicated to him he was much more strict and unyielding, but by this he secured their complete, unshaken, and sincere devotion to him.

There were cases when several people who were not able to understand and appreciate him considered his response inattentive and cold. Being attracted by the external affection shown by other directors and considering this as an indispensable condition for calming the inner man, in their faintheartedness they left the Elder Joseph. Such fallings away never provoked any dissatisfaction in the humble Elder and aroused neither envy nor embitterment. If anyone who had left him returned after having understood his mistake and after having perceived his deprivation, he would again receive him with the same fatherly love, completely forgiving and forgetting all that had transpired.

During confession, Batiushka was always serious; his observations somehow penetrated especially deep and awoke a consciousness of one's sinfulness and irresponsibility in the sight of God. The very sight of him--his face illumined by some inner light, his angelic smile which reflected the purity of his soul, his half-lowered eyes through which meekness and humility gazed upon the penitent--all together irresistably acted on the soul. He was a living witness to the truth of the words, ``God resisteth the proud, but giveth grace unto the humble''\footnote{James 4:6.}; and, ``My strength is made perfect in weakness.''\footnote{II Cor. 12:9.}

One hopeless sinner, through his visits with the Elder Joseph, was renewed in spirit and changed for the better. This man was suffering from melancholy and was close to despair, for he could see no amendment in his life. On the advice of acquaintances, he started going to Optina and there, after speaking with the Elder, he received solace every time. The enemy tried in every way possible to divert him from these trips by suggesting that since he was not corrected, the trips were useless; nonetheless, he continued going to see the Elder. He finally was reconciled to his difficult situation and his inner cross; he bore his depression courageously, considering it a chastisement for his lack of correction.

One monk, sincerely well-disposed toward the Elder, always went to him to disclose his thoughts. But the enemy of our race, desiring no good thing, and hating even the voice of him who confirms one on the path to salvation, sought to bar this monk's way to the Elder by bringing upon him coldness and disbelief. The monk stopped going to the Elder. But, by the mercy of God and the prayers of his guide, he soon realized that he had been given over to a temptation of the enemy, and returning to the Elder, he said, ``Batiushka, forgive me for my frankness, but I have lost all faith in you.''

The Elder answered in a fatherly, soft, and pacifying tone, ``Well now, my son, what is so surprising about your temptation? The holy apostles also had doubts of faith in God and the Saviour, but after their disbelief they were strengthened still more in faith, so that nothing could separate them from the love of Christ.''

The monk who had been tempted immediately felt a change in his soul, and he completely subjected himself to the will of the Elder, who fulfilled the words of the apostle, ``Ye which are spiritual, restore such an one in the spirit of meekness.''\footnote{Gal. 6:1.} Batiushka used to say that someone who is under an elder, although he may lead his life poorly, nevertheless stands more firmly than one who lives without direction.

While still under the Elder Amvrosy, one of Fr. Joseph's first spiritual daughters, seeing how her guide was making spiritual progress and apparently being prepared by God for eldership, began grieving in her faintheartedness that a time would come in which she would not be able to spend so much time with him. She once told him of her fear, adding that she doubted whether she would be able to bear it. Then Fr. Joseph answered, ``Well now, one must be ready for everything; Mother Amvrosia Kliucharevaya\footnote{The foundress of Shamordino Convent.} used to spend as long as she wanted with Batiushka every day; later, however, she was happy if she came only once a week.'' Certainly, the time did come when Fr. Joseph could not speak for long with everyone. She found herself receiving spiritual benefit, peace, and comfort if she were present when he came out for a general blessing only, and she began to be more satisfied with one word from him, treasuring this more than the previously prolonged directions.

Soon after her entry into the convent, one novice was greatly troubled because of her weaknesses and lack of correction, and on this account she found no inner peace. When she revealed this to the Elder Joseph, he quickly reassured her by his firm directions, saying that so long as one is experiencing spiritual turbulence, there can be no repentance, but that only self-condemnation can bring peace.

Afterward, this same novice suffered from a difficult temptation, and she insistently asked the Elder to deliver her from it. The Elder said to her nothing more than a few short words, but with such strength and power that at that very moment she felt that she was healed and this did not again repeat itself with her.

Another of the Elder Joseph's spiritual daughters related to him that the thought that she would assuredly drown began to disturb her (inasmuch as earlier someone had foretold this to her), and she had given way to great fear and consternation. Finally, she decided to speak of this to Batiushka. The Elder struck her lightly on the head and said happily and quietly, ``Well now, if you drown, you drown. You yourself know that you won't drown, so what's there to worry about? It's all the same with God how you die.''

From that time on her fear of sudden death disappeared. ``Sometimes even one glance from him would tell me more than volumes of homilies,'' she concluded.

One novice had fallen into deep depression, and having gone to Optina, she said to the Elder, ```Life is difficult, Batiushka!' To this he answered me, `It will be even more difficult in old age, but indeed, by what can one better save oneself, if not by afflictions?' After some time I was again with the Elder and complained again saying, `I can't continue any longer. It's very difficult; I'm going to another convent.' Batiushka said, `No, don't go. They won't let you go from there to Optina, since the journey will be very expensive.' Batiushka removed my grief as though it had been a fur coat. Thus it became easy for me and I never again asked to go anywhere.''

There was not even a shadow of partiality or flattery in the Elder Joseph. Were he given a large sum of money for remembrance of the dead or were a country woman to bring him a simple towel, to the first just as to the last he would return the same thanks. ``May the Lord save you,'' a \textit{prosphora},\footnote{See Glossary, p. \pageref{prosphora}.} a leaflet, a small icon as a blessing, and a warm, fatherly word on parting. He strictly fulfilled the words of the Lord, ``Give to every man that asketh of thee,''\footnote{Luke 6:30.} and he gave to all whatever was possible.

Besides the mystical effect of his grace-filled words on the spiritual disposition of a person, Fr. Joseph had the undoubted gift of also healing bodily illnesses as well as the passions.

A Shamordino nun had been suffering from a severe illness in the stomach for a long time. Both the convent's medicinal supplies and the treatment of the Kozelsk doctors not only failed to help, but they did not even ease her sufferings. The nun decided to try the last resort, the Optina surgeon's assistant, a monk and a very experienced physician. Knowing of the nun's intention, the Abbess, who was preparing to go to Optina, suggested that she go with her. The nun unwillingly accepted the invitation, since she knew that her pains would increase because of the journey, and that she would only hamper and trouble Matushka. After they had arrived at Optina, first of all, of course, they went to the Elder. The sick nun told him she was suffering from her stomach. The Elder, smiling affectionately, remarked, ``Tell me, have you come to see our doctors?'' and going to a holy icon he began to pray; then he dismissed the nun and ordered her to go to the monastery hospital. She went and returned with a medicine vial, but then she noticed that she felt no pain. From that time her illness disappeared without leaving a trace, and the medicine vial remained untouched.

G. A. related that she had been suffering from severe headaches. Once, while sitting in the hut, she felt such a strong attack of migraine that she even became frightened. Soon the Elder came out for a blessing, and she could only utter, ``My head hurts so much, Batiushka.'' He smiled and struck her on the head. The pain was gone in a flash. When the Elder left, she told everyone there from joy, but by evening the headache had returned.

Having understood her mistake, she told the Elder the next day, and he, having admonished her, said, ``Well, be patient; it'll hurt a little. It'll hurt a little, and then it'll go away.'' Indeed, the headaches continued for a little while more and then they stopped once and for all.

Mrs. E. says of herself, ``In my youth I smoked, paying no heed to the doctors who forbade me to do so, and so it became a habit. Then, no matter how much I tried to stop smoking, I couldn't. Finally in confession I told this to Fr. Joseph and asked for his holy prayers. The Elder did not even begin to reproach me, but said gently, `Don't smoke; it's bad for you,' and then added, `Buy yourself some sage, use it instead of tobacco when you make cigarettes, and smoke it.' But this did not work for me. Since I couldn't smoke the sage, I continued smoking tobacco. When I was at Optina again, I asked the Elder to pray. `What, you couldn't smoke the sage?' he asked smiling, and at that he blessed me with the words, `May the Lord help you.' On my way out of the skete, I felt strong conscience pangs. I was so strongly affected by the fact that Batiushka treated my infirmity so gently and without the least reproach, that I stopped smoking immediately.''

The same person said, ``In the first week of Great Lent, I was fasting and was suffering greatly from pain in the small of my back. When I told the Elder, he ordered me to massage the sore spot with arnica. I did as he told me and the pain went away completely; at that time, as before, I had been trying every possible medication, and none helped. I firmly believe that I was saved from all my infirmities by the prayers of Batiushka Joseph, and I am certain that it was by his prayers alone that my whole life went well.''

A mother came with her family to Optina Hermitage to see Batiushka. Her son's foot was hurting and he could not walk. They carried him to Batiushka in their arms. After he had visited the Elder, her son walked by himself; not a trace of the illness remained.

A certain woman living at Optina became seriously ill; she requested that she be taken to Batiushka's hut. He received her, and having put a prayer-rope into her hand, he went through to the bedroom, saying, ``Wait a little.'' When he came out, she had completely forgotten about her illness.

A peasant woman A. was ill. Her thyroid glands were swollen very badly. She had been operated on twice, but the swelling had increased all the more, so that she could not even turn her neck. She resorted to Batiushka for advice as to whether she should have another operation or not. He said to her, ``There is no need for an operation, but rather have a \textit{moleben}\footnote{See Glossary, p. \pageref{moleben}.} served to the Great Martyr Panteleimon, and you will be restored to health.'' The peasant woman had a moleben served, and the swelling of her glands went away, leaving no trace.

\begin{longquote}{A novice related the following concerning herself:}
While Batiushka was still alive, I was in great grief. It was inexpressibly difficult for me, and of those around me there was not one person who was sympathetic. Finally, it came to the point where my chest became diseased, and blood came up through my throat. Nevertheless, I fulfilled my obedience with great difficulty, hiding my weakness from everyone. Hay-making time came, and I had to go out like the others, although I felt complete weakness. I lost all spirit, and in the evening fervently asked the Heavenly Queen to help me. Having fallen asleep, I saw that I was in the hut near the icon of the Mother of God \textit{It Is Truly Meet}. I was waiting, but Batiushka did not receive me. Finally, out came the reposed Abbess with whom I had been very close and I said to her, ``Matushka, people entreat Batiushka for their close ones. Why haven't you asked for me?''

``But I have entreated Batiushka for you'', she answered.

She had just said this when through the door came Batiushka himself in a pure white robe and said three times, `Christ is risen!' In my sleep I started saying `It is truly meet...' Having said it to the end, I ran toward Batiushka and told him of my grief and how my chest hurt. Then Batiushka rolled up the skirt of his cassock into a ball and pressed the sore spot on my chest with it and said, `God is merciful.' Then he embraced me, and I woke up completely well. To this day, my chest has never again hurt me. Later, having come to Batiushka in reality, I told him how he had healed me in my dream. When he had heard me out, he said, `Well, glory be to God! It went away completely? It doesn't hurt now? Well, give thanks to God!'
\end{longquote}

O.D. told of an instance where he received sudden help simply by remembering the Elder's name. Having been at the skete cottage, he travelled to the monastery under obedience to get fish for some feast day. When he had finished his business, he went to the Elder to get a blessing and then started back. On the way, his horse became frightened at something and dashed off, whereupon the swingletree fell under the trace and excited the frightened horse all the more. O.D. jumped out of the cart and said to himself, ``But Batiushka, I went with your blessing.'' The moment he called to mind the Elder's name, the horse stopped with a snort, its whole body trembling. He went up to it, set it in order, led it out onto the road, and safely travelled on.

By the Elder's prayers, the Lord often sent His wondrous help in instances where people could do nothing. 
\begin{longquote}{Mrs. M.P.T. relates one such noteworthy instance:}
My husband, a person who had little faith in God, came down with a serious nervous disorder. He went to all the famous doctors, and was treated by various advanced methods, but nowhere did he receive help.

I respected and loved the Elder Joseph very much and from the very beginning I went to him with all my spiritual afflictions. Now, of course, at such a distressing time for me, I often turned to him for both prayerful help and advice about what I should do with my sick husband.

Before I went with my husband to the doctors in Moscow, I received a letter from Batiushka ordering me to tell my husband to prepare and receive the Holy Mysteries there without fail. I thought that to relate such advice to my husband would be useless, but to my amazement he, though grudgingly, nevertheless agreed to partake of Holy Communion, and he did so in one of the small out-of-the-way churches of Moscow. After communicating, he seemed to become a little better, but not for long. At first there were severe attacks of depression, then recurring palpitations and weakness more than before. I began to fall in spirits, and my husband weakened to such an extent that he could not carry on his normal activities. Despair seized me and my mind was filled with gloomy thoughts.

But what do you think? In the month of January, in the cold weather, my husband said to me, `Let's go to Optina.' Knowing my husband's attitude toward religion in general, I could not believe the seriousness of his intention and paid no attention to his words. When he returned from work the next day, he asked me, `Well, have you prepared? We're going to Optina tomorrow.' For me it was so unexpected, so new, that in the depths of my soul I perceived in it the strength of the prayers of Batiushka Joseph.

On January 30, we left for Optina and on our arrival we immediately went to the Elder, who received my husband first, then me. What they spoke about, I do not know, but when I went in to the Elder, he said to me, `He'll be all right.' Then the Elder told us both to prepare for Communion, and on the day of the Meeting in the Temple\footnote{Celebrated on Feb. 2.} we communicated. By the prayers of the righteous Elder, my husband patiently stood through all the long Optina services and fulfilled all the prescribed rule. The Elder himself heard his confession.

After having some tea, we went to the hut and Batiushka received us separately again. He gave my husband a small icon of the Mother of God of Chernigov with Saint Theodosy on the reverse to put around his neck, and having blessed him, he said, `This is for you as a remembrance.' Then Batiushka called me and said that that day right after Vespers we were to leave for the Tikhonov Hermitage. There we were to serve a \textit{moleben} with the \textit{akathist}\footnote{See Glossary, p. \pageref{akathist}.} at the relics of Saint Tikhon;\trans{Saint Tikhon of Kaluga (or Medinsk), whose memory is celebrated on June 16.} immediately after this we were to go to the holy well where my husband should without fail bathe in the spring after a short \textit{moleben}. After having some tea, we were to depart for Orel without delay.

Hearing this, now that we were in the midst of the cold month of February, I thought that my husband would in no wise agree to do it, and I told Batiushka so. Thereupon he answered me, `Well then, ask a monk to pour some heated water from the spring over him; but it would be better to bathe. Those with tuberculosis bathe there and receive healing.' Having heard Batiushka's words, I asked for his holy prayers. In the evening, we left for the Tikhonov Hermitage.

At night there arose a heavy, cold snowstorm, and we travelled from the station to the monastery with difficulty. There we caught the end of the early Liturgy, and then we stood through the \textit{moleben} to Saint Tikhon with the \textit{akathist}. The time came to go to the spring. My husband's attitude changed, and he began to delay saying, ``Now how are we going to travel in such weather?'' In my mind I called upon the precious prayers of the Elder and, as well as I could, I persuaded and urged my husband to fulfill that which Batiushka had blessed. Suddenly, my husband decided to go immediately. The horses were ready, so I took a few more warm things, and we went to the well. The snowstorm was terrible and cold; along the way we did not meet a soul.

Having arrived and served a \textit{moleben}, my husband went with a monk to the men's bath house, I to the women's. There was no one there; the floor was covered with ice. Having immersed myself three times, I quickly dressed and went out to the court where I saw my husband already standing in a fur coat. The thought flashed through my mind, `He didn't bathe', but to my great joy, he declared that he had bathed nicely. Soon we went to the guest house where a samovar and dinner were awaiting us. My husband was happy and peaceful; he found everything very tasty, although during the whole time of his illness he had had no appetite for anything.

According to the Elder's blessing, we departed the same day for Orel. From that time on my husband visibly began to recover, so that now, by the prayers of the Elder, he is completely well and carries on his work as before.
\end{longquote}

In the case described above, the strength of the holy Elder's prayers is seen quite clearly; for who could bring about a change in the soul of a person of little faith and, without any external influence, make him submit willingly to the extremely difficult demands of a simple monk? On the other hand, in this incident we also see the deep humility of a true Elder. Fr. Joseph, like his teacher the Elder Amvrosy, never manifestly revealed his gifts, but always tried to conceal himself by sending the sick and suffering to some holy place.

There are so many cases which clearly revealed his gift of clairvoyance, the fruit of his humble soul which was illumined by grace, that it is impossible to relate them all. We will relate only the most noteworthy ones here.

A certain gentleman who served on a private railroad wanted to change his position for a more profitable one. Having referred earlier in such important cases to the Elder Amvrosy, he was very grieved that now there was no one from whom he could confidently ask advice. After some hesitation, he decided to take the suggested, profitable position, and he said to his wife, ``Now, if Batiushka Amvrosy were alive, I would ask him what I should do, and I would be at peace.''

His wife said to him, ``Let's go to his disciple, Batiushka Joseph; he's the Elder now. Ask him.''

``Why should I go to Fr. Joseph? I don't have any confidence in him, so I can't ask his advice.''

``Well, if you at least get a blessing for the new position it will be better,'' said his wife.

``Well, if it's just to get a blessing,'' he agreed, ``but ask him I will not; I've already completely made up my mind.''

They came to Optina and went to the grave of Elder Amvrosy where the man even began to weep from grief. His wife finally said to him, ``Let's go to Batiushka Joseph now.''

``Let's go,'' he said, ``I'll get a blessing and go, but you can stay on if you like.''

The Elder soon received the visitors, and the man began to speak first about the good position offered him. ``It's useless,'' the Elder said to him, ``I suggest that you do not change. Here, you can receive bonuses and earn a pension; therefore, it will be more advantageous than that job they are offering you.''

``But what kind of pension can there be?'' objected the gentleman, ``After all, I don't work for the Tsar.''

``No, wait a little bit anyway,'' concluded the Elder firmly.

The gentleman left the Elder in great agitation and confusion, and he said to his wife, ``What should I wait for? I'm just letting an advantageous position slip by.''

But his wife talked him into obeying the Elder and waiting a little. ``If you let this position go by, God will send another,'' she said. Finally he agreed to wait.

And what do you know? After a short time this railway started working under the Tsar, and the work became governmental; the man received bonuses and a right to a pension. The words of the Elder turned out exactly; this position became more advantageous than the one they had offered him. Afterward, the gentleman made a special trip to visit the Elder to give him his sincere thanks.

A rich lady of Kozlov, a landowner, had read with enthusiasm the life of the Elder Amvrosy. She suggested to her two daughters, who had just finished the Institute, that they go to the Optina Hermitage. The younger daughter willingly agreed, but the older one did not want to since they were expecting a large group of guests and young people to come to the estate, and she preferred to spend the time gaily. The mother tried to convince her, but the daughter begged to remain home with the governess.

The landowner went to Optina with her younger daughter where they both liked it so well that they wanted to stay a little longer. The guest-house keeper advised them to go to the Archimandrite and then to look at the unoccupied quarters. Upon coming to the Elder for a blessing, they told him that they liked it so much there that they wished to extend their stay. But Batiushka answered decisively and seriously, ``No, you should go home right away.'' This answer surprised them very much and they were puzzled why Batiushka should act so inhospitably toward them, even to drive them out. The next day they again went to the hut. The Elder, coming out for a blessing and seeing them, said sternly, ``What! Are you still here? Leave right away!'' and as he turned around quickly to go back up the steps, he added, ``-- or else you might not even find the coffin there!'' Many in the hut heard these words which made a profound impression on them all. The landowner finally became very alarmed and not paying any attention to the arguments of the guest-house keeper, immediately collected her things and left.

On approaching the house, they were surprised to see a large concourse of people on the porch; in the doorway, a coffin soon came into sight. It turned out that the older daughter, who had refused to go to Optina, had gone out riding, and was killed when she fell from her horse.

Mrs. E. I. T. related that, after the death of her sister, the wife of a priest, and his subsequent departure from European Russia, she had fostered their five children. In July, 1898, she took the eldest nephew to the Elder Joseph and asked him to pray for him and to make him listen to reason since he did not want to finish the seminary, but persistently strove to join the police in order to be independent of his aunt. The Elder, seeing his stubbornness, did not oppose him, but advised him to spend one more year in the seminary and then to try for police service. The nephew agreed to set aside his decision for a year. During that year he changed his mind about abandoning studies, and he satisfactorily finished the course.

A certain family was celebrating Pascha at Optina Hermitage. After Vespers, they went to the Elder to say farewell since a hired carrier was supposed to come for them. However, Batiushka said to them, ``No, stay, and tomorrow attend Matins and Liturgy; it will be a solemn service with Archimandrite Agapit serving.'' They answered that it was in no way possible for them to remain since the carrier had been hired, and he would not agree to wait for them until the next day. But the Elder again said, ``No, he won't come.'' Returning to their room, they waited for the carrier who, in fact, did not come, and so they stayed. It turned out that there had been a storm on the river, and the carrier could not get across.

Hieromonk Savva (now deceased), who was suffering from obesity and an ailment of the legs, found difficulty in going to the Elder for confession. Finally, he went to him with the intention of getting a blessing to confess to some other confessor in the monastery. At this time, Batiushka's confessor, Fr. Melchisedec, had died. When Fr. Savva went in, the Elder met him with the words, ``I wanted to send for you; Fr. Melchisedec has reposed, so be my confessor.'' Not having expected this, Fr. Savva could say nothing; later he told others that he had wanted to leave the Elder, but the Elder, as though seeing through to his thoughts, forestalled him by the request that he himself become his confessor.

A woman who desired to see Fr. Amvrosy decided to go to Optina, but those at home tried to talk her out of it, for they feared the distance and the difficulty of the journey. In October she read in the news paper about the repose of the Elder. Grief and regret took hold of her, for by her faintheartedness she had deprived herself forever of the blessing of such a great Elder. Then she saw in a dream that Fr. Amvrosy blessed both her and her sister with small enameled icons. She did not remember what they portrayed, but only that on one of them there were many faces and on the other a few. Afterward, she wrote to an acquaintance in Shamordino and asked her to somehow obtain from the Abbess some of Fr. Amvrosy's remaining icons for her as a blessing. By this time, however, all the small items had been given out, and there was nothing to send her. Then during a visit to Optina, her Shamordino acquaintance told Batiushka Joseph about this. The Elder said, ``Go to the shop and ask them to give you the box with the enameled icons previously purchased. While crossing yourself, put your hand into the box and take the first icon that comes to you as a blessing for the handmaid of God Sofia, and the second one for her sister.'' The acquaintance did as directed and was amazed when she saw that the first image was of the Kiev Saints (many faces) and the second, the Synaxis of the Archangel Michael (few faces). After having first placed these icons on the grave of Fr. Amvrosy, the Shamordino nun sent them to her acquaintance. She soon received an answer from her, full of gratitude and surprise, since the icons turned out to be exactly the ones she had seen in the dream. Moreover, her husband was transferred after a short time to work in Kiev, where at first they lived near the Lavra, then later in the vicinity of St. Michael's Monastery.

A person of independent means who was very devoted to the Elder settled in the vicinity of the Optina Hermitage in order to be nearer to Batiushka Joseph. Once she was troubled and overcome by the thought that she was bothering the Elder in vain, and that there was no reason for her to live there. Having decided in her soul that she was going to the hut for the last time, she came to the skete. The Elder received her for a blessing; just as she was about to tell why she had come, she was amazed by the sight of rays of light flowing from the Elder's eyes. She was dumbfounded, but the Elder smiled gently, struck her on the head, blessed her and said, ``Compose me better confession booklets.''\trans{These were small booklets with a listing of various transgressions and sins, and also a space provided for writing notes to aid in the confession of one's sins.} The Elder's face had an expression as though he were saying, ``Well, do you understand?'' Then he extended to her a booklet entitled Blind Faith and Dead Faith and said, ``Here, read.''

The same person related that when she had not received any letters from her son for a long time, she asked the Elder whether he had died and how she should commemorate him. The Elder answered, ``It makes no difference how you commemorate, just commemorate.'' From this answer she concluded that her son had died, and soon she did receive news of his death. She was stunned: she did not want to weep, or to pray, or even to go to the Elder, but considered her life finished. Going almost out of her mind in her grief, she somehow wrote to the Elder about her condition and sent her letter with the monastery purchaser to Optina (she lived in Kozelsk). At 4:00 in the afternoon, her despair suddenly passed, and she felt a lightness in her soul. At that very hour the Elder had gone out of the hut for a blessing and, seeing her cousin, he said, ``G. writes that her Lelya has died.'' Undoubtedly, the Elder, having received her grief-filled letter, prayed for her and the strength of his prayer saved her from despair.

The nun V. E. was subject to depression and despondency. Once she went to the Elder and, having told him that she could not rid herself of her oppressive anguish, she asked him for permission to go to her brother in America. He occupied a prominent position there, and she had not seen him in a long time. Batiushka said, ``No, it's better if we take Communion a little more often; what do we want in America? We could set off for someplace closer and even so not arrive! After all, there's only a little time left for us to live.'' This nun, although she had always feared death, heard these words from the Elder very calmly and began to prepare herself for her passing into the next world. It was August, 1910; then in November of that same year, she went to the neighboring men's monastery for Vespers, and there in church, she suddenly died.

Just before her death, she had dismissed her cell attendant, but later wanted to take her back again. The cell-attendant did not want to go back, but Batiushka said, ``Just go for the time being.'' It so happened that she only had to live with the nun for a few weeks.

In B. Convent there lived two sisters. One of them died. The remaining one asked to go to Optina to consult with the Elder, for she wanted to share her cell with someone else. The Elder did not give her a blessing, but said, ``Live for a while by yourself. When you become ill, go quickly and ask the Abbess to tonsure you into the \textit{mantia}.'' She was extremely surprised by the words of the Elder, since she felt completely healthy. Time passed and everyone had forgotten the incident. Suddenly she became seriously ill with a fever. Remembering the words of the Elder, she went to the Abbess and asked for the tonsure. The doctor considered her case to be hopeless. They tonsured her, and on the fourth day she quietly and peacefully passed away during Bright Week.

Three people came to Batiushka; to one of them, the healthiest, he brought out a bundle and said, ``Take it, it may prove useful for something.'' Upon coming to her room she unwrapped the bundle; in it there were various rags. They laughed over the gift and soon departed. In exactly one year, the one who had received the strange gift was afflicted with sores, and all the rags were used as bandages.

E.V. related that, owing to lack of funds, she wanted to return to Belev from Optina on a freight train. Batiushka said to her, ``You won't have to,'' and he gave her fifty kopecks. At the station she had to let three trains go by because they would not take her. If it had not been for Batiushka's fifty kopecks, she would have had to go by foot.

A novice of Belev staying at Shamordino was always asking the Elder to let her stay for good since it was difficult for her to live in that convent where everyone had to meet her own expenses. But Batiushka said, ``You have to get by at Belev somehow.'' Although a nun agreed to take her into her cell, she still did not have the wherewithal to live. The novice became grieved and again asked to go to Shamordino because, she said, there was no way and no place for her to live in the Belev Convent. The Elder again declined, saying, ``Wait a little, and you will be taking money to put in the bank yourself.'' Sure enough, by the prayers of the Elder, God sent her benefactors who began to send her money, and little by little she set money aside in a savings account.

M. E. M. relates that in 1899, a large building for living quarters was being built alongside her cell. Somehow a mistake was made in the measurements and the new framework went right up to her wall; therefore her cell was completely darkened. She went with her grief to Batiushka Joseph who calmed her saying, ``Well, don't grieve, the Lord will arrange a yet better one for you; you will have a large, bright cell.'' She thought that Batiushka was speaking of her death, but soon the Abbess gave her another cell, larger and brighter than the previous one. But here new sorrows arose, for the cell was dilapidated. After fixing it a little, the carpenters said that it would hold out for about five more years. Having gone to the Elder, she told him everything. Batiushka said, ``In vain have you repaired it. You've spent money to no purpose; it won't even hold out for a year. Underneath, all the beams have rotted; it's being held up by only one.'' Then he explained to her in detail how to fix the cell. ``Batiushka, where will I get the money to fix it?'' she said.

``Take this for now,'' answered the Elder, handing her ten rubles from the table and adding, ``Borrow some more from someone else. When you have enough money, return it.''

Not daring to contradict the Elder, she used the money to fix the cell whose beams, as the Elder had said, were so completely rotted that she could have knocked them down herself. After the repairs, since she had no money and no hope of receiving any, she became very grieved that she was not repaying her debt. While at Optina, she asked the Elder's forgiveness that she again had not brought the money. The Elder said, ``Soon you'll have some, then give it back.'' Soon after she received a notice that she had received one hundred rubles. She couldn't understand from whence the Lord was sending it to her. It turned out that her uncle had died; he had never helped her while he was alive, but now his wife had sent this money from him.

The family of a landowner came to Optina. The father did not acknowledge elders, and he would go to Batiushka Joseph for a blessing only, never conversing with him about anything. When his daughters, who respected the Elder very much, came to him, Batiushka began to tell them that the bricks prepared for the construction of a bell tower on their estate were worthless. The young ladies, who had just come from their home and had heard nothing about it there, were very surprised. They thought that their father had undoubtedly discovered it just before their departure and now, contrary to his usual custom, had told Batiushka of his misfortune. Upon coming to the guest house, they addressed their father, ``Papa, you didn't tell us anything about our bricks being ruined''.

``What bricks? I know nothing''.

``But surely you must have told Batiushka about it. We heard it from Batiushka.''

The landowner became angry and said, ``You know that I never talk about anything with your Batiushka. I got a blessing and left.''

When they returned home, they found out that their contractor had become drunk and had allowed the bricks to fall; thus they were all broken.

Another woman came asking for a blessing to renovate the home on her estate. The Elder inquired what she intended to change around and where. She began to explain, but Batiushka sketched a plan on the table with his finger saying, ``Here you have the entrance, and here the dining room, and here such-and-such.'' The woman was so engrossed with this sketch, that only upon leaving the Elder did she begin to wonder how Batiushka could know the arrangement of her house so well.

Before the coronation, a person said at the general blessing, ``Batiushka, what great celebrations they are preparing!''

And the Elder answered, ``Indeed, the celebrations are being prepared—if only there weren't going to be such misfortune!''\trans{Great multitudes of the common people had come to Moscow to witness the coronation and to share in the joy of the occasion. The day following the coronation, arrangements had been made at the Khodynka Meadow for 100,000 people to receive a commemorative mug, some sausage, sweetmeats, and sweetcake. This field, the only area large enough to accomodate the expected crowds, was used for training soldiers and was crisscrossed with a series of shallow trenches and ditches. Over twice the number expected had gathered during the night. Early in the morning, when as yet few police were on hand, a rumor swept through the multitudes that the arrangements were insufficient for the number of people who had come, and the crowds began to press forward in their eagerness to receive the presents. Many stumbled in the trenches, fell, and were then trampled as those behind them were pushed onto them by the masses of people who continued to press forward, not realizing the disaster which was taking place. S. Oldengburg relates that in all, 1,282 died and several hundred were wounded.

The emperor and empress were deeply moved at this tragedy and spent the day going from one hospital to another. They provided for the burial expenses of those who had died and gave 1,000 rubles to each of the families of the dead and wounded.}



\begin{longquote}{One highly respected priest relates of himself: }
In the course of more than ten years, I had the greatest comfort of enjoying the advice and direction of the late great elder, Batiushka Joseph. If sometimes in my personal, family, or pastoral life something happened which was difficult, confusing, or sorrowful, right away, as a child to a tenderly loving mother, I would quickly turn to my father and benefactor, Batiushka Joseph. By letter I would convey my spiritual difficulty and would ask for his holy prayers. Even before receiving an answer, I would become lighter in spirit. And when I received an answer, I would be completely at peace. Twice I visited Optina and saw the Elder personally; while there conversing with Batiushka, I enjoyed and experienced moments of my life which are unforgettable. I remember how it was the first time. The very sight and locality of Optina Hermitage was so beautiful, that I was involuntarily drawn to it and it made a delightful impression. I liked everything there, but most of all Batiushka Joseph. Belabored by years and by exertion, the comely Elder sat on his bed with a welcoming smile. I was not able to speak with him peacefully and at length, for just before coming to him, I had received a telegram demanding that I go home. Since I was travelling with my wife and son and had left my sick daughter at home, I thought that she was going to die. But Batiushka calmed me by saying, ``It's nothing, don't get excited; Lord willing, everything will be all right.'' Having received a blessing, we started home. In the train car, we received a second telegram urging us to hurry home. Again, we began to worry about our daughter. Having arrived, we indeed saw that there was no hope of her recovery. But by the prayers of Batiushka, there was a sudden change in our daughter's illness, after which she became well.

After two years, we were all at Optina again, and this time our daughter was with us. We prepared for Communion, received Holy Unction, and on several occasions were deemed worthy to visit the Elder who gave all of us small icons and various pamphlets and booklets as blessings. He blessed me with a beautiful icon of Saint Ambrose of Milan and said, 'May this hierarch be to you an example of faith and zeal for the salvation of the spiritual children entrusted to you.' This blessing was precious to me. When I look at it, I always recall the meek, much-loving Elder. Having blessed me with the icon, the Elder clarified my misunderstandings in pastoral practice. The trouble was that people would sometimes come to our convent with such spiritual burdens that I was put in a difficult position, not knowing how to act, since I did not know whether to let them take Holy Communion or not. But having referred all my perplexities to the Elder and having received directions, I was put completely at ease.

By the prayers and blessing of the Elder, I do as he blessed me to do. But I am grieved, for it is difficult to be deprived of such a benefactor, irreplaceable guide, father, and teacher, who comforted and helped me always and in all the circumstances of my life. But, the will of God be done! I believe and hope that he, standing before the throne of God, prays for us, his orphaned children.
\end{longquote}

\begin{longquote}{The nun L. relates of herself: }
On August 2, 1908, my cell-attendant and I went to Optina Hermitage. We especially wanted to visit Batiushka Amvrosy’s grave. We also wished to receive benefit and edification from the Elders, and were particularly attracted to Batiushka Joseph. After we arrived at the Hermitage, something incomprehensible happened to me. I felt slightly ill and simply could not go to the Elder. Finally, giving in to the arguments of my cell-attendant, I went. On the way I kept on thinking, ``Well, what am I going for, they say that there are crowds of visitors and the Elder sends them all to Shamordino or to the Tikhonov Hermitage. I can't do it; I have no money and bathing is bad for me. I can't confess to the Elder; I'd tire him. I'll receive Holy Unction without it.'' When we arrived at the Elder's, I stood silently looking at him. Suddenly he said quietly and affectionately, ``Well, what are you here for? Look at the portrait of Elder Amvrosy. Here is his bed, here is his whole cell; he also received visitors just as I do, sick and sitting on the bed.'' The icy shell simply fell from my heart; it became so easy for me. I was no longer silent; one thought after another slipped off my tongue. I would not succeed in saying a whole thought before the Elder would give me an answer. Then he again said, ``Why were you afraid to come? True, you have no money to travel to Shamordino and to Tikhonov, and you can't bathe. Do you want to see the Mother of God?'' The Elder answered my thought. ``She Herself will be in your cell today. I cannot confess you; I have not confessed anyone for a long time, but go to Father Sergei. He will confess you and give you Holy Unction.'' My cell-attendant was also comforted. We wept from joy. We went to the monastery and saw that a procession was taking place with the Kaluga icon of the Mother of God, which, as it turned out, had come from Kozelsk. They soon brought the icon of the Mother of God to our room also. The next day, having partaken of Holy Communion and received Holy Unction, we hurried to Batiushka Joseph. He met us with the words, ``Well, truly how good, how easy, how joyful it is! You took Communion and the Lord rejoiced. Indeed, now, the mercy of God has come upon you!'' And the mercy of God certainly did come upon us, although the Elder did not know of it; only our confessor knew. The Elder said briefly, but warm-heartedly that he did not want us to leave him, but we had to give place to others. The next day, having bought prayer ropes, and a portrait of the Elder, and several leaflets of spiritual content, we headed for Batiushka. Upon coming to him I had just thought, ``How nice it would be if Batiushka would give me a prayer rope from his own hands,'' when the Elder, having smiled, quickly took his prayer rope from his own hand and hung it on mine, and at the same time he took my prayer rope and put it on his hand. Tears gushed forth in streams, and I could only say, ``Dear Batiushka, I didn't expect this!'' The Elder suddenly took the cap off of his head, turned to my cell-attendant, straightened his hair, smiled, and put it back on again. My cell-attendant started sobbing, threw herself at the feet of the Elder, and thanked him for something. Batiushka also blessed her with a prayer rope. I was puzzled as to what it all meant. It turned out that my cell-attendant had been thinking, ``Our picture doesn't resemble him; if only Batiushka were without his cap.'' On the picture we had bought, Batiushka was not wearing his cap.

The next day we came to bid farewell to Batiushka. Not wishing to tire the Elder, we remained in the hall waiting for him to come out. Soon the door opened and Batiushka came out. But what a sight he was! His face was literally radiating light. He was so white and youthfully radiant that we were startled and lowered our eyes. Batiushka glanced at us and his was such a tender, grace-filled glance that it is impossible to describe. We will never forget it. He swiftly turned from us so that we did not succeed in bidding him farewell. We had to come the next day to bid farewell to the Elder. Afterward, we were told that on that day Batiushka had received Holy Communion.

We were with the Elder again in 1910. The whole conversation had the same character as the last one. Everything was examined in detail and foreseen by the Elder. Again he postponed my trip to Shamordino, saying that we would yet go at another time. And it certainly did turn out that we did go. When parting with Batiushka, we asked if he would bless us to visit him again. He meekly and happily smiled saying, ``Well, why not, come again. God knows, God knows.''

These lines are true; the Lord is my witness. But now, this luminary, this wonderful crystal-pure soul, is not among us, and my heart is painfully crushed.
\end{longquote}

\begin{longquote}{From the notes of E.: }
I found it very difficult in the cell in which I was living because I felt oppressed there. I went to Batiushka for a blessing to move to another cell but he said to me, ``Be patient. You will live in good cells but will not have peace of soul; you will be as though in hell.'' I did not obey and I left this nun. However, I did not find peace and I moved back to the first one. Despite her fatal illness (she had cancer of the breast) which filled the whole room with a stifling smell, I experienced neither sleepless nights, nor the burden of work beyond my strength. By the prayers of Batiushka, I endured everything with good spirits. I lived with this nun to the end of her life and buried her.

After the death of this nun, I was placed with another elderly nun, with whom it was also difficult. I went to Batiushka. He asked me with a smile, ``How are you getting along?'' I began to weep. Batiushka said ``And how did her sister put up with her?''

``Batiushka, her sister went on pilgrimages often, but I am always with her.''

``Well, may the Lord help you,'' said Batiushka ``Ask the abbess to take you out.'' Then I remembered the words of Batiushka, ``Be patient,'' and truly, I have much need of patience.

Once I was sitting in the hut; a long time had passed and still the Elder had not taken me so that I could speak with him. I ran out of patience and thought, ``Now I will go to Batiushka Anatoly (Zertsalov).'' Batiushka perceived my thought.

Right away he came out and came to the farthest hut where I was sitting and said, ``Here, take these ribbons and sew them to the \textit{paraman}\footnote{See Glossary, p. \pageref{paraman}.} or else you'll get bored.'' I was astonished by the Elder's clairvoyance. I recognized my sin, ``Forgive me, Batiushka. I wanted to go to another hut to Batiushka Anatoly.'' Batiushka said nothing, but only smiled. From that time I would always be patient with a good heart, whereas in other huts I could not even wait five minutes.

Once I was very sad and went to Batiushka. He received me tenderly like a father and I said to him, ``Batiushka, it seems that I won't see you again because my father died and my means are scanty.''

Batiushka said, ``And it's very likely that now you won't have to see me.'' My heart was breaking. I thought to myself, ``My God, surely Batiushka will get worse and leave us.'' He nodded his head in affirmation. Soon Batiushka did pass away; I had visited the Elder when he was scarcely alive, and close to death.
\end{longquote}

Still another instance of the Elder's clairvoyance: ``I had a brother who was insane. We were barely able to bring him to Optina. I went to Batiushka and told him of my woe and Batiushka said, `Don't worry, all will soon pass.' And to be sure, at the present time he is completely well.''

To the question of one nun, ``What can I do to help my relative who drinks vodka to excess?''

Batiushka answered, ``Be patient. He will soon stop drinking by himself.'' Six months did not go by before her relative passed away.


\begin{longquote}{The nun M. S. said of herself:}
I lived with a sister for a very long time. I went to the Elder, who asked, ``How are you two getting along?''

I answered, ``Sometimes it's all right, so something comes of it.''

Batiushka said in a low voice, ``Most likely you'll be sad when they separate you.''

Then I asked Batiushka, ``Pray, Batiushka, that we live together until we die.'' In an undertone he said, ``Not for long.'' And what do you know... they transferred her to a school in a short time.

I received a letter from an acquaintance in Sevastopol saying that she had taken to wine-drinking heavily. When I was at Optina, I explained to Batiushka about her, but he did not believe it. I showed him the letter. Then Batiushka said, ``Well, if she drinks, then we'll have to serve a \textit{moleben} to the Saviour, the Mother of God, and the martyr Boniface.'' He gave her a little print and said with assurance, ``She will not be sick with this malady much longer''. In a short time she actually did stop drinking and sent a letter of thanks.

A lady wrote to me to ask Batiushka which of the three suitors who were proposing to her she should marry. She did not care for the first and third, but she liked the second. Batiushka answered, ``Marry Nicholas, the third one, or else things won't go well.'' I was amazed, for she had not written the suitors' names to me, nor had I said anything to Batiushka. I wrote Batiushka's answer to the lady. She wrote a second time saying that she preferred the second suitor. Batiushka again said, ``In my opinion it would be better for her to marry Nicholas, but if she doesn't obey -- as she pleases.'' She obeyed and married Nicholas and to this day they live together very happily. The one she had liked had an accident and drowned while crossing a river.
\end{longquote}

A lady asked Batiushka at a general blessing, ``What should I do? I am very grieved for they say that my daughter has died in a foreign land.''

Batiushka said, ``Who told you? No, it's not true; she's alive. Pray for her health.''


\begin{longquote}{A novice of Belev relates:}
I was living peacefully with the nun A., but I wanted to live with the nun A. S.. I came to Batiushka and said, ``I can't live here. Bless me to leave this nun.''

``And with whom do you wish to live?'' asked Batiushka. I said that I wanted to live with A.S.

``Well, wait a little. The time will come when you'll live with her.'' Now I am living with the one I wanted to, and by the mercy of God, we live very happily together.
\end{longquote}

\begin{longquote}{The rassophor novice M. says of herself:}
I entered the convent with a sincere desire, but I was very poor. There was absolutely no way I could support myself. After thinking, I said to the nuns, ``I'll go into the world, get a job, earn some money and then come and put on black.'' I had lived almost a year in the convent and had gotten accustomed to the life, but I saw that without means it was impossible to continue there; nevertheless, I left this decision up to Batiushka. We went to Batiushka and all three of us went in to him.

He said, ``Here come to me two nuns and one laywoman. This laywoman has been living in the convent for a long time. The sisters say to her, `Put on the monastic habit.' But she can't afford to put it on, so she says, `Well, I'll go into the world, earn myself some money and then put it on.''' We were all amazed. It was as if the Elder had overheard our whole conversation.

The elder nun asked Batiushka, ``Well then, Batiushka, will you bless her to go or not?''

The Elder struck her with his prayer rope and said, ``No.'' There the conversation ended. When I was alone with Batiushka, I explained my situation to him. He said, ``Don't go. God will send you a job and you will live well in the convent.'' And thus it came to pass. I was taken in as a cell-attendant and I lived in a cell at the convent's expense. Then Batiushka foretold to me that I would live with Mother K. as a cell-attendant. After I had lived for three years in my first situation, I went home. When I came back, the Abbess blessed me to move in with Mother K., who received me like her own sister. Thus the words of Batiushka turned out to be true, and to this day I am living very well.

I will also tell about my sister. I came with her to Batiushka. My sister wanted a blessing to buy a house. Batiushka said, ``If you buy a house, who will live in it?'' And beyond all expectation a change came about. Her husband bought seven \textit{dessiatinas} of land\trans{This is 18.9 acres. A \textit{dessiatina} is 2.7 acres.} and started planning to build a home, although the sister did not want to. She went to Batiushka to ask about this matter. Batiushka said, ``You don't have to,'' and he turned to the window. His face became sorrowful and serious. He did not hide this from me. Suddenly, the Elder said abruptly, ``You don't need to; as long as you're receiving a salary, that is enough.'' Subsequently, my sister's husband became an alcoholic and he was dismissed from his job. My sister divorced him, went to Tashkent, and got a job.
\end{longquote}

A certain novice had a relative who feared for his own sick mother and sister, that in the event of his death, his brother-in-law with whom they lived might maltreat them. He went to Batiushka to get a blessing for his sister and her husband to separate. Batiushka answered, ``No, you don't have to. It sometimes happens that a healthy tree falls, while a rotten one creaks.''\trans{There is a Russian proverb, ``A creaking tree lives two centuries.'' It is often said of people who are chronically ill.} As it happened, the healthy brother-in-law died and the sick relatives survived.

 
\begin{longquote}{From the notes of O. M.:}
By the unfathomable judgments of God's providence, it was very difficult for me to live in the convent. I had to endure everything: persecution, slander, and poverty, although I was the daughter of rich parents. I was very grieved and despondent. Moreover, I had been there a long time and they had not yet tonsured me rassophor. Owing to this grief, I went to Batiushka for a blessing to move to another convent. I explained everything and Batiushka said, ``Don't worry. There will be a new abbess. She will tonsure you and comfort you.'' But since a terrible depression was tormenting me, I went home. I stayed there for a long time because I became ill; I was having heart attacks. Meanwhile, our abbess died. Expecting a change, I decided to return to the convent. The Elder's words came true exactly as he had said. The new abbess tonsured me rassophor and was like my own mother.

For ten years, my brother's whereabouts were not known. I worried about him very much. I was afraid that he might have become an unbeliever, and I thought, ``Is he alive or not?'' I did not know how to pray for him. I turned to Batiushka with this grief. He, in his deep humility, would say nothing directly, and he pretended that he knew nothing and said, ``Who knows. People go to places far away, get set up there, forget about everyone.'' He added resolutely, ``Pray for his health. Maybe you'll hear that he's alive.'' And so it happened. After my father's death, they started making inquiries for him to settle the inheritance. And what do you know? To the amazement of all, it turned out that my brother was alive and well.
\end{longquote}

One Belev nun was suffering from hemorrhages. She went to Batiushka. Batiushka said decisively, ``Go to Moscow and have an operation.'' She declared that she did not have enough money, and she was very surprised at the Elder's answer. After an hour, a nun from Moscow came to Batiushka. After leaving him, she gave the nun from Belev the address of a hospital she could go to. The sick one went; she was received free of charge and had the operation.

Another nun buried her aunt and grieved over her greatly. She went to Batiushka and told him of her sorrow, and Batiushka said, ``Well, what can one do? In the summer you'll go home; stop at Kiev and pray there.'' She said that there was no way that this could happen. But beyond all expectation, it did happen. There were some people who took her free of charge.

A Belev novice who was a chanter was dismissed from the choir for some misdemeanor. Being very grieved that she was without an obedience, she asked a blessing from Batiushka to request an obedience for herself. Batiushka answered, ``One does not ask for an obedience; wait until it is given to you. Better go a little more often to Matins. They will get a good look at you and give you an obedience.'' After a short while, she was given an obedience.

The same novice went to Batiushka for a blessing to fix up her cell at her own expense. The Elder did not give her a blessing but said, ``It's not necessary; you won't have to live there.'' Soon she was moved to another cell.

A lady, one of Batiushka's spiritual daughters, came down with stomach problems. She went to Batiushka and said that because she was being treated, she ate dairy products on fast-days. Batiushka said, ``You can be cured, but you have to eat lenten food. You'd better be concerned about your throat; have it treated.'' The lady never thought that her throat was diseased; she had noticed nothing special, except that once her voice dropped. She went to the physician and showed him her throat. He found that if it had not been caught in time, it could have developed into throat consumption. It was three years before her throat was completely healed.

A nun received a blessing from Batiushka to go to Saratov. When Batiushka gave her the blessing, he said, ``If you have some misfortune, you'll have to endure it patiently.'' She went and had a pleasant visit. On the way back she thought, ``Well, what now? Batiushka said I would have to endure something, yet how nice my visit was.'' She had not quite reached the convent when all her baggage was lost.

One poor nun bought an inexpensive cell. She fixed it up and went to Batiushka with her joy. The Elder said, ``Now why did you fix it up? All the beams in your cell are rotten; it will soon fall down.'' The nun began to assure him that it did not seem so, but Batiushka said, ``Nevertheless, look at it, or else it will fall.'' Having returned home, the nun summoned the carpenter. It turned out that the cell was being supported by little. The beams had all rotted.

Someone relates: ``The first time I came to Optina Hermitage was in 1906 with N. N.. Batiushka blessed her to stay at Optina for good. I asked to stay with her. Batiushka said, `No, this winter you will not live together.' But N. N. was not able to stay at Optina since they had not given her leave from Moscow. We decided to live together in a clinic. I had already doubted the certainty of Batiushka's words and I thought that we were going to spend the whole winter together. But suddenly N. N. became ill, and she was taken away from me to another clinic. Thus, Batiushka's words turned out to be true.''


\begin{longquote}{The nun A. G. relates: }
I had an uncle who was old and a bachelor. He was very religious and he always fasted and prayed, but he did not acknowledge elders. Once when he came to see me, I tried to persuade him to go to Optina with me. Uncle agreed and we went. I explained to Batiushka that I had come with my uncle. Batiushka took him right away, blessed him somewhat slowly, looked him over silently, and said, ``Well, old man, how old are you?''

``I've been living seventy-five years,'' Uncle answered.

Batiushka again looked at him intently and silently for a minute and said, ``Well, now it's time to die; arrange your affairs, you have to prepare for death. It's time.''

Uncle had always thought of death and spoken of it unfearingly, but this time he was obviously gripped with fear and could barely say, ``Yes, it is an inescapable road.'' Batiushka spoke of it with such certainty that I burst into tears. After all, Uncle had been my benefactor. He stayed and spoke with Batiushka but was in a very agitated state when he left. This was the third week of the Great Fast. In Saint Thomas week he reposed.

My sister, who was a pious and well-to-do woman, enjoyed good health and lived with her husband very happily. She loved Batiushka very much and had the custom of writing to him before the birth of her children to ask his blessing and holy prayers, which she did this time as well. When she received his answer, she became pensive and began to prepare for death, saying that she was certainly going to die. Everyone laughed at her. Her husband was even offended at the fact that she, a healthy young woman, full of strength, was suddenly getting ready to die. Nevertheless, she made her preparations and received the Holy Mysteries. Within ten hours after her daughter's birth, she calmly and peacefully passed away in full consciousness. They did not bury her for three days. Only after the doctor's examination on the fourth day would they bury her. The daughter is still alive. Obviously the Elder had warned her in the letter.
\end{longquote}

A. C. relates: ``The first time I came to Batiushka, I came with my husband. He gave me a book \textit{The Royal Way of the Lord's Cross}. I told Batiushka that my cross was heavy; my husband was sick, I myself was sick, and we had but small means. Batiushka fell silent. Soon after we left Optina, my husband was dismissed from his job. Four years later, I was going to take him to Moscow to be treated in the clinic. First we went to Batiushka to ask for a blessing. Batiushka let us stay at Optina until spring. At the end of this time, my husband began to think of going and did not know what kind of life to choose for himself. I went to Batiushka to ask how he would bless him. The Elder thought seriously and said, `God will bless him to stay here to finish his life.' With tears of sincere joy I prostrated myself before Batiushka and thanked him. My husband was also pleased and thanked Batiushka and made no more plans to go anywhere. In a year, he came down with dropsy, and his legs became seriously swollen. I became alarmed and turned to Batiushka and asked him to allow my husband to eat non-lenten food, though it was a fast. Batiushka did not permit it and said, `Don't worry, he won't die from this illness.' And truly, he did not die. However, in three years he came down with another illness. I went to Batiushka and he said, `Well, prepare him for death now.' In three days he died. I was very grieved and asked Batiushka for a blessing to go to Voronezh to my husband's family and also to pray at the relics of Saint Mitrofan. Batiushka blessed me, but just as I came to bid him farewell, he said, `No, don't go this year.' Two years went by. The third year, Batiushka blessed me and said, `Go, and while you're there have yourself treated.' After I arrived at Voronezh, I went to the doctor. They had a consultation after the examination and they found that I had a serious disease which would require an operation. I wrote to Batiushka, who blessed me to have the operation. I was terribly frightened. When I saw all the preparations for the operation, an especially tormenting fear took hold of me; but by calling on the name of the Elder and mentally asking for his holy prayers, the fear passed so that it became very easy for me. This fear twice again possessed me before the operation. Through Batiushka's prayers, the operation was performed satisfactorily. I was still unhealthy, but Batiushka did not bless me to go to the doctors any more. He said, `Such is your cross. Bear it cheerfully and patiently. The doctors won't help you now; but if you want to go, do so at your own risk. I won't bless you'.''


\begin{longquote}{A woman relates:}
In 1905, I went to Optina Hermitage for the first time, as if to a summer resort, to visit and rest. I did not acknowledge elders and had no desire to consult them. Once I went into the woods for a walk and met the nun M. whom I did not know at all and who was going to the skete. I spoke with her, and she invited me to go with her to Batiushka Joseph. I flatly refused, but finally, because of her persistent requests, I gave in to her and went. Upon coming to the skete and seeing so many nuns, I was again sorry. Why had I come? I did not even know what to say to the Elder. I had barely sat down when Batiushka's cell attendant came to me and said, ``Do you need to see the Elder?'' I do not know myself why I said, ``Yes, I do.'' I was immediately called in to the Elder, but I did not know where to begin. The Elder asked, ``Why have you come?'' I said that I was ill and had come to the summer house on a visit. Batiushka said, ``There are no summer homes here. You have no money and there is nothing for you to do here.'' True, I had no money. Although he did not know of my illness, he asked me straightway, ``Are you very irritable?'' I had a strong nervous disorder. I left Batiushka very dissatisfied that he did not bless me to stay here even for a little. I decided to disregard the Elder and to ask the Archimandrite. The Archimandrite gave me permission. When I sat down to write a letter to my sister, I had only written a few lines when someone struck me in the arm. I could not write anymore. Immediately the thought came that I had disobeyed the Elder. I went to the skete to ask forgiveness for my disobedience. Batiushka said, ``Here you don't obey; but I tell you: leave, or else it will be too late.'' In two days I left for Tikhonov and from there returned to T.. There I found my acquaintance V. E. seriously ill. She had ulcers in the intestines and the doctors refused to treat her. I gave her an icon of Saint Tikhon and water from the well. I do not know how the sick one prayed, but the next day she felt better. Her illness passed completely, and she asked me to take her to Tikhonov and Optina. Only then did I understand why Batiushka had said, ``... or else it will be too late.'' I had already decided to do nothing without the Elder's blessing. I wrote him, asking whether he would bless me to come with my acquaintance or not, and added saying that I did not have any money for travelling. Batiushka answered that I should accompany the sick one and that God would send money for the journey. In a short while, my family gave me some money and we went. With us there also came a hospital attendant who was getting along terribly with her husband. Her husband demanded that she turn over her salary to him, and he threatened to murder her. She refused and put her money away in a savings bank. She turned to Batiushka with this grief and he said to her, ``You don't have to give him your money. Be patient, soon all will come to an end.'' She did not understand the Elder's words and repeated her question. Batiushka replied again, ``Be patient, soon all will end.'' Soon the hospital attendant died.

In 1906, I again visited Optina. After I had returned to T., suddenly the thought came to me of visiting P. B. with whom I was only slightly acquainted. I found her terribly, helplessly ill. The physicians could not make a correct diagnosis; one said one thing and another something else, but finally they diagnosed a stone in the kidneys. The poor patient suffered both physically and spiritually. She had become an unbeliever and had not received Holy Communion for many years. She was very glad to see me. Having found out where I had spent my summer, she asked me to tell her what this Optina was like. The patient listened very willingly, especially about the Elder, and asked me to write to the Elder about her illness, if possible. I wrote, and a month went by with no answer. The patient decided to have an operation since the disease was getting worse. ``After all,'' she said, ``what can a simple monk understand about such things?'' A day before the operation, the patient drew up a will in case of her death. I advised her to take Communion, but she did not agree, saying, ``I don't want to spoil the comedy. It's all the same; I don't believe in anything.'' On the eve of the operation, the patient received a letter from Batiushka in which he advised her above all to take Communion and then to have the operation, for without Communion the operation would not help. The patient had not even finished reading Batiushka's letter when a change came about in her, and she at once called for the priest who had just given Unction to another patient. She spoke a little with the priest, but Communion was deferred until the next day since she had drunk milk on that day. On the day of the operation, the patient received Holy Communion. The operation was successful; a deep abscess had grown in the kidney. The patient was getting better when she received a letter from Batiushka with the message, ``If the patient gets weak, then I advise that she receive Holy Communion.'' This letter surprised us, since she was already preparing to be discharged. And what do you know? In a few days her healthy kidney became seriously inflamed, her whole system was poisoned with uremia, and she had strong uremic attacks. She was unconscious and her pulse was fluctuating; the physicians had given her up as dead. Another letter was received from Batiushka in which he advised that she be given Holy Unction. On that very day she felt better and slowly began to recover. Finally, she recovered completely.
\end{longquote}

A woman asked Batiushka, ``What can be the matter? Four years have passed and no one knows the whereabouts of my son.'' She already accounted him as dead.

Batiushka answered, ``Serve a \textit{moleben} to the Kazan Mother of God and pray for his health. Your son will be found.''

After a short time her son sent her ten rubles and his address. She wept from joy for the whole day.

A person who was leaving Optina said to Batiushka on parting, ``When I go back, what will I do without work? Where can I get work?''

Batiushka said to her, ``What kind of work is there for a sick person? After all, it's difficult for a sick person to work, isn't it?''

Within a week, her hands began to ache so much that not only could she not work, but she could not even hold a glass in her hands.


\begin{longquote}{The Optina novice D. says concerning himself:}
I came to Optina for the first time in 1891. The Elder at that time was the renowned guide of human souls, Batiushka Amvrosy, whom I met at the newly-built Shamordino Convent--his own offspring at the summerhouse Rudneva. I told Batiushka of my desire to become a monk. Batiushka said, ``A monk you will be, but wait until your brother grows up.'' I was afraid that my desire would go away since my brother was only seven years old, and I told Batiushka so. He firmly stated, however, that the desire would not leave and that I would become a monk. Then asking me what my profession was (I was a joiner), he told me to make a coffin in the Shamordino workshop, and showed me what size. In two days I brought the coffin. Batiushka commended me, gave me a \textit{prosphora}, and blessed me with an icon of the Saviour. I was deemed worthy of speaking for a little with Batiushka. There was much on my conscience, and Batiushka told me everything with amazing accuracy, just as though he had been present with me. I had not even intended to speak about much, although I had sins which were as yet unconfessed. From that time on, these sins were no longer repeated.

Soon after this, the Elder Amvrosy reposed. I visited his successor Batiushka Joseph; I told him everything about myself and what Batiushka Amvrosy had told me, and he said, ``It is the will of God that you live in the world until you are forty years old.'' (I was only twenty at the time.) ``Such is the will of God. You have to stay until the affairs of your home are settled.'' And that is how it happened. I stayed in the world until I was forty. During those twenty years, I went to Batiushka Joseph twice a year. What benefit I received for my soul! I would come to him with a dejected spirit, as though a stone were lying on my soul, and I would leave feeling easy and well with renewed hope. Even just coming into the skete, my soul would feel lighter. I would ask the cell-attendant how Batiushka's health was. ``Batiushka is weak,'' would be the usual answer in later times, ``but I'll announce you right away.'' I would go in to the Elder; smiling, he would ask, ``How are you getting along?'' I still would not know where to begin, but Batiushka, by the spirit of clairvoyance, would speak concerning all the things which had caused me to come. I would feel as if the whole burden had been lifted. I witnessed many cases of clairvoyance with Batiushka. I always confessed to him. Once when I was preparing to go to Optina, I imagined that Batiushka would not confess me, that he would chase me away. I went in to Batiushka, and he said, ``Come tomorrow; I'll hear your confession; I won't chase you away.''

My sister had been matched to a merchant's tradesman who was asking a dowry of five hundred rubles. Everyone liked him, and I wrote to Batiushka to ask his blessing. Batiushka answered, ``He is asking for the money because he needs it to pay his debts. You shouldn't give him your sister even without a dowry. Find out everything about him really well.'' It turned out that he really did owe much and, besides that, he was a great drunkard.

Also, we had to build a home. There was one for sale in the neighborhood, but we did not like it. Nevertheless, I asked Batiushka how he would bless us. Batiushka said, ``God will bless. Buy it. Only first serve a \textit{moleben} to Saint Nicholas.'' We served the \textit{moleben} and bought it at a good price. Together with the reconstruction, it came to six-hundred and ninety rubles. It was light, roomy, and a new house such as this could not have been built for a thousand rubles.

After having stayed at Optina on one occasion, I came to Batiushka before leaving for Shamordino. Batiushka said, ``May God bless, but there is nowhere for you to hire horses now. Go by foot, and when you return you'll see that someone will be glad to bring you for a five-kopeck coin.'' I was at Shamordino the whole day. In the morning when I came out, two carts came up. A peasant overtook me and said, ``Where are you going? Get in; I'll take you.'' I asked for how much. He said, ``I'd be grateful even for a five-kopeck coin.'' I came to Batiushka and told him everything. He only smiled and blessed me to go to the refectory to have dinner.

How many times I came to Batiushka and wanted to leave the world, but he always postponed it. Finally this year, I went to Batiushka at Christmas-tide. I went to him in great confusion of spirit, thinking that again he would not bless me. But Batiushka greeted me with a happy new year and said, ``Now it's time for you to stay in the monastery or else you will be grieved. Your home affairs will go better without you. Ask the Archimandrite to take you into the monastery, for there is no room in the skete.'' It is already the sixth month now that I have been living in the monastery. My home affairs are going quite well. But there is no more the dear Elder, no one to make decisions, to give comfort in all the sorrowful circumstances of life.

Eternal be your memory, our dear, never-to-be-forgotten Batiushka! How sad and difficult it is seeing the skete without you. You have left us orphans and departed for the habitation of all them that rejoice!\footnote{Cf. Psalm 86:7.} By the sight of your modest grave you call us, even as you directed us during your life, to take the road to the eternal and never-ending Kingdom.
\end{longquote}

One woman had two benefactresses who undertook to help her. Intending to write to them, she asked Batiushka for a blessing to do this. Batiushka, hearing the family name of the first said, ``God will bless. Write.'' But he kept silence concerning the second. To her repeated questioning he said, ``Write, but she will not read your letter.'' She wrote, but received an answer from others who informed her of this lady's death.

A sister relates: ``In 1902, I came to Batiushka during Compline. Going in, I asked if Mother N. had written of one of my concerns. Batiushka answered, `No, she hasn't written anything. But O. is writing \textit{there} and asks to come.' (Batiushka put such an emphasis on the word \textit{there} that my attention was irresistably drawn to it and it forced its way into my memory.) Batiushka continued, `I haven't had any time to write an answer, but by word of mouth say to her, ``God blesses, but only after the fast because there are many people here now.''' The following day I went back to the convent and very calmly related Batiushka's words to Mother O. whose face filled with amazement when she heard me. When I asked the reason, she answered that she had written nothing to Batiushka concerning her desire to go there. During Compline she had written me a note asking me to get a blessing from Batiushka for her to come, but she had not managed to send it. Then I understood why Batiushka had put an emphasis on the word \textit{there}; he spoke at the same time Mother O. was writing. In the same moment that her note was being written, he gave her an answer, which corresponded exactly to her note.''


\begin{longquote}{The monk D. of Optina Hermitage related the following concerning himself:}
In my early youth the desire to enter a monastery arose in me. With this purpose, I came to Optina Hermitage. The great Elder, Hiero-schemamonk Amvrosy was then alive. I explained my desire to him, but he told me to wait until the end of the term. I did not understand what term he meant; I only understood that it was not yet time for me to enter the monastery. So I went back to my home town V. where I soon got a job in a shop with a merchant. After having lived with him three years, I happened to be looking at an announcement in the magazine \textit{Niva} when I saw a notice that the Elder Amvrosy had passed away. This news surprised me very much. Despite the displeasure of my master, I decided to go to Optina immediately with hopes of being there for the burial of the venerable Elder. To my great sorrow, the body of the deceased Elder had already been given over to the earth. His immediate disciple Batiushka Joseph had taken his place, and to him I turned with my longstanding desire of entering the monastery. He answered me, ``Wait, a little. There will be a place for you.'' Not knowing what to do after hearing these words, I left the Elder. I had lost my previous job, and only one year remained until my departure for military service, so I decided to look for a job in the village K., near Optina. I soon found one with another merchant, with whom I had lived earlier while still a boy; I came to him without making any agreement as to a salary. After I had lived with him for a year, I informed him that I had to depart for military service. The good merchant gave me one hundred rubles for the year's salary and a ten ruble bonus. I was unspeakably happy. I, a penniless pauper, had received such a bountiful reward. I now had the opportunity, it would seem, to dress a bit better and to put something away for future needs. Having been conscripted into the military service, I spent less than a year in the service when I was appointed a surgeon's assistant, a duty which I fulfilled for more than four years. At a time when the other poor soldiers were enduring great need, I lacked nothing, through the prayers of the Elder. After the term of military service, I immediately joined the community of the Optina Hermitage brotherhood.

When I remember the past now, I am always amazed at how the merciful Lord, through the prayers and blessings of the Elders Amvrosy and Joseph, led me to favorable places and lightened the heaviness of the difficult military life. Wondrous are the works of God!
\end{longquote}


\begin{longquote}{Mrs. B. relates:}
I came to Optina for the first time in 1896. After visiting Batiushka Joseph, I received from him the life of the reposed Elder Amvrosy compiled by Archimandrite Agapit. While reading this wonderful life, I felt disturbed about a few details concerning the Elder's jokes and facetious sayings, for I felt that there was no reason to write about them. Having read the book, I returned it to Batiushka, but said nothing about my impression. I had not yet returned to the guest-house when a nun came to me and brought from Batiushka a supplement to the above-mentioned life. In this supplement, the author said that it had reached his ears that many lay people were dismayed by the details described by him concerning Elder Amvrosy, and he explained the reason which persuaded him to compile these details.

Before going to Optina, I had heard much about Fr. Clement Sederholm. When I went to Optina in 1898, I wanted to obtain a copy of his life; but every time I went by the Optina bookstore, something hindered me from fulfilling this desire. The day of my departure had come, and I went there expressly for this purpose, but it turned out that the store was closed. I went to the skete to Batiushka Joseph and had barely gone in to him when he said, ``I will give you a book you don't have.'' With these words he ordered the cell-attendant to bring the book, \textit{Fr. Clement Sederholm}.

In 1905 and 1906, I lived in Saratov and underwent there all the horrors of the disturbances which had seized Russia. Strikes on the railroad had cut off Saratov from the rest of Russia. Neither letters nor newspapers were received and there were bloody fights in the streets. Nevertheless, news had reached us that in Petersburg and Moscow the Russian people were uniting for the protection of the Faith and the Tsar. Saratov's Most Reverend Bishop Germogen gathered all those in whom the fire of love for God and the Fatherland had not yet burned out. These were simple working people, small scale merchants and several priests, and I alone from the higher society of the city. Vladyka Germogen entrusted one priest and myself with the compilation of regulations for the new Union which was being formed. Just then, however, the first train came after the strikes, and brought me a multitude of letters and newspapers, with a package and letter from the Optina Hermitage. A spiritual daughter of Batiushka Joseph, V. I., who was living at the monastery, wrote to me with his blessing and good wishes, and sent me the proclamation and regulations for the Union of the Russian People, along with Batiushka's blessing for the opening of such a Union in Saratov. It goes without saying that I immediately brought the letter and the package to Vladyka Germogen. In a few days, after numerous meetings in the room for spiritual reading at the archpastoral home, we opened the chapter of the Union of the Russian People, with which I sympathize and which I serve until now with all my strength. Let me add that I had not been to Optina for two years and had not corresponded with Batiushka, but had only sent him greetings at Pascha and Christmas.

In 1910, I was already living at Optina when K., the wife of a general, a wealthy widow who lived without relatives, came to visit me from M.. After having returned from visiting Batiushka, she told me with some surprise that Batiushka had first of all asked her if her servant was a good one and then added that it was necessary for one to be careful with servants. When she visited Batiushka before her departure, he again spoke to her only about the servant; so firmly did he speak that she became worried and hurried to depart for home. She had barely returned to M. when she discovered that the yard-keeper of her home had been arrested and that she was being summoned to court. A whole band of thieves that had committed several murders and robberies was arrested in M.. Upon the person of the chief of the band was found a list of people whom they were planning to rob, along with sketches of their apartments and information about them. Mrs. K. was also included on this list. The information about her had been given by the servant.
\end{longquote}

From all of the above it is obvious how great was the strength of the Elder Joseph's prayer and how well those who entrusted themselves to his advice with faith passed their earthly pilgrimage. The Elder, despite his gentleness and compliance, was always firm in his decisions, because he acted by the inspiration of the Holy Spirit which dwelt in his pure heart. He said to several people that they should always hold to his first word and decision; for he, under the influence of persistent requests, might change his mind, which then would be man's decision, forced upon him. However, the first is always inspired from above. There were many who asked the Elder's advice, but then took their own way contrary to it. They suffered bitterly and were tormented by regret, but it was too late.

We will set forth a few examples.

\vspace{1cm}

One woman related: ``My sister-in-law cunningly called me home in order for me to obtain my share of my father's estate. The Elder did not bless me to go, but I persisted and went. How much grief I saw there! My second-oldest brother took offense with me and the whole affair resulted in much unpleasantness''.

``My niece had many suitors; her parents wanted to marry her to a merchant, but the young woman liked another. They went to Batiushka who blessed them to marry her to a teacher. He said, `Marry her to the teacher. He will be a deacon.' But the mother did not obey and married her daughter to the merchant. Soon after the wedding, he left her amid great vexation and grief.''

A woman asked Batiushka what trade her relative should go into; he wanted to trade in wine. The Elder answered, ``No, he might be ruined in trade and go to jail.'' However, the relative did not obey the Elder's advice and deposited security. He soon lost his trade and had to sell his home in order to pay off his debt so as not to be imprisoned.

A peasant who was finishing his term of military service asked Batiushka what he should do in life, and he asked for a blessing to get married. The Elder said that he should wait a year before making any plans, and that he should not get married at all. In a year he was taken to war again. Having returned from war, he went to the Elder and again asked for a blessing to marry. The Elder did not give him a blessing, but said, ``Enter the Tikhonov Hermitage and pray that the saint will heal you of your bodily and spiritual illnesses.'' The soldier did not obey and got married. He had such misfortune that he was unable to find a job anywhere. The wife's family finally drove him out of the house. He went to the mines and to this day is filled with remorse, having neither bodily nor spiritual peace.

The nun M. S. relates: ``I had an unmarried sister who came from Warsaw to Moscow and had no job. She wrote a letter to me, requesting that I ask Batiushka how he would bless: to go to Efremov and there open her own shop or to stay in Moscow and wait for a job. Batiushka answered, `Let her stay in Moscow and look after children,' which appeared offensive to her, since she thought that Batiushka was sending her to be a housemaid. She rejected his advice and said, `Am I able to look after children?' But soon she got a job as supervisor at a nursery, a so-called `day nursery.' She wrote to Batiushka and asked his forgiveness, for she had not understood his words and wanted to disobey. When the nursery closed, she was again without a job. She wrote to Batiushka and asked what she should do: open a shop or wait for a job. She received the answer, `Wait for a job'. But she did not obey, for her patience ran out. After her business went to rags and ruin, she was sent for from Warsaw to return again to the place where she had lived for twenty-five years. Batiushka did not bless her to go to Warsaw and said, `Go to the Trinity Lavra of St. Sergius. You'll find a job there.' She did not obey, but left for Warsaw. Along the way she became ill and her 5\% coupons were stolen from her, so she had to come back.''

A nun whose father was ill asked him to turn his land over to her while he was yet alive. He answered, ``When I get well, everything will be yours.'' She asked Batiushka and he blessed her to ask her father again to turn the land over to her while he was yet alive. However, she was ashamed to. The father became well but did nothing. After his death, Batiushka blessed her to demand her portion, but her mother was against it and she was afraid to grieve her mother. Meanwhile, her brother sold everything, so she had to depend upon him for assistance. She went to Batiushka and related her grief to him, but he said to her, ``You see, I told you, but you didn't obey.''

The novice O. M. related: ``My brother, who was working in a factory, walked by a machine and stumbled on some scraps of iron. He injured his right hand, the thumb of which was torn off, and he collapsed, bleeding. The directors willingly offered him five hundred rubles in compensation and a position as an inspector, but he did not want to take it since he had been advised to go to court. I asked Batiushka who said, 'Yes, they'll try the case and then some, but they won't give him his hand; it's better to receive less and that willingly.' But my brother did not listen. He has been going to court for ten years now and still has not received anything.''

Left alone after the Elder Amvrosy reposed, the Elder Joseph toiled less over material than over spiritual matters. His firm and authoritative decisions benefitted and calmed those who came to him with faith. There were also astonishing instances of Batiushka Joseph's receiving direct orders from his deceased guide, the Elder Amvrosy.

Once, at her request, Fr. Joseph allowed a nun to visit her parents. Suddenly, during the period of rest after dinner, he clearly heard the voice of the deceased Elder, ``She shouldn't go. Tell her to stay.'' Fr. Joseph sent for the nun and explained to her the will of the Elder, which she accepted with tears of compunction.

The Elder Amvrosy did not give a blessing for the Shamordino sisters to build separate cells and to live alone. Some two years after his death, one nun, owing to her bad health and love for the solitary life, requested the Abbess and the Elder Joseph to allow her to live alone. The Elder, having given in to the insistent entreaty of the nun, as he himself related later, became disquieted in spirit. He suddenly saw before him the Elder Amvrosy sternly saying, ``You are acting contrary to my blessing.''

A machine for kneading bread was donated to the Shamordino Convent, but it was too heavy for the sisters to use. They told the Elder and explained that knowledgeable people advised them to make a horse-driven kneader out of the machine. Somewhat undecided, the Elder answered, ``I have to think about it.'' Later the sister managing the constructions in the convent came to the Elder with various other business questions, but the Elder, not listening to her, said with animation, ``Without fail you must make a horse-driven kneader out of the machine; this is what Batiushka Amvrosy desires.'' The sister gazed with surprise at the Elder, but he answered her perplexity by saying, ``I lay awake all night and thought about the kneader, not knowing what to decide. I suddenly heard Batiushka Amvrosy saying to me, `Without fail, a horse-driven kneader must be made.'''

They made the horse-driven kneader and the machine works very well.

Many turned to the Elder Joseph without going to see him in person, and received answers from him by letter. They would do nothing without his advice, since they knew from experience that with his blessing everything would go well. He received a great many letters from convents and monasteries. These letters, which are preserved as priceless treasures and which will be published in time in all likelihood by Optina Hermitage\trans{The collected letters of the Elder Joseph were never published. This was written in 1911, only a few years before Russia was engulfed by the Communist revolution.}, were just as simple, powerful, and wise as were his words.
