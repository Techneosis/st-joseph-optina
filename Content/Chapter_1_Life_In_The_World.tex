\chapter{Life In The World}
\hugered{E}LDER HIERO-SCHEMAMONK JOSEPH, in the world Ivan Evfimovich Litovkin, was born on November 2, 1837, in the village of Gorodishcha, in the Starobelsk District of the Kharkov Province. His parents Evfimy Emilyanovich and Marya Vasilyevna Litovkin were simple people, but very pious, kind, and intelligent. His father was the village-head for seventeen years and enjoyed great respect and love. Both he and his wife were very merciful to the poor; they gave from their possessions with a generous hand, and often without the other's knowledge. Evfimy Emilyanovich lended to whomever was in need of money, although it would sometimes happen that he did not get it back. He loved to receive into his home monks whose obedience it was to ask for alms, and without fail he would give to each of them five gold rubles for his monastery.

In accord with the ways of olden times, they loved very much to go to church and to read spiritual books, especially the lives of the saints. They had six children, three sons and three daughters. When the second son was born, his parents named him John in honor of Saint John the Almsgiver,\trans{In combination with the name of his patron saint, the writer uses the more formal Slavonic ``Ioann'' (John). ``Ivan'' is the popular form of the name.} for whom they had a special devotion. Truly loving their children, they tried to give them Heavenly treasure rather than earthly wealth; therefore they raised them in the fear of God, piety, and obedience. The father had a very lenient character, but the mother was more strict. The children feared her very much, but loved her just as she loved them. Of all the children, the father loved Ivan most, who even in his facial features resembled him. Once he took him for a ride on the droshky and punished him along the way for some prank. Afterward the gentle father felt remorse for a long time and would say, ``Oh, he's just a boy, and I struck him!''

Ivan's older sister Alexandra, who especially loved him, began to teach him grammar, so by the time they put him in school, he was already able to read. In general, he was a good student. ``In school,'' Batiushka related, ``they didn't punish me; but once the teacher ordered me to punish my younger brother Peter. This was very hard for me and pained my soul.''

The mother would take all the children with her to church, and at home would make them pray together with her. According to stories told by the Elder himself, from his early childhood he loved to be in church, where he even sang in the choir. ``I remember how mother would wake me to go to Matins and the Liturgy. I didn't want to get up out of bed early, but there was nothing to be done: I had to get up. Because of that, in church and for the whole day afterward I used to feel so good and happy in spirit!''.

Then he would continue, ``At home, mother would also make me read the Akathist to the Saviour or to the Mother of God. Sometimes it happened that we would be standing at prayer and suddenly through the window we would see a bear being led along the street; there would be a lot of noise and many people. It was so frightening! We would begin to pray even more zealously.''

The visitors hearing this would sometimes ask him, ``Surely the children wanted to go closer to the window and have a look at these strange sights?''

``No, that was impossible; our mother was strict,'' he would say.

From childhood Ivan was sickly; because he was scrofulous, he became hard of hearing in one ear and near-sighted. In addition, he often met with various mishaps. When he was still small, his sister accidently set him on a very hot stove-couch. Another time, he scalded himself with water. At another time, a boy who was racing by knocked Ivan off his feet, and he bit off the tip of his tongue. Notwithstanding, he was a very lively and happy child by nature.

His sister, who afterward became a nun, related that little Vanya\trans{The affectionate and diminutive form of ``Ivan.''} was a very affectionate child. With his tender and sensitive soul, he could somehow feel another's grief, although his inborn modesty and shyness did not allow him to speak out. When he noticed that someone at home was melancholy, he would quietly linger about him and then snuggle up and caress him fondly.

The father always said of him, ``Something special will come of this boy.'' It is worthy of note that if Evfimy Emilyanovich said something of his children, it later came to pass exactly so. Among other things, he expressed even then the desire that one of his children become a monastic. The first to respond was his daughter Alexandra (afterward the nun Leonida), and after her, in time, his son Ivan. Vanya's catechist, a protopresbyter of very good life, on his part said of the boy, ``Wait and see, something unusual will come of him.'' Thus from earliest childhood, all noticed in him the special seal of the favor of God.

When Vanya was four, he lost his good father, who died suddenly. Nevertheless, the little one fared well under the wings of his tenderly loving mother. When Vanya was eight years old and was playing with friends, his face suddenly changed; he raised his head and hands upward and fell unconscious on the spot. They carried him home and put him to bed; when he revived, they began asking him what had happened. The boy said that he had seen the Queen in the air.

``Now why do you think you saw the Queen?'' they asked.

``Because she had a crown with a cross,'' he answered.

``Well, why did you fall down?'' they asked again.

In reply to this, he cast down his little eyes and quietly said, ``Near her there was such a sun ... I don't know, I don't know how to say it!'' he added quickly and began crying.

This wondrous vision left a deep impression on the soul of the eight-year-old child. From that time on, he completely changed; he became quiet and thoughtful, began to withdraw from children's play, and was constantly with his mother. The gaze of his gentle eyes became still deeper, and his child's heart became inflamed with faith and love for the Heavenly Queen.

Soon after this, and not long after they had moved into a newly built house, there was a great fire in their village. The child saw the fear of those at home and he too understood the danger. Not knowing where to turn for help, he stretched out his little hands toward the church near their home, named after the Protection of the Most Holy Mother of God, and cried, ``Heavenly Queen! Leave us our home! After all, it's completely new!'' The child's prayer was heard; everything around burned, but the home of the Litovkins remained unharmed.

By this time, Ivan's older brother Simeon was married, as was also his sister Anna. Before two years had passed, the mother took her daughter Alexandra to the Borisovsk Hermitage and left her there, having entrusted her to the Heavenly Queen. All at home were much saddened by the departure of the gentle and loving girl, but Vanya grieved most of all, and he wept over the separation from his beloved sister.

But soon a new, still more difficult separation awaited him. A year later, in 1848, when he was eleven years old, cholera appeared at their village; and the first victim to this terrible visitor was his mother. The boys Ivan and Peter now became orphans. At the funeral, little Ivan exclaimed with much grief, ``Heavenly Queen, what are you doing? Sister went away to the convent, and now you've taken Mama away from us!'' Everyone present involuntarily began to weep when they heard the outpouring of the child's grief; the priest, that same protopresbyter who had foreseen in the boy the future chosen one of God, wrote down these words and recorded the time. Later they found out that his sister Alexandra Evfimovna, who was living in the convent, had known nothing of the death of her mother; on that very day and hour she became very despondent, and falling into a light sleep she saw a coffin with her mother inside floating down a river and heard the very same words, ``Heavenly Queen, what are you doing? ...'' and the rest.

From that time on, the life of the orphans changed markedly. The older brother, Simeon Evfimovich, became the head of the household, and although he was a very capable person, he was subject to a great weakness--he drank wine.\footnote{The Elder always spoke well of him, as of an honorable and noble person, whom evil company had led to a dissolute life because of this weakness in his character. While in the military service he was deprived of his rank several times because of this weakness, but afterward he re-qualified. The Elder saved two of his letters which he even planned to have printed as an example of humble confession and sincere, deep repentance.} After the mother had reposed, he very soon left home, and all of the parents' estate deteriorated; after a year, when Alexandra Evfimovna came from the convent, she found nothing at home. Her relatives begged her to stay and take care of the orphans; Ivan especially wept inconsolably and begged his sister not to leave him. She related, ``They would grab me by the belt with their little hands and weep bitterly, 'Sister, take us with you to the convent. There's no one for us to live with here!' My heart simply bled, looking at them. Vanya would even repeat these words in his sleep.''

It was unbearably difficult for Alexandra Evfimovna to part from them, especially from her favorite; but she firmly decided to give up everything and go her own way for the sake of God, and to take care of her small brothers from the convent when possible. As a result of this grief and anxiety which was beyond her strength, she suffered ill health for the rest of her life. Although she was later healed by the Kozelshchansk icon of the Mother of God of an agonizing illness which befell her soon after returning to the convent, she always remained weak and sickly with overwrought nerves. She related afterward that she had almost decided to take Ivan with her and to place him in the monastery closest to Borisovka, but the above-mentioned protopresbyter convinced her not to do so. Apparently, it was pleasing to the Lord that His faithful servant be tested.

The older brother kept Ivan with him, while the married sister took the younger Peter. But soon Simeon Evfimovich was constrained to go and live with strangers, and therefore he sent his brother elsewhere. From that time on, the trying experiences of a bitter orphanhood began for him. The poor child who had been used to the tender kindness of his parents was forced to see and endure much. His lot was difficult and harsh; he was tried in everything: cold, hunger, beatings, and labor beyond his strength.

At first his brother placed him with a tavern keeper whom he knew. Here the poor boy was especially miserable. His brother came often to play cards and drink wine with the owner, and the boy was forced to do everything by himself since he and the tavern keeper were the only two living there. Unable to endure such a life, Ivan left his master and returned to his brother. Although his brother felt sorry for him, he soon placed him with an Armenian in Nakhichevan. He did not stay there long, however, and was placed in a grocery store in Taganrog. Here he had to go for food to the owner's apartment at the end of town. Once while he was going to the store with lunch pails, he became very tired, sat down to rest, and right on the spot fell into a deep sleep. While he slept, someone took off his boots and stole the lunch pails, and for that reason the police took him to the station. At home, of course, he received punishment from the master for both the boots and the dinner.

Ivan left this master also and went by foot to Novocherkassk, to his cousin who was a deacon there. He arrived there famished since he had had no more than a few biscuits with him. Concerning this trip he said, ``For two days I ate absolutely nothing. I just didn't know how to beg, and people wouldn't give of themselves; so I humbled myself by not eating. When I came to Novocherkassk, I searched for the church where my cousin served. While waiting for the Liturgy to end, I sat on the church porch. Here two Cossack women with some rolls passed by me, and one of them said, 'This boy is probably an orphan and hasn't eaten anything,' and she gave me a roll. How happy she made me; it seemed so delicious to me, just like manna from Heaven!''

Although it was pleasant living with his cousin the deacon, it was impossible to stay for long. The latter sent him to his father-in-law, a priest. Here everyone came to love Ivan very much, and he lived well there. Fr. Joseph used to recall one episode from his life with this priest. Once at Christmas-tide, the priest's daughter proposed to tell fortunes in the mirror at midnight; she was frightened to sit alone and so she sat him down beside her. ``I really wanted to go to sleep,'' he said, ``and in order not to fall asleep, I began to pray; so we saw nothing.''

Later, his sister Alexandra succeeded in placing her brother in an iron shop, but it was equally difficult for him everywhere. Once the master, a former church warden of the cathedral, sent him to give a message to the workers who were at the top of a structure. He climbed up quickly, but when he started to descend, his head started to spin and he fell down unconscious.

When he was completely without a job, he would apply for day labor. ``He who wants to labor,'' he said later, ``always finds something.''

He would have to carry five sacks of flour weighing thirty-six pounds each and other such loads. Once while hauling boards from rafts, he stumbled, fell into the water, and began to drown, since he had fallen between two rafts and there was nowhere he could swim out. Suddenly, some invisible hand pulled him up to the surface of the water; the Lord had saved him from death.

On another occasion, he was taking leather goods somewhere with the proprietor. They stopped to spend the night at an inn. The owner went to sleep in the cottage, while Ivan stayed out in the cart. During the night there was a fire, and in the confusion everyone forgot about him. From fatigue he slept so soundly that he heard nothing, and when he awoke in the morning, he could not tell where he was, since everything around him had burned down.

Still another time, Ivan had to go from Stavropol to Rostov to find a position with some master. It happened that a certain Armenian had bought potatoes in Stavropol and was also heading for Rostov with two carts. Ivan made arrangements with him to be taken there, although both of them had to go by foot a good part of the way since the carts were loaded with potatoes. They came to the Don where there was a ferry and barracks built for the Cossacks. But the Don had begun to overflow, because a wind blowing from the sea made the water rise. The Cossacks refused to take them across. Not wishing to lose time, the Armenian proposed to Ivan that he travel with him along the shore. They went, but the water kept rising more and more. Then Ivan, not heeding the pleas of the Armenian, returned by foot to the barracks. He had to walk there in water up to his belt. On reaching the barracks, he spent the night there. The soldiers gave him dry shoes; but when he began taking his boots off, it turned out that they had frozen to his feet. By the next day the water had gone down, and they took him over to the other side of the Don, where he continued on his way to Rostov by foot.

Thus young Ivan constantly experienced dangers, but the Lord preserved the orphan everywhere. Despite the fact that his youth was spent in coarse and often corrupt surroundings, no evil injured his pure soul. He never drank wine or played cards. Once, he was served some weak grape wine called \textit{chikhir} when he was in great grief, but having drunk a small glass, he immediately felt tipsy, and his head began to spin. Thereafter he vowed to himself never again to take into his mouth any kind of wine. ``If a person becomes tipsy from such weak wine, drunk in such a small amount, then what would happen to him if he should drink much?'' thought the judicious youth,

He was completely indifferent to the female sex. He was once asked, ``Did you like someone in the world?'' Whereupon he answered with a naive simplicity which best of all proved his honesty and innocence, ``Well, since I was nearsighted, I couldn't make out anyone from afar, and I was ashamed to approach anyone closely--I was shy. It used to be very difficult for me when the master sent me to call someone when we had guests. There was no way I could make out whom I was supposed to approach.''

In the world, he generally experienced melancholy, and the unfailing companion of his sorrowful life was prayer--the only inheritance left him by his pious parents. His only place of comfort was the church to which his piety always drew him.

Finally he succeeded in finding a good position in Taganrog with the merchant Rafailov, who being a pious person, went with him to the church services. His job was to take the rye which the \textit{chumaks}\trans{Ukrainian ox-cart drivers.} brought in from various places, and then to distribute it in the city to various shops, both Russian and Greek. During the summer there was a lot of work, in the autumn less, and in the winter there was absolutely nothing to do. So after the coming of autumn, the master suggested to Ivan that he go to the village mill which was leased to a certain Ukrainian man. Not liking the noise of many people, Ivan was glad to live in the quiet of the village. During that winter, there was almost nothing for Ivan to do, since the mill was being repaired by the man who was leasing it.

The old miller turned out to be a pious man; he always went to the church services on feast days. Both of them became close friends with the parish deacon, and together they would read the holy books which were found there at the church. On feast days, the three of them would have heartfelt conversations. Ivan Evfimovich lived this way until spring. He still had no thought of monasticism, but about this time he received a letter from his sister, the nun Leonida, in which she suggested that he take up the monastic life in the skete of Optina Hermitage, long renowned for its spiritually experienced Elders. Then there began to burn in Ivan Evfimovich a strong desire to leave the world and to enter a monastery.

The master had come to love this well-behaved youth on account of his modesty, honesty, and love for work, and he had become so well-disposed toward him that he had begun to consider giving him his daughter in marriage. Once after Liturgy, the eldest son of this merchant said to his parents, ``Today in church I didn't pray so much as I watched Ivan Evfimovich and was moved by him; how well he stands in church, how diligently he prays, looking not once to the side, but entirely rapt in God. We couldn't find a better man for our sister; we shouldn't let him get away.'' Thus, the good qualities of his soul were manifest and attracted love and respect.

But the young Ivan was far removed in thought and heart from worldly attachments. His pure soul was now attracted to live with the hermit monks. His favorite pastime from childhood had been reading the lives of the saints. Now these wonderful examples, illumined by the splendor of Heavenly glory, rose before him and outshone all the beauties of this world. His soul did not find full liberty for itself in the workaday, bustling world, and within him there appeared an unquenchable desire to tear himself from the commotion of life, at least for a while, and to go on a pilgrimage to worship the holy places. His cherished dream was to go to Kiev and with trembling heart and lips to venerate the holy relics of the living temples of the grace of God which accomplishes all things. This dream moved him to pray fervently, ``Cause me to know, O Lord, the way wherein I should walk.''footnote{Psalm 142:10.}

When he learned of Ivan Evfimovich's desire to go on a pilgrimage to Kiev, Rafailov zealously set about to persuade him to remain in Taganrog, and revealed his desire to give him his daughter in marriage.

``It's always like that,'' said the Elder years later, as he recalled this circumstance. ``As soon as a person starts thinking of going on the path of salvation, right away there appears an obstacle and temptation.'' Truly, when he lived among sorrows, adversities, and trials, people passed him by without noticing him; but as soon as his soul began to separate itself from earthly concerns, the world was now at his bidding. For the poor orphan, his entire future stretched before him as one of endless need and constant dependence; then of a sudden, an offer of marriage to a rich merchant's daughter and the possibility of becoming an independent and wealthy person. Surely, is this not a temptation? Is it not a snare for the inexperienced fledgling? But he who was already ensnared by the love of Christ quickly and easily escaped by immediately coming to a decision, whereas another would likely have wavered. Not hesitating for a moment, he again asked his master's permission to go on a pilgrimage, and the rest he entrusted to God's will and direction.

His good master Rafailov, on seeing the sincere and fervent yearning that the youth had for God, dared not hold him back any longer. He dismissed him with peace and love, and only requested that by all means he return and stay forever in Taganrog.

With a knapsack on his shoulders, the young traveller set off for the goal of his cherished thoughts. Sacred Kiev was already standing before his mind. Along the way he stopped at his deserted birthplace, and after venerating his parents' graves, he cast a farewell glance at the place where his happy childhood had flashed by so quickly, and he continued on. Stopping by the Holy Mountains of the Kharkov Province, he spent several joyful days there and came to compunction in his soul, observing the peaceful life of the monks of that quiet monastery.\trans{This refers to the Monastery of the Dormition located in the Holy Mountains.} Nevertheless, he did not want to remain there; he was attracted to where the broad Dnieper casts its quiet waves.

From the Holy Mountains he set out for the last stop on his journey, the Borisovsk Women's Hermitage, where as was mentioned before, his sister the nun Leonida lived.

Mother Leonida was a frail, tender soul, loving and gentle. Left without any means, she was constrained to live as a cell-attendant, and besides her monastic obediences, to fulfill all the heavy chores for the nuns with whom she lived. But having lost her health, she couldn't work at all, and another nun took her into her care and subsequently became her spiritual friend. At that time the Borisovsk Hermitage was noted for its strict monastic life and for its being under the spiritual guidance of the Elders of Optina. In the Hermitage also, there were experienced and wise Eldresses guiding the young sisters. Mother Leonida and her friend the nun Anatolia were under the immediate supervision of the wise Eldress Schema-nun Alypia, who in turn was a disciple of the Optina Elders Leonid and Makary. Once when she was at the Optina Hermitage with her Eldress, Mother Leonida with tears told the Elder Makary of her grief over her orphan brothers and expressed her fervent desire that her beloved Vanya go to a monastery. ``Don't worry,'' the Elder said to her affectionately, ``he'll be a monk.''

At the time when Ivan Evfimovich came to Borisovsk Hermitage, the Superior was the Abbess Makaria, who led an exalted spiritual life. Discipline under her was very strict; for example, nuns were not allowed to receive men in their cells, even though they be their own fathers. Only men doing necessary work for the cells were allowed in the court or in the entrance hall. The sisters would prepare food for them and carry it out to the porch; in the wintertime, the workers carried the small pots to the reception room and ate there.

Learning about these strict rules of the convent, and that there would be no other chance of seeing the one person who was close and dear to him, the loving brother did not falter at the obstacle which had arisen. He took a shovel and axe in hand and went to the indicated cell in the capacity of a worker in order to chop firewood and clear the snow. Mother Leonida, half dead from sickness, shed bitter tears as she gazed through the window at her ``little brother,'' and she couldn't get enough of looking at him. With what love, with what tenderness would she have embraced her little orphan, but the vow of obedience is holy, and she submitted and prayed. The wise Eldress Alypia, knowing the spiritual condition of her disciple and the pain and distress of her tender and loving heart which had already suffered so much agony, decided to allow the modest and God-fearing boy into the cell, and took upon herself full responsibility before the strict abbess.

O happy hours! How many tears of joy and sorrow were poured forth here, how many memories were brought before them, how much was said that was heartfelt, intimate, profound! The Eldress Alypia more than once listened to their conversation, and noticing in the youth a true yearning for God and readiness for struggles, said to him, ``Forget your Kiev and go to Optina, to the Elders.'' Ivan glanced questioningly at his sister and in her eyes read the answer, ``Obey.'' It was enough; Kiev was forgotten.

The next day, with the same knapsack on his shoulders and with a last blessing from the Eldress and his sister, who had taken the place of his mother, Ivan set out on his way anew, but now not to where his heart had attracted him, but whither the hand of obedience had directed him. Reaching Belev, he stopped at the convent to get directions for Optina Hermitage. The nuns of this convent, like those of Borisovsk, were under the Elders. The Elder Makary was no longer living, but in Optina a new lamp had been kindled, the immediate disciple of the Elders Lev and Makary, the Elder Amvrosy.

It happened that on this day two Belev nuns were driving to Optina; upon learning that the young traveller had come from afar seeking the Elder, they took him along on the driver's seat. Having arrived at Optina, the nuns went to the Elder Amvrosy, and in the course of the conversation they said to him, ``Batiushka, we have also brought Brother Ivan along with us.'' They called him ``brother'' in jest, because of his monastic inclinations.

The Elder looked at them seriously and said, ``This Brother Ivan will prove useful to us and to you.'' Thus the great Elder Amvrosy, not yet knowing of whom they spoke nor yet having seen him, already foresaw his high calling and he prophetically foretold what benefit Ivan would subsequently bring to Optina itself and to all the women's convents under the Elders.

The new arrival stood trembling before the Elder Amvrosy; simply and unpretentiously, Ivan told him of his life and asked for a blessing to go to Kiev, whither his soul had been long yearning, and where, perhaps, he could stay permanently. The clairvoyant Elder struck him lightly on the head and said, ``Why do you want to go to Kiev? Remain here.'' Again longed-for Kiev eluded him, but Ivan believed deeply that the words of the Elder contained the indication of God's will for which he had so ardently prayed. Therefore, with the word ``Bless,'' he bowed to the Elder and entrusted himself to him. This was on March 1, 1861.
