\chapter{Father Joseph's Appointment As The Brotherbood's Confessor And Skete Superior}

\hugered{A}T THE END OF 1893, the skete Superior and general confessor of the brotherhood, Elder Hieromonk Anatoly\footnote{There is a separate booklet on him published by the Kazan-Amvrosiev Hermitage.} came down with the illness from which he died. In accord with the common choice and desire of the whole brotherhood and in agreement with the proposal of the Superior, Archimandrite Isaaky, Hieromonk Joseph was appointed confessor by a decree of the spiritual consistory.\trans{Each diocesan bishop had a spiritual consistory, under the immediate governance of the diocesan hierarch, for the administration of the diocese and for the implementation of matters involving the spiritual court.}

Yes, he indeed was a luminary who had been well prepared, concealed until the time.\footnote{Fr. Joseph became Elder exactly thirty-three years after entering the skete.} He took upon himself this great and difficult pastoral service simply and humbly, just as he accepted everything, "seeking not his own, but that of his neighbor.”\footnote{Cf. I Cor. 10:24.}

With full trust and with filial devotion and love, the brethren of Optina entrusted their souls to him. Many of the older hieromonks, together with the Superior, had already been confessing to him for a long time; now everyone was turning to this wondrous man and “no one went away from him empty-handed and unhealed.”

As a small example of the spiritual relationship of a true disciple with the Elder, we will take the following excerpt from the diary of one diligent monk:

“Lord, by the prayers of my Elder, Father Joseph, bless me to write down his words for the benefit of my soul.

“Once Batiushka said to me, 'Tend to yourself and it is enough.' He also said that one should reproach oneself more often, be patient in everything, and thank God for everything. Then the Elder told me the following for my edification:

“'Once, one of the holy Fathers heard a beggar reproving himself. It was wintertime, and being half-naked, barely covered by a mat he lay on a heap of dung and was shivering because of the cold. Meanwhile, he was saying to himself, “You don't want to bear this, wretched one? The holy martyrs endured worse. They were naked in the winter and spent their time in prison, with their legs rammed into the stocks. Behold, you can stretch out your legs, and are even covered with a mat.'"

This story brought great benefit to my soul.

“'Batiushka,' I said, 'I am greatly overcome by laziness. I know it's not good, yet I'm still overcome.'

“The Elder said, 'In the Gospel it is written that “the violent take the Kingdom by force,''\footnote{Cf. Matt. 11:12.} and therefore it is necessary to force oneself in everything. It pays to cut off passions in the beginning while they are young, for they are like small barking pups; frighten them and they run away. But let them build up strength and enter into you, and they will stand up against you like lions. You won't have the strength to fight them.'

“I asked, “In what should one force oneself most and what should one abstain from most, Batiushka?' The Elder said, 'In sleep, in food, in drink, in conversation; and most of all one should not speak in church.'

“'Batiushka, sometimes I have such zeal for everything good, but then such negligence and laziness comes that I have no desire to do anything. I sleep and eat without limit; I don't feel like praying; I abandon my cell rule--what should one do on such occasions?'

“'At such times one should force oneself in everything good and in prayer; it depends on you when you force yourself to pray in times of negligence and laziness; but when you have zeal for everything good, then it is from God.'

“'Batiushka, where should I begin in order to have a little more attention within myself? Perhaps I shouldn't go out anywhere unless it's absolutely necessary.'

“The Elder said, 'This is where you should start. Sit in your cell and your cell will teach you all things.\trans{\textit{Apophthegmata Patrum}, Moses, no. 6.} Be patient, for thoughts of leaving will trouble you, only don't give in. You see, all the saints went along this road. Read how Saint Theodore the Studite orders one not to stay and converse when one leaves church; it is also mentioned in the rules. But we have distorted the order of the monastic life and have turned everything upside down.'

“Batiushka also quoted these words that the Elder Lev wrote to one spiritual daughter, ‘Behold, you have three weapons for spiritual warfare: humility, patience, self-condemnation. With these conquer.'

“When I asked whether I could give things to others, the Elder said, 'You may, if you have something unnecessary, but give nothing of the monastery's without a blessing.'

“'And may I accept something from someone if they offer it?'

“'You may accept things too, for it is a sign of humility, as Abba Dorotheus says. But if you do not need it, of course, it is better not to take it.'

"'Batiushka, for a long time, a thought has been telling me not to possess anything, as the holy Fathers teach.'

“The Elder answered, 'Keep what you need and what is necessary, but do not collect unnecessary things. If you do not possess anything and yet you grieve, then what's the use? It is better to hold to the middle way. You may possess, only do not become attached to anything, and be as though you had nothing; the saints had such a temperament.'

“'Batiushka, the Jesus Prayer is going badly with me. It appears such a simple thing that I should be able to say it always and everywhere--but no, I forget.'

“The Elder said, 'Yes, although it is a simple thing, it is hard to keep it. You say it a few times and forget. You remember, say it another ten times, and again, dispersion. In a day, you may do a hundred and imagine that you have learned prayer. Therefore, in the beginning, it is better to keep count until you obtain the habit.'

“After confession once, I said to my dear spiritual Father and Elder, ‘Batiushka, how much I would like to be always assiduous in my calling, in the fulfillment of the monastic vows, and in all my works and actions before the Lord. Sometimes my heart seems to burn with love for God and is ready to fulfill His will, but no sooner do I decide to take up an attentive life, than the enemy immediately gets the better of me. My soul grieves over such negligence. Time is flying; when will I begin to live assiduously?'

“The Elder replied, 'Yes, what can one do? We are always careless before God, but then, it is a matter for prayer. You know that it is written by Saint Macarius the Great, “We must cry out shamelessly to God for Him to have mercy on us and help us, because we can do nothing good of our own strength or of ourselves. But if we pray often, importune God, cry out shamelessly in great humility, then the Lord will help us.'''

“I had not been to see Batiushka for a long time. The dear Elder had been ill and had not received anyone for a month. Our dear Batiushka became very thin, yet more wrinkles were added, but his appearance became still more attractive, truly like that of one of the monastic saints. Having received a blessing and having expressed my sorrow over his illness, I asked about a question which had been tormenting me for a long time, and received a profound spiritual answer. I was very greatly comforted by Batiushka and asked him not to deprive me of his spiritual guidance.

“Glory be to God! Batiushka received me again, and I, the sinner, received comfort and new strength for my despondent and ever-feeble soul. The grace-filled Elder strengthened me again concerning prayer. He also ordered me never to abandon the Rule of Five Hundred.\footnote{For the Rule of Five Hundred of the Optina Monastery see p. \pageref{cell-rule}.} 'If you are unable to fulfill it with prostrations, then you may do it without prostrations,' said the Elder. 'If you are not well or tired, then you may do it sitting; but, of course, if you are well, you should do it standing and with prostrations.'

“'Batiushka, by your holy prayers, glory be to God, I am becoming somewhat accustomed to prayer, even though it is only external.'

“The Elder said, 'Good. It does not matter that it's only external. By an unfailing fulfillment of the cell rule, the desire will grow to occupy oneself with the Jesus Prayer even apart from the rule.' Here Batiushka told me something very important and beneficial from the writings of a certain holy Father; but the enemy stole it from my memory, and I have forgotten what was said. It is surprising how Batiushka can find the right thing to say immediately. From this ability, it is evident that he is a true guide and himself a practicer of monasticism. I, the wretched one, am happy that I have a father so experienced in the spiritual life, an unerring true guide, whom I love with the true love of a son.

“Again Batiushka said to me that the Jesus Prayer should be pronounced distinctly and with a pause between each prayer. Thoughts do come--it's the devil's wont to cast thoughts at us to distract our attention from prayer. Then it is necessary to delve into the prayer more zealously, and the thoughts, that is, the devil himself, burned by the terrible Name of Jesus, will flee. Whereupon the Elder surprised me exceedingly. I was just ready to tell him something else when he, answering my thought, said, 'And sometimes the enemy, taking hold of the heart, annoys you with hatred and condemnation of someone.' And this is just what I wanted to say, for I was grieved at someone, but Batiushka had anticipated me.

“Lord, preserve me from deception and delusion and foolish zeal through the prayers of my Father!

“How bad my memory is! I hear much from my dearly beloved Elder, but I do not remember everything unless I write it down. For to my soul all his words are spiritual bread, gold . . .!

“Once Batiushka said to me, 'Many weep, but not over that which they should; many grieve, but not for sins; many are seemingly humble, but are not truly so. In order to be successful in the Jesus Prayer, one must conduct oneself humbly in everything: in the way one holds one's eyes, in gait, in clothing.'

“The Elder said that the Jesus Prayer grants great benefit to the one who says it. One ought by all means to get accustomed to saying it, so that it will be a comfort, especially in times of illness. If someone is used to saying it all the time, then he will say it in times of illness also and he will not be so bored, for the prayer comforts him. But if a man, when he is healthy, does not occupy himself with the prayer, then, when he falls ill, he will not be in any condition to pray, since he has not acquired the habit; thus it will be difficult for him. It is necessary, therefore, to learn and to become accustomed to the prayer while one is healthy, and to say it often. And although it may not be pure, nevertheless, you will be pronouncing the prayer with humility: Lord, have mercy on me, a sinner! ‘A heart that is broken and humbled God will not despise,'\footnote{Psalm 50:17.} as it is said.

“And as before, I went away from the holy Elder comforted.

“The Elder said that one should not move to another monastery without a special reason. You won't find peace in your soul. How much temptation, hatred, and spite there will be! One may seem to be your friend, but in his heart he has a knife prepared for you. You should not move of your own will; you will not be at peace and none of the comforts of life will console you. It is another matter, however, if you are moved because of obedience.

“I always went away from the Elder comforted and at peace, for I accepted his words as from the mouth of God. Lord, help me to guard all the advice of my beloved Elder, and guide Thou my steps.”

\vspace*{.75cm}

In January 1894, the Elder Hiero-schemamonk Anatoly, who was the head of the Optina Skete, reposed. Archimandrite Isaaky and the whole brotherhood had long before chosen the Elder Joseph as his successor. However, nine days had not passed after the death of the head of the skete when the Superior of the monastery received a decree from the spiritual consistory that Hieromonk * * * might be appointed to the position of skete head, if the Superior of the monastery were to agree.

The venerable Archimandrite was not troubled in spirit, and firmly and openly withstood outside interference. He gathered the brotherhood and suggested that they choose a skete head. The brotherhood unanimously chose Fr. Joseph. Hieromonk Leonty (now reposed), the Keeper of the Vestments, who was distinguished by his exuberance and straightforwardness, exclaimed more loudly than all with the enthusiasm peculiar to him, “Worthy! Worthy!” Archimandrite Isaaky presented the diocesan administration with the report of the unanimous election of Hieromonk Joseph as successor to the reposed skete head, Fr. Anatoly. This choice was confirmed, and on the twenty-fifth of March a decree was received.\footnote{Fr. Joseph was made Superior exactly thirty-three years after he had entered the skete.}

Thus, true humility always declines from honor and never seeks it; but the Lord Himself will bring forth His chosen one and exalt him in the sight of all.

With the acceptance of the superiorship, cares and labors were added to the Elder Joseph. These, however, did not burden the humble soul of the obedient monk. He also bore this yoke of superiorship without complaint and with meekness, showing himself to be an instructive example of a true pastor. Under him, everything proceeded in good order. In administrative matters he was practical and cautious; he did not undertake to alter anything which had been introduced by previous Elders and Superiors. But where several changes were required, he made them so humbly that no one was offended, and everyone willingly submitted to him.

Because of his humility, he used to ascribe everything to the good disposition of the brotherhood itself. Thus he wrote to his sister, Mother Leonida, soon after his assignment as skete Superior, ``. . . I direct the skete with all my helpers, the brotherhood. All are obedient and humble. I only have to tend to myself and worry about my own soul, and all the brothers, glory be to God, are peaceful."

In the skete, many flowers were being cultivated, which, on the one hand, demanded unnecessary expenditures, and on the other, drew the brotherhood from their monastic duties. The new Superior cut down the amount of flower-growing and permitted it only on the main walk and around the church.

Very characteristic, by its strength and simplicity, is the answer that the Elder-Superior gave to a lay visitor who asked why there were no greenhouse cucumbers in the skete now. “The Archimandrite said that we can wait until they ripen in the beds,” said Fr. Joseph with his meek smile. This humble answer, imbued with the strength of the monastic spirit, pleased the visitor so much that, despite his love for rearing greenhouse plants, he approved of and esteemed the order of the new head, with his whole soul.

He was able to a surprising degree to unite within himself the qualities demanded for the duties of a skete Superior and for the duty of Eldership. He was firm, strict, and exacting with the brotherhood. He taught them humility, patience, and unfeigned obedience, as well as monastic conduct in general. He did not so much teach them with the authority of a Superior as he inspired them with a father's love. At the same time, as Elder, he was able to calm all, to set everyone at peace, and to bring them to obedience and submissiveness. The brotherhood said of him, “What our Batiushka doesn't accomplish by an order, he accomplishes by his humility. He speaks and looks in such a way that even if you don't want to humble yourself, yet you do.”

He demanded that they ask his blessing for every thing and was never burdened by any question. He required this in simplicity of spirit, as a monk who was used to sanctifying every step of his life with a blessing.

He knew the whole church \textit{typicon}\footnote{See Glossary, p. \pageref{typicon}.} extremely well, and also the melodies;\trans{From olden times, \textit{stolpovoye} chant has been chanted in the Optina Hermitage, and the \textit{stichera} are chanted to \textit{podobny}. The latter are used in the Shamordino Convent.\\
\-\hspace{2em}[\textit{Stolpovoye} chant, literally "column" or "table” chant, is the same as \textit{znammeny}, the ancient Slavonic ecclesiastical chant. It was so called because the notation was written vertically in columns and arranged in tables. The \textit{stichera} are hymns chanted according to special melodies whose titles are indicated at the beginning of every group of \textit{stichera}. A chanter must have memorized these special melodies, the \textit{podobny} (in Greek, \textit{prosomia}), which then serve as the pattern for chanting the \textit{stichera}.

Apart from such technicalities, this note indicates that at the Optina Hermitage, as also at the Shamordino Convent, the traditional Russian chant had been retained when elsewhere the harmonizations based on 19th cent. Western music had become the rule.]} when the chanters made a mistake, he would go up to the choir and point out what was to be sung or read. When he was preparing to serve, he read Compline in his cell and would come to church for the vigil service. Sometimes he served as a simple hieromonk alone with a hierodeacon.\trans{As Superior of the skete, he could have served only when the clergy concelebrated on the feast-days, delegating services at other times to other clergy in the skete.} In the altar, he stood with such piety and attention, immersed in prayer, that it seemed he beheld no one in the church except God and himself. He never engaged in any conversations, and having said the necessary exclamation, he would again stand in a prayerful attitude and say the Jesus Prayer to himself with a prayer rope. Occasionally a monk would come to ask concerning an obedience or for his own spiritual need, and the Elder would always satisfy him with a fatherly, compassionate answer; sometimes, however, he meekly let it be known that it disturbed his prayerful state of soul.

The monks relate that he was so immersed in prayer, he often did not notice when they came up to him, and only when questioned a second time would he come to himself.

While serving, he was always calm and concentrated. He did not like agitated hurriedness or dull sluggishness; he would say the exclamations distinctly. His serving, in general, would bring about a feeling of compunction in those praying.

In the inner direction of the brotherhood, he was very wise and kept a proper measure of strictness and of condescension and never took action in any matter simply on hearsay. He used to say that one learns patience sooner by staying in one obedience; therefore, he did not like to move monks from one obedience to another often. It would sometimes happen that the elder person on an obedience would come to the director to complain about a brother under his command and would ask that he be replaced. Batiushka would ask: “What did he do?”

“He was rude to me."

“And what did he say to you? And why?” the Elder inquired. From the hot-headed monk's answers, it would be obvious that there was little well-grounded reason to replace the brother and that the older one was acting according to passion.\footnote{St. Abba Dorotheus teaches that he who does not resist an arising thought or feeling, but agrees and acts on sinful suggestion, is acting according to passion; but he who resists passions is he who fights and acts against a sinful feeling.} In such cases, the Superior would say to the complainer, “Go, put yourself in his place. Then you'll find out whether it is easy or not.” Of course, he did not always act thus, but when justice demanded it; in general, he demanded uncontradicting submission and humility from those in obedience.

In administrative matters he had good and experienced helpers who greatly lightened his burdens. These monks were faithful and devoted to him. He could depend on them completely, especially since they, like true Optina monks, did nothing without having asked for a blessing from their Superior and Elder. Nevertheless, he never laid cares aside but examined every matter thoroughly, for he remembered that he must give account to the Lord for those things entrusted to him. He considered it a superfluous expenditure if he built anything or made repairs without the utmost necessity. Once he was told that the floor of the refectory was in need of repair and that it would have to be changed. Batiushka did not give a blessing for this immediately, but ordered that it be carefully looked over. It was discovered that the floor, although old, was still able to hold out for a few more years.

His relationship with the skete brotherhood was most loving and fatherly; if someone fell ill or was nearing death, Batiushka would always hasten to his bedside.

Once Fr. Joseph was preparing to go to Shamordino, but he fell ill and did not go. Shortly thereafter, he was told that an elderly hieromonk in the skete was nearing death. The Elder visited him, arranged that he be tonsured into the schema and, on the day of his death, made an urgent request that he be given Holy Communion. “See, it's good that I didn't go,” Batiushka said later, “otherwise I would have grieved if he had died without me.”

At the end of that same year, 1894, the venerable struggler Archimandrite Isaaky, Superior of Optina Hermitage, took rest from his great labors in the eighty-fifth year of his life. This lamp of monasticism was quietly extinguished; he prepared for departure peacefully and in full consciousness. The ship laden with the virtues entered the haven and finished its voyage of many years. For the struggler, the longed-for day arrived for which he had been preparing all his life, but for the Optina brotherhood, it was a day of sorrow and lamentation because they were deprived of an experienced helmsman and wise director. But this loss was hardest to bear for the skete Superior and Elder, Fr. Joseph. In the course of the last four years of Archimandrite Isaaky's life, when the Elder Joseph had become his spiritual father, they became so close to one another spiritually, and they so mutually supported one another, that with one heart they had directed the flock entrusted to them. Consequently, this separation brought about a great change in the inner order of the monastery.

Shortly before his death, the dying Superior blessed his beloved son, who was also his spiritual father and Elder, with the Kaluga icon of the Mother of God,\footnote{The Elder ordered this icon to be placed near him on a table a few days before his own death; and his body was accompanied by it to the tomb.} and said, “The Most Reverend Bishop Gregory blessed me to be Superior with this icon.” Then, as a keepsake, he gave him his favorite staff, a gift from his brother Archimandrite Melety,\footnote{He was an \textit{ecclesiarch} of the Kiev Caves Lavra.} whom he honored very much. The last word of comfort that the dying one said to the Elder was, “One must always hope in God; ‘he that hopeth in God is like Mount Sion, and will not be shaken forever.'”\footnote{Cf. Psalm 124:1.} Thus unto the last minute of his life Archimandrite Isaaky expressed his good disposition toward Fr. Joseph especially, who suffered much at losing such a Superior. But his parting admonition corresponded to the Elder's own disposition to the greatest degree possible. Being strong in faith and hope, he bore the loss well.

The Most Reverend Alexander, Bishop of Kaluga,\footnote{The former suffragan bishop of Moscow.} came to the burial of the universally respected Superior. After the funeral services, Vladyka gave the brotherhood full freedom to choose for themselves a new Superior. When he took his leave of the Elder Joseph, he said to him, “Help them with your good advice.”

The next day the elections took place. The brotherhood first turned to the Elder and asked him whom he would bless them to choose. The Elder pointed out that the Superior of the Meshchovsk Monastery, Archimandrite Dosifey, who had been tonsured in the Optina Hermitage, was a capable person, strict in his monastic life, and familiar with the Optina typicon. The brotherhood with one accord hearkened to the Elder's direction and elected Archimandrite Dosifey.

The new Superior did not govern the monastery for long, but during that period of time he directed all his attention toward supporting eldership, and in all matters he was the first to turn to the Elder for advice.

But then, after two years, there arose a misunderstanding and a certain disagreement between them. This circumstance gave occasion for the Elder Joseph's great love of peace and his amazing ability to remain firm and unbending in his monastic convictions to be revealed in all its fullness.

Whatever were the circumstances in which they would try to persuade the Elder to turn to those who honored him, among whom there were many influential people, he would customarily answer, “Why should I write and arouse lay people against the monastery?” What an edifying lesson for the many who are dissatisfied with the directions of their Superiors!

The Elder loved to repeat and always adhered to the wise saying, “If this matter is not from God, then it will come to nought of its own accord.''\footnote{Cf. Acts 5:38.} Thus, in truth, it always came to pass.

All the Kaluga archpastors who visited the Optina Hermitage showed special respect for the Elder Joseph, but especially Bishop Makary, a man of prayer who understood him spiritually. This hierarch-elder pure in heart and full of grace, understood and valued the spiritual height of the Elder Joseph more deeply than anyone. When he first visited, he conversed with him for a long time and then at dinner, he told the Superior that he liked the Elder very much and that he had made a profound impression by his deep humility. “He will be a great man,” added Vladyka.

The Elder Joseph, despite his way of life which followed that of the ascetics of old and despite his love for inconspicuousness and simplicity, greatly appreciated science and social activity. He took an interest in everything that took place in the social activity of the community. Being in contact with many people active in both church and secular life, he always convinced them not to abandon their responsibilities, for he considered that having good and useful members of society is just as necessary and profitable as having good monks.

Once at a general blessing, someone expressed his regret that one of the novices of the skete had entered an academy. “It is obvious, Batiushka, that it is hardest of all to be a simple monk!”

“No,” answered the Elder, “to be a good bishop is harder yet, and such bishops are needed now.\footnote{This novice, according to the prediction of the Elder, subsequently became a bishop, “rightly dividing the word of truth.”} His life in the skete was not unprofitable; it will bring him much benefit.”

In 1896, by the intercession of Archimandrite Dosifey, the Elder was awarded a pectoral cross.

In the summer of 1899, Bishop Makary of Kaluga again came to Optina Hermitage and again conversed for a long time with the Elder. Batiushka, as Superior of the skete, told Vladyka of his desire and request to abolish the custom of allowing women into the skete on the seventh of September.\trans{This was the eve of the feast to which the skete was dedicated: the Nativity of the Theotokos.} Vladyka reacted favorably, and soon an ukase was sent. The brotherhood warmly thanked the Elder for this benefaction, but several women were very dissatisfied and were ready to accuse Batiushka Joseph of revoking that which the Elder Amvrosy had permitted. Batiushka, with the modesty characteristic of him, remained silent for a long time, but finally, he spoke out. “And what if not only I, but the Queen of Heaven Herself does not desire women in the skete? Do you not know that it was always the desire of the reposed Batiushka Amvrosy? Only, he was not able to fulfill it since the head of the skete, Father Anatoly, was very much against it. Now I am bound to obey the will of the Heavenly Queen and the reposed Elder."

Then he recalled how the wandering fool-for-Christ’s-sake, the woman mentioned earlier,\footnote{See p. \pageref{lady-before}.} had said to him with reference to the skete, “You should purify it.” The Elder added that Batiushka Amvrosy also understood these words of the fool-for-Christ to apply to this situation.

After this explanation, everyone fell silent and understood that undoubtedly the Elder had received some kind of decree from above.

During his conversation with the Elder, the Most Reverend Makary said that he needed a Superior for the N. Monastery and that he wished to take from Optina the father treasurer. “No,” the Elder firmly objected, “the treasurer is most necessary for us; our Archimandrite is ill.” Sure enough, the health of Archimandrite Dosifey soon became so bad that he was forced to retire, and the father treasurer, at the advice and order of the Elder, was chosen as Superior by the brotherhood.

At the beginning of Archimandrite Dosifey's illness, from which he died, the Optina brotherhood expressed the unanimous desire to have Batiushka Joseph for their Superior. The Elder only smiled at this and jokingly said, “Some good Superior I'll be. I won't even go to church. During the winter I won't go out of my cell at all."

But the brotherhood did not accept this answer and maintained their position. “We will do everything for you. You need only sit and give the orders and we will all be obedient to you.” Of course, considering the weak health of the Elder, it was out of the question, but this incident bears witness to the love, trust, and respect which Fr. Joseph enjoyed among the brotherhood.

As a true monk of Optina, the new Superior\footnote{The highly-respected Father Archimandrite Xenophon, who governs the monastery at the present time. [This was written in 1911.\scriptsize -- TRANS.\footnotesize ]} treated the Elder with love and deep respect and gave the brotherhood an edifying example in his own person.

For twelve years, Fr. Joseph was head of the skete and spiritual father of the whole brotherhood. He was known to the Holy Synod and everyone turned to him, for he was a spiritually experienced Elder and the successor to the great Elder, Batiushka Amvrosy. He received many letters from everywhere, and he read all of them himself. Nor did he abandon his great work until his very end.

During the last five years, he obviously began to lose strength and he often said, “I'm getting somewhat weak.” Sometimes, feeling especially overstrained, he would not receive anyone for two days. As soon as he felt a little stronger, however, he would again take up his struggle. To requests that he take care of himself, Batiushka usually answered, “It would always remain on my conscience if, feeling a little better, I did not receive anyone.” Often, when his cell-attendants had not told him of those who had come, he himself would ask, “Isn't there anyone there?”

This short sketch of the life and mutual relationships of the Optina monks clearly shows that Optina Hermitage is the bearer of the ideal of monasticism and eldership, according to the traditions of the great pillars of monasticism. There were, of course, isolated exceptions, as there were everywhere, even in the community of Saint Pachomius the Great himself. But the general spirit and external order of the monastery always served as a beacon for the world and for other monasteries as well.

One should not think, however, that in monasteries life always passes peacefully, and that nothing happens to interrupt the general harmony. On the one hand, one should remember that the wily enemy of the human race cannot remain at peace when he sees the spiritual progress of man, so he always tries to sow evil and discord. On the other hand, one ought not to forget that in the spiritual life such temptations are necessary, as the Apostle says, in order to reveal the ability of the true servants of God.

In the course of his half-century of monastic life, the meek and humble Fr. Joseph had to bear many and various unavoidable griefs and trials in the monastic life, by which the enemy essayed to shake his courageous soul, but he was not successful in “stealing the treasure of the soul."\footnote{Troparion to the Righteous Job.} Fr. Joseph's humility conquered all the devices of the evil one and all these temptations only manifested his sublimity in all its greatness. His gentleness was so great that not only did no one ever hear a single complaint proceed from his lips, but even when others expressed their just indignation, the Elder with a pure angelic meekness would say, “It's nothing; I'm alright.” When some had lost their patience and asked the Elder for permission to speak in their defense, the Elder categorically forbade this and said, “Why? There is no need to. Remember how the Elder Leonid forbade it even when Batiushka Makary begged him.”