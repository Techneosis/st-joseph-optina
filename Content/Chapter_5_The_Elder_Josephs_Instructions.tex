\chapter {The Elder Joseph's Instructions}
\hugered{A}LL the Elder's instructions are permeated with the spirit of the teachings of the holy Fathers and the Elders. For monastics, he placed obedience and self-condemnation above everything, since both of these virtues give birth to humility, which causes God to dwell in the soul and brings to fulfillment the word of the Gospel, “The Kingdom of God is within you.''\footnote{Luke 17:21.}

The Elder never allowed anyone to refuse an obedience. He said that whoever bears a given obedience to the end is deemed worthy of a blessed repose. He often used the example of a certain hieromonk who never refused any obedience and, although unhealthy, never declined to celebrate the divine services when appointed to do so. During one Liturgy, having just taken Communion, he suddenly reposed in the altar wearing all his vestments.

When people complained of the difficulty of obedience and the grief entailed therein, the Elder's face would light up with joy and in his eyes there would shine a tender, fatherly love and he would say in an especially spirited manner, “Well, what of it? Because of it you will be martyrs.” He always stressed that one must have patience in everything, in every place, and endure to the end. “Whatever you take up,” he said, “keep to it, be patient with everything that comes up, don't move from your place, always blame yourself, and you will be saved.”

Once one person told the Elder about a conversation she had heard, that previous Elders, as it were, did not especially constrain their disciples to bodily struggles, but rather sought for the interior monk. Batiushka said, “You tell them: If you don't grow the tree, from whence do you expect fruit? The tree on which the interior monks grows is fasting, temperance in all things, attendance at church services, bodily toil. Then the fruit that is the interior monk will grow on it."

Once at a general blessing, someone called to mind a doctor's assistant who, after a fainting-spell in which he saw much that was instructive in the other world, entered the Optina Skete while Batiushka Amvrosy was yet alive. When he left for military service, however, he again kept company with unbelieving comrades, who after convincing him that he had had a simple hallucination, led him astray so that he remained in the world at a hospital. Batiushka said of this, “The words of the Gospel were fulfilled in him: 'If they hear not Moses and the prophets, neither will they be persuaded though one rose from the dead.'”\footnote{Luke 16:31.}

Once Batiushka was asked why the letters of a certain spiritual but uneducated man breathed such spiritual intelligence that they penetrated right into the soul. Batiushka answered, “As a ray of the sun cannot penetrate fog, thus the speech of a man who is educated but who has not yet conquered the passions cannot have an effect on the soul. However, he who has conquered the passions and has attained a spiritual understanding has access to every heart, even without a formal education."

A woman said to Batiushka, “Give me a rule of prayer; but I fear I'll find it difficult.” The Elder answered, “That which is imposed is always difficult.” Thereupon he told about Archimandrite Isaaky, who while living in the world was preparing for monasticism by ascetic feats; he did a thousand prostrations daily. After he entered the monastery, he told the Elder Lev about this and he had him do fifty prostrations instead of a thousand. In a short time he came to the Elder and said, “Forgive me, Batiushka, I'm ashamed to confess it, but for some reason I can't do fifty prostrations.” The Elder had him do twenty-five. A little time passed and he again appeared before the Elder and said that he could in no way understand why when he was living in the world he had been doing a thousand prostrations, whereas here he couldn't do even twenty-five.

Then the Elder Lev explained to him: “In the world the enemy was helping you. You did them and became proud of it; but here you are doing them not of your own will, but from obedience. Now you can recognize your feebleness and are being humbled; that is why it is difficult."

Someone said, “It would be better for me to eat a half pound of meat in the world during the fast, than to be here eating bread to excess."

The Elder said, “Even if you eat two pounds of bread, you still are not sinning against the rules of the Holy Church.”

Once he was asked at a general blessing, “Will we be with you in the next world? Will you help us, and in general answer for us?”

“Yes," answered the Elder, “but only they who were unquestioningly obedient to me in everything will be with me and only for them will I answer.”

One woman said, “I grieve that I have no means and have to work for everything. I'm always lacking one thing or another."

The Elder replied, “'Seek ye first the Kingdom of God and all these things shall be added unto you.'\footnote{Cf. Matt. 6:33.} In days past monks lived by this precept, and the Lord supplied them with everything, even more than was necessary. When I first arrived, no one had his own samovar in the skete, but there was one large samovar for the entire skete in the Superior's cell. After services, the brethren took turns drinking tea there, and all were happy and satisfied. So it was with all other necessities; they sought first of all to fulfill the church services and their obediences. Because of it, the Lord added everything unto them and they lacked nothing. Now they enter the monastery with a weak will and think only about clothes and food; for this reason the Lord does not send us sinners anything.”

The Elder gave this instruction to a certain nun: “If the deeds or sins of your neighbor disturb you and take away your peace of soul, then remember this:

\begin{enumerate}[label=\textit{\alph*}.]
\item If the sinfulness of your neighbor, which you want to correct, disturbs your spiritual peace and irritates you, you also sin. You cannot correct sinfulness by sinfulness; it is corrected by meekness. 
\item Zeal which wants to destroy every evil is itself a great evil.
\item Remember that there is a beam in your own eye, and you are pointing out the mote in your brother's eye.
\item There are imperfections which are inevitable as well as ones which are even beneficial. Good is tried by evil.
\item The example of God's long-suffering should bridle our impatience which robs us of peace.
\item The example of our Lord Jesus Christ shows us with what meekness and patience we must bear human sinfulness. If we are not in a position of command over others, then we must look at their evil-doing with dispassion.
\item Every man condemns those sins of his neighbor by which he himself is censured.
\item Nothing quiets and pacifies us about our neighbor's actions so much as silence, prayer, and love.”
\end{enumerate}

Someone speaking of a relative who had committed suicide tried to prove that God could forgive even him. To this the Elder replied, “It is not our place to pass judgment concerning this: the Lord is able to forgive even this, but it is our concern to fulfill what the law commands. Judges must pronounce punishment on the offender according to the law, although the Tsar is able to show mercy. Someone (Saint Basil the Great) was praying for the Emperor Trajan when he heard a voice saying, 'Do not dare to pray for such people.'”

On one occasion when he went out for a blessing, the Elder gave a discourse on confessing in which he advised, “If you can't say your sins, then it is better to write them down.” He then related how one woman wrote down her sins and gave the list to Saint Basil the Great and asked for absolution. He, however, sent her to Saint Ephraim the Syrian, who absolved all her sins except one, and sent her back to the bishop. On coming into the city, the woman discovered that the archbishop had passed away. With terrible grief and in tears, she fell at his coffin and laid upon it the paper on which her sins were written, and she herself fell down all in tears. When they questioned her, she told them of her grief. Then a priest took the paper off the coffin and opened it; it was completely blank. All her sins had been blotted out."

The Elder said, “Man's conscience is like an alarm clock. If the alarm goes off, and you get up right away, knowing that you've got to go to your obedience, then afterward you'll always hear it. But if for a few days in a row you don't get up right away, saying, 'I'll stay in bed a little longer,' then finally, you won't be awakened by the alarm at all.”

Another time he said, “Gainsaying is the strongest trait in man. If it is in accord with his own desire, a man will sometimes do an even more difficult task; but just tell him to do something easy, and immediately he'll get upset. Nevertheless, one has to obey, although the order may appear wrong. Five disciples came to an Elder and asked to be received into the monastery. He told them to go plant some cabbages with the roots at the top and the leaves in the earth. Two of them started to plant as he had ordered, but the three said, “That's no way to plant cabbage,' and began to plant them their own way. When the Elder came to see how they were doing, those who had planted as he had said, he received into the monastery, but not the others.”

Once while walking in the forest, the Elder was told that in N. Convent there were recluses. Batiushka answered, “This is a dangerous path, for in seclusion the passions grow. It is more profitable to be with others. Over there, on the road, where people walk, no grass grows, but where they do not walk, there it is thick. They go into seclusion from impatience, although it is to our benefit when we get buffeted. The more a tree is shaken by the wind, the more roots it puts out to strengthen it, but one which has been amid calm falls over right away.”

A nun from this convent said, “Batiushka, bless me to reserve in advance a permanent cottage for myself.”

''Reserve some patience for yourself! It is said, 'In your patience gain ye your souls.'\footnote{Luke 21:19.} Without patience not even a temporary home will be built, let alone a permanent one. Patience gives birth to consolation, and such consolation is genuine. But we always search for something a little easier. What is easy for the body is not beneficial for the soul, and what is beneficial for the soul is labor for the body. With labor must we enter the Heavenly Kingdom.\footnote{Cf. Acts 14:22.}”

Once, while walking in the forest, the Elder was told, “I hope you live on for a while, Batiushka.”

The Elder replied, “We will live on for a while; until the new instrument is prepared, the Lord will not take the old.” But then his face suddenly became serious, and having fallen silent for a bit, he added, “One must try to live as well as one can, \label{ch5obedience}or else, because of our disobedience, the Lord will take away the Elders and He will leave no one."

The Elder was told that there was a bad harvest again. “Yes,” he said, “there is a dearth of everything, except sin. The Lord sends us bad harvests because now even the simple folk have completely stopped observing the fasts. And so one is thus forced to fast, whether he wants to or not."

Again he said, “At the present time many suicides are taking place, not only from disbelief, but also from lack of patience. They do not want to endure anything. If the Lord had not given man the natural desire to live, then almost all would kill themselves. Saint Basil the Great wrote about a pagan philosopher who said, ‘Previously, I wanted everything to go my way, but seeing that nothing was done as I wanted, I began to wish that everything be done as it is done; so it was that everything started to be done as I wanted.' You see,” added the Elder, “even the pagan actually realized the truth that one inevitably must endure everything that happens. All the saints asked God for patience, which means they also had need of it.”

Once, someone asked that the Elder himself reveal to the penitent his hidden sins which he did not want to confess. The Elder answered, “No, this must not be done. Everyone must confess and repent on his own, or else he will receive no benefit. As for forgotten and undisclosed sins, one must endure everything that comes to pass.\trans{The Elder Joseph here verifies the Church's teaching that if we suffer here in this life, it is for the cleansing of hidden or forgotten sins, of which we, as mortals, have a multitude. He emphasizes that if we are patient in these afflictions permitted by God, if we “endure everything that comes to pass,” we shall receive forgiveness and healing from God. Here we have the clear teaching of the purging of sins before death, which confirms the Church's doctrine that there is no purgatory or purging of sin after death.} Batiushka Amvrosy never revealed hidden sins except in very rare and special cases, when he saw that the person might die without having repented.”

Someone complained that he was continually sick. The Elder said, “There's nothing to be done. Obviously the Lord wants to save us and cleanse us from the defilement of sin. Although it is not easy to endure the cauterization of our wounds, nonetheless one must endure everything for the sake of spiritual well-being. If we live an uncorrected life, then let us have a contrite and repentant heart over it. The Lord will regard this and not deprive us of His mercy.

“Our path is one of sorrows. We will travel thereon until we reach our appointed fatherland--eternity. In the world there are more sorrows. Although we have our share of them too, they are not like those of the world, for ours are for the sake of God. Only it is sad that we take little care for eternity and will not endure patiently the smallest reproach. We ourselves exaggerate our sorrows when we begin to complain. One must have patience and courage in all things. They will be for us like the anchor of a boat which keeps it from being smashed against a rock during a storm.”

“How is complete dispassion obtained?” the Elder was asked.

“With complete humility,” he answered.

The Elder was asked if the words of the Apostle, “He that regardeth the day, regardeth it for the Lord, and he that regardeth not the day, to the Lord he doth not regard it”\footnote{Rom. 14:6.} mean that to fulfill or not to fulfill is all the same.

“No,” answered the Elder, “the holy Apostle said this lest we judge anyone, because no one knows what is hidden within a man. For this reason the Apostle also says: “Let not him that eateth despise him that eateth not; and let not him which eateth not judge him that eateth.”\footnote{Rom. 14:3.} A person appears to be fulfilling everything, but perhaps he is doing it coldly or is exalting himself. Another fulfills nothing, yet he reproaches himself, repents, humbles himself, and gives thanks to God for everything.”

Batiushka was told of a lady who died without the Holy Mysteries because she did not want to receive Holy Communion from the priest in her parish about whom she knew many bad things. The Elder grieved over the deceased and said that one should not be disturbed by a priest's life, since his hand only performs the actions, but it is grace that perfects the Mystery. He then told of how one saint who became fatally ill desired to receive the Holy Mysteries. The nearest priest was very depraved and, it seemed, completely unworthy of the priesthood. The saint was troubled, but then, having overcome his thought, he summoned the priest. During Communion, the Lord deemed him worthy of a vision: he saw that angels were administering Communion to him.

A priest wrote to the Elder that a young woman who had reposed in his parish was appearing to a friend, who was still quite young, pure, and innocent. The Elder answered, “The blessed Diadochus advises us not to trust even true, grace-filled apparitions. The Lord will not rebuke one for not trusting, since he does so not out of disdain for God, but in order not to fall into demonic deception. One ascetic of the Kiev Caves Monastery saw an angel praying in his cell and thought that if he were from the evil one, he would not be praying. Then this imaginary angel gave him this advice: 'You do not have to pray anymore because from now on I am going to pray for you.' The ascetic hearkened and fell into deception, from which he was barely saved by the prayers of the holy Fathers.\trans{St. Nicetas of the Kiev Caves Lavra. The account concerning St. Nicetas may be read in \textit{The Arena} by Bishop Ignaty Brianchaninov, pp. 31-34.}

“Therefore Miss A. might say to this apparition, ‘Although you pray, I cannot believe in you and I ask you not to appear to me; for I am a great sinner and unworthy of grace-filled apparitions.' Let her pray more zealously to God that the Lord deliver her from these apparitions, for in them is hidden great danger. But let me ask, does not A. have the subtle thought that she is living correctly? If she does, then this is why the devil is tempting her. All the great saints considered themselves to be the most sinful and most wretched of all people. Only through such humility did they become pleasing to God. That the deceased appears in radiance and in Heavenly apparel means nothing.”

The Elder wrote to the same priest, “You write that you have several people who are treating others for snake bites through reading prayers. You should take a look at these prayers, for sometimes they are senseless or even mixed with blasphemy. Such prayers only make the demons rejoice, from whom, perhaps, there even comes a little help. One must pray with the prayers of the Holy Church; it is more proper that they be read by a priest of God at that, and not by lay people.

"You ask me to give you directions on deliverance from distraction in prayer. It is impossible for us sinful people not to be distracted in prayer. Nevertheless, one must try with all one's ability to collect one's thought and to enclose it in the words of the prayer, that is, to penetrate into every word of the prayer. Let no one be disturbed by coldness and hard-heartedness, but while acknowledging oneself as unworthy of consolation and compunction, one must force oneself to pray. If prayer is cold, it does not mean that it is unpleasing to God. Sometimes such prayer can take the place of struggle if a person humbles himself and condemns himself in everything before God.”