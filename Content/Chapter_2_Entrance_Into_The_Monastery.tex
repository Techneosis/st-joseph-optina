\chapter{Entrance Into The Monastery}
\hugered{A}CCORDING TO the Optina tradition, every new novice was given an obedience in the kitchen. The new “Brother Ivan” was likewise assigned as cook's helper in the skete. From the first days of his new life, the good qualities of his soul were revealed; obedience without contradiction, silence, and modesty began to strengthen and develop under the influence of spiritual nurture. The quiet monastery life was in accord with his heart's desire. Here he rested in spirit and rejoiced that the world was left behind, and that its noise and bustle did not intrude. Having seen and suffered everything in the world, he could fully appreciate the tranquillity of the monastery.

Great Lent and Pascha went by; the Fast of the Apostles was approaching. On the day before the Fast, the brotherhood ate supper and went to the evening prayer rule. They were cleaning up in the refectory. Brother Ivan was picking up the dishes, and having scraped some of the remaining sour cream out of the cups with his finger, he simple-heartedly licked it off. Suddenly, he felt someone lay a hand on his shoulder from behind. Looking around, he saw before him the head of the skete, Fr. Pafnuty. Blushing, Brother Ivan stood as one guilty, but the superior, smiling affectionately, said to him, “Brother Ivan, do you want to move in with the Elder?”

“I lost my head,” related Fr. Joseph later, "and instead of answering, ‘As you bless,' I said, 'I do.'" Then excusing himself before the Superior, Brother Ivan added, “Forgive me, Batiushka, I'm always eating.”

Fr. Pafnuty said, “Eat and drink, but just understand the matter."

The next day he moved to the Elder Amvrosy's hut where he lived an even fifty years. On the one hand, the proximity of the Elder comforted and made him happy, but on the other, it often disturbed his inexperienced mind. The constant reception of visitors, the constant bustle and talk quickly began to burden him. Because of his inexperience in the spiritual life, he did not tell the Elder of his feelings and confusion, and thought after thought worked upon his confused soul. Once he sat at the table, leaned on his arm, and gave himself over to the thoughts which had captivated him: longed-for Kiev with its sacred treasures loomed again before him in his imagination, and holy Athos with its hermits, anchorites, and skete-dwellers. “The real monastic life is there. Holiness is there. The wondrous elder-ascetics are there . . . And here --none of that; here it's all people; here one isn't even always able to go to church. Never is there time to pray and read. Here, the Great Russians don't even chant right. It's better to leave and go to Athos.”

Caught up in a wave of “doubting thoughts,” the young novice did not notice that the Elder had come into the cell where he was sitting. Suddenly he heard behind him the voice of Fr. Amvrosy, who struck him lightly on the shoulder and said, “Brother Ivan, it's better with us than on Athos. Stay with us!”

These words so struck the young novice, that right on the spot he fell at the Elder's feet in repentance and compunction. From that moment, he clearly understood that all his disturbing thoughts were simply a “temptation.” Why seek for another place or other elders since here he was the immediate disciple of an Elder who read his innermost thoughts? He gave himself over to him completely, forever. Right up until the Elder moved to Shamordino Convent, he was his inseparable cell-attendant and helper. His love and devotion to the Elder were boundless; his obedience was edifying to all. Not only the will, but every word of the Elder was law for him.

After he had settled for good in the Elder's cell, Brother Ivan had much to endure, of course, from inadvertent clashes, as well as trials deliberately brought upon him by people. But the aim of the spiritually experienced Elder Amvrosy in this regard was to teach the young novice humility little by little. From his very entrance into the group of the Elder's cell attendants, he was placed under the immediate direction of the head cell-attendant, Fr. N., a man of most severe character. Stern and morose, he never gave instructions and did not show the new novice what to do or how to do it; but if Ivan from his inexperience did something wrong, he would receive a harsh scolding from his strict teacher. This first school of patience, in all probability, served as a beginning of the cultivation of that blessed self-condemnation which the Elder Joseph so often extolled later to his spiritual children, and which made him so peaceful, meek, and joyous for the rest of his life. Injustice and a multitude of trifling things annoy the ordinary person, but when he learns to find fault with himself through attention to his own conscience, then he first of all judges and condemns only himself, and therefore accepts the judgment of his neighbor as a deserved chastisement from God for his sins. Not only is he not annoyed, but he even thanks his neighbor. With strict attention to his conscience and with the help of enlightening grace, conceit gradually gives way to self-condemnation and a correct view and knowledge of himself. Then all former stumblings burden his soul, and later ones are noted in the most minute detail, and they trouble the sensitive conscience; as a result, he comes to continual, sincere condemnation of himself for everything which he experiences which is in opposition to the law of God.\footnote{The extant notes of the Elder Joseph, written in his own hand, in which, with deep humility, he brings to mind all his previous, most common, human transgressions and repents of them as being a great sinner, may be used as an example.}

Soon, another cell-attendant, the elderly Fr. Maxim (in the mantia\footnote{See Glossary, p. \pageref{mantia}.} Fr. Michael), came to the Elder and replaced Fr. N. He also was of a stern disposition, but had a good heart. Brother Ivan succeeded in subduing him by his meekness, and they became friends. Despite this, Fr. Michael, by reason of his character, quite often shouted at him, but Ivan endured everything patiently and always tried to please and calm him in all things. As a result, Fr. Michael chose Fr. Joseph as his own confessor after the death of Fr. Amvrosy, even though he himself was both older and a priest.

As is generally known, there lived in the Optina Skete the Hieromonk\footnote{See Glossary, p. \pageref{hieromonk}.} Clement Sederholm, who was the son of a German pastor, a convert to Orthodoxy and a monastic, and Fr. Amvrosy's most dedicated disciple and secretary. He was a man of rare sincerity and high, noble spirit, an ascetic, an exact fulfiller of the monastic rules. But he also possessed an extremely short temper, as well as a purely German compulsion for order. As the Elder's secretary, Fr. Clement had frequent dealings with his cell-attendants, who partly because of their simplicity and partly because of their unintellectual approach, frequently irritated him beyond description. He did not get along with a single one of the cell-attendants. Fr. Clement was tormented by it; he repented, confessed to the Elder, and humbly bowed and asked forgiveness of the cell-attendants for his lack of restraint, yet he could not control himself. But here too, Fr. Joseph came out the victor. Fr. Clement himself said, “Fr. Joseph is the only person with whom I just can't get irritated.”

In this manner did his invariably kind and humble disposition influence everyone. He was peaceful with all and could pacify and humble anyone by his own humility, meekness, and compliance. After three years, on April 15, 1872, he was tonsured rassophor\footnote{See Glossary, p. \pageref{rassophor}.} and became Fr. John.

Blessed is he who enters a monastery with a clear understanding and full readiness to walk the path of obedience steadfastly. Such a one will daily ascend the ladder of perfection.

Ivan Evfimovich entered upon the monastic way with just such a readiness. He recognized profoundly that the principal basis of the monastic life is renunciation of oneself; therefore, from his very first steps, having taken up the cross of obedience, he followed after his Abba who was leading him to Christ.

On June 16, 1872, he was tonsured a monk with the name, Joseph.\footnote{After St. Joseph the Hymnographer, who is commemorated on April 4.} It goes without saying that his generally serious disposition became especially concentrated and deep from that time on.

He kept full obedience to his Elder, neither contradicting him nor doing anything without his blessing or knowledge.

Once the Superior (at that time still an hegumen\footnote{See Glossary, p. \pageref{hegumen}.}), Fr. Isaaky, came to Fr. Amvrosy. While sitting and waiting in the reception room until the Elder could receive him, the Abbot took a book from the table. Just then the cell-attendant, Fr. Michael, came in. The wise and spiritually experienced Superior, wishing to edify him, asked, “Fr. Michael, will you bless me to read this book?"

Fr. Michael answered good-naturedly and said with a low bow, “As you wish, Fr. Abbot; whatever book you please.”

Fr. Isaaky quietly began to read. After a short time, Fr. Joseph walked in for some reason. The Abbot turned to him with the same question. “Fr. Joseph, may I read this book?”

The Elder's true disciple answered modestly, “I'll ask the Elder right now.”

This answer very much pleased the Superior; it showed how Fr. Joseph had learned to live by the will of the Elder.

In 1877, Monk Joseph was ordained hierodeacon.\footnote{See Glossary, p. \pageref{hierodeacon}.} It happened completely unexpectedly and in a manner which clearly manifested that the hand of God was directing his way.

On December 7, the nameday of Fr. Amvrosy, Hegumen Isaaky served the Liturgy in the skete church, and afterward came to congratulate Fr. Amvrosy and remained with him to drink tea. The cell-attendants, Frs. Michael and Joseph, served them. At tea, the Superior started a conversation with the Elder about a certain monk whom he proposed for the diaconate. The Elder, for reasons known to himself, found this untimely, and as a spiritual physician, asked the Abbot to set aside his intention, and he suggested instead another more worthy monk whose name he mentioned. All this time, Fr. Joseph was standing near the Abbot with the tray; the latter, having glanced at him intently, said to the Elder with a smile, “Well, Father, you don't want mine and I don't want yours. Instead of them, let's ordain Fr. Joseph!” The Elder was extremely pleased by this suggestion, and Fr. Joseph, having suspected nothing, and being unprepared for ordination, left his cups and food and soon found himself on the road to Kaluga. The Most Reverend Bishop Gregory\footnote{Especially honored even to this day by all the people of Kaluga for the holiness of his life.} received and treated him kindly, but he was not satisfied with his knowledge of the catechism; however, having ordained him deacon on December 9, he dismissed him with his archpastoral blessing and admonitions. The reason for Fr. Joseph's lack of learning was that his difficult obedience as cell-attendant did not give him an opportunity to diligently study the necessary subjects.

According to the custom of Optina Hermitage, every newly-ordained hierodeacon or hieromonk was required to serve the Divine Liturgy daily for six weeks. But because of his poor health, Fr. Joseph did not serve out his forty days. An inflammation appeared in his right side and he almost died; but by the prayers of the Elder, the Lord had mercy on him.

The life of Hierodeacon Joseph did not change greatly. Only more work and more worries were added. As before, he helped the cell-attendant, Fr. Michael, and as before, the latter shouted at him, but humble Fr. Joseph quietly endured all undeserved wrongs and every rough treatment. As before, he did not have a separate cell and he slept in the reception room, which was occupied by the Elder all day until eleven o'clock at night. When it was Fr. Joseph's turn to serve in the skete, he had to attend Matins at one o'clock, but only around midnight could he lie down to rest a little on his humble cot. Once, while waiting in the dark corridor until the Elder had finished his reception, and being exhausted from the labors of the entire day, he fell asleep sitting on the threshold. Elder Amvrosy, on the way to his room, stumbled over the sleeping figure. Thus awakened, Fr. Joseph only smiled meekly, while doubtlessly his great Abba prayed that the Lord strengthen his true disciple in the struggle of patience and humility.

According to the teachings of the holy Fathers, obedience gives birth to humility. This proved completely true in Fr. Joseph. Through his uncontradicting obedience, he attained to that humility which subsequently raised him to the spiritual heights.

Beholding the spiritual progress of his disciple, the Elder, as a wise director, guarded and cultivated Fr. Joseph's pure and receptive soul since he foresaw what gifts he would be deemed worthy of in time. Knowing how Saint John of the Ladder reproaches those teachers who do not test the brothers possessing a strong soul and who thus deprive them of crowns for patience and humility,\trans{\textit{The Ladder of Divine Ascent}, Step 4, pars. 27, 28.} Fr. Amvrosy more than once put his ever-faithful and stout-hearted spiritual child through tests in which he would be proven an example of monastic freedom from anger.

Thus on one occasion, when the Elder was busy with a monk, he had to search for some verse in the Sacred Scriptures. The Elder rang for Fr. Joseph and ordered him to look up the verse that he needed. Fr. Joseph, being well acquainted with the Sacred Scriptures, quickly looked up the indicated words and silently showed the Elder. Then the Elder, seeing that he was versed in the Scriptures and wishing to protect him from fatal high-mindedness, humbled him, saying, “Stupid, you don't understand anything; that's not it at all,” and lightly striking him on the cheek, ordered him to leave. In a little while the bell sounded again, whereupon Fr. Joseph appeared again, just as quiet and good-natured, as if nothing had happened to him. The Elder repeated the order to look up the same verse, and Fr. Joseph, making not the slightest retort nor showing any kind of impatience, again took the Scriptures silently, looked up the very same place, and gave it to the Elder. The visiting monk was amazed by the inner disposition of this true worker of obedience and of Christ-like humility; the Elder, glancing at him with a vigilant eye, made the sign of the Cross over him.

The aforementioned episode was not exceptional in his life; this was his constant and invariable disposition. This did not at all mean that he was indifferent or without feeling by nature; no, he had gained a victory of the spirit over the passions. Batiushka himself used to tell his spiritual children how in the beginning he could not conduct himself as a monastic ought, and how the Elder would teach and scold him because of it.

“Once a well-known ascetic, Hiero-schemamonk\footnote{See Glossary, p. \pageref{hiero-schemamonk}.} Alexander, came to Fr. Amvrosy from the St. Sergius Lavra,” recounted Fr. Joseph. “I served tea and heard him telling about a monk who had formerly lived with us at Optina, and he didn't speak of him especially well; I said aloud, 'Yes, he was like that here, too.' Then Fr. Alexander turned to me and said sharply, 'And how do you know? You yourself are worse than everyone!' I was so ashamed that I quickly went out of the cell; afterward Fr. Amvrosy scolded me sternly. 'You deserve what the schemamonk\footnote{See Glossary, p. \pageref{schemamonk}.} taught you,' he said. 'How many times have I told you not to intrude?' Well, from that time on I feared as it were fire to intrude upon conversations of older people,” Fr. Joseph added humbly.

The solemn opening of the Shamordino Convent, built by Elder Amvrosy eight miles from the Optina Hermitage, took place on October 1, 1884. Bishop Vladimir of Kaluga came to the new community for this solemn occasion. At the Liturgy, there were several ordinations. Fr. Joseph, who was the future guardian and director of this convent after the repose of the founder, Elder Amvrosy, was ordained to the rank of priest. It was as if he received the grace of ordination there in order to pour it out doubly on the convent's inhabitants afterward. From that time on, the Shamordino sisters began to consider Fr. Joseph as “their own,” and little by little, a spiritual bond between them was established.

From the very first day of his priestly ministry, he began serving earnestly, distinctly, without mistakes and confusion. His serving was always unhurried and pious; he became especially joyful on days when he served. As is known, because of his sickly condition, the Elder Amvrosy did not go out to church, and on Sundays and feast days the all-night Vigils were held in his cell. The new Hieromonk Joseph now bore this duty, and also, with the blessing of the Elder, brought him the Holy Mysteries every fortnight.

Not long before his being ordained presbyter, Fr. Joseph, not having had his own separate corner for more than twenty years, was given Fr. Michael's cell, for the latter, already a hieromonk, had settled in the skete. Now Fr. Joseph became the senior cell-attendant. He considered it his chief duty to take care of the Elder and to be concerned for his peace. It was observed that he very much loved order and accuracy. Knowing how many dissatisfied and grumbling visitors blamed both the Elder and his cell-attendants for being forced to wait for a long time to see the Elder, Fr. Joseph tried to placate everyone. To do so, he often came out to those waiting, listened to the needs of each, and at a convenient time related them to the Elder.

Everyone, monks as well as visitors, came to love the quiet and serious Fr. Joseph very much, and everyone turned to him with confidence. It seemed as though not even one person could be displeased with him. There was not a shadow of partiality in him--he treated everyone equally and in a friendly manner. Once an old monk was asked, “Why do you all love Fr. Joseph so much?”

He answered, “Because he is a true monk. He does not waste one word, and one just wants to listen to him."

Both the novices and cell-attendants also loved and respected and willingly obeyed him, and esteemed him for his fair-minded strictures,

When Fr. Joseph went out to the visitors, he attentively listened to them, but nothing was ever said from himself. He would bow and say, “I'll ask,” or “I'll pass it on to the Elder.” When he brought out Fr. Amvrosy's answer, he related it with exactness, neither adding nor changing anything. As word about him began to spread, those coming to Fr. Amvrosy who had not spoken with him before would first call upon Fr. Joseph, his close and faithful disciple.

One young person came to Elder Amvrosy for confession and, while waiting to be summoned from the reception room, was in great agitation and confusion as to how he ought to confess his sins. At that moment Fr. Joseph walked into the reception room, gave him a book of essays by Bishop Peter and said to him, “Here, this is a good book to read before confession.” The young man began reading this book and he immediately found just what he needed to clear up his confusion. Just as he had finished reading this helpful page, he was called in to see the Elder. He marveled at this incident.

While satisfying all the petitioners as far as was possible, Fr. Joseph was also able to protect the Elder. When he saw that the Elder was already overly tired, he would draw the conversation to a close, and with humility would firmly remind his beloved father that it was now time to rest. The Elder, because of his love for his neighbor, often forgot about himself and the reception sometimes lasted until eleven o'clock in the evening; then Fr. Joseph would go in as if for some other reason--to wind the clock, for example--and would look at his Elder lovingly. Having understood his beloved and humble disciple, the Elder would say to the visitor, “Well, good-bye now; Fr. Joseph has started winding up the clock, which means it's time to break up.”

Fr. Joseph, despite his troublesome obedience and his constant presence among people, occupied himself very much with reading the works of the holy Fathers, the teachers of monasticism. He drew from them those inexhaustible stores of spiritual wisdom with which he was enriched and by which he brought profit to those who came to him for advice and comfort. His favorite, and we might even say, inseparable book was the \textit{Philokalia}.\footnote{See Glossary, p. \pageref{Philokalia}.} He preferred to read books in Slavonic, and when one said to him that they are difficult to understand, he used to answer, “That which is profitable is gained by labor.” In general he read only the works of the holy Fathers and did not like the newer writers; he would say, “New writings, although spiritual, nourish only the mind, but leave in the heart a cold emptiness."

“When did you find time to read books at your obedience?''\trans{The word “obedience" as used here signifies an assigned task or duty in the monastery.} he would be asked.

“As soon as I had a free minute, I would take up a book right away,” he answered.

“And when would you see the Elder? Surely there were always people with him late into the night, and then you had to read your evening prayer rule.”

“Sometimes I had to see him at tea-time; but moreso when I came to receive a blessing for the night, then I would say what was necessary, and truly, is much time needed to say what bothers the conscience?” added the Father humbly, showing by this that he had learned from the Elder's way of life.

One cannot speak about his inner life, for it was hidden and known only to God and the Elder. In general, he was a person of profound, inner activity, and always maintained stillness of heart and unceasing prayer. That he practiced the “Jesus Prayer” is witnessed by his many directions on this subject, which were written down word for word by his spiritual children. Just as a vessel full of fragrance cannot be concealed, so the inner virtues of this pure soul could not but be revealed. Great humility and the grace-filled light of inner prayer illuminated and attracted people's hearts to him. The Elder Amvrosy, seeing in him his future successor, did not prohibit him from conversation with the visitors; he even directly blessed a few to refer their spiritual needs to Fr. Joseph. It often happened that the Elder Amvrosy would answer a question by replying, “Ask Fr. Joseph.” The amazing thing about this was that Fr. Joseph's answers were always identical with those of the Elder on the same subject. Many began to notice this and at first, in order to be convinced of it more assuredly, they tried to verify it. They would ask Fr. Joseph something; when he had answered, they would go and ask the same of the Elder Amvrosy, and he would answer with the very same words and thereupon, without fail, he would smile and wink, as if hinting that he knew what was going on. The Elder obviously did this to strengthen their faith in his disciple; but later, when Fr. Joseph's spiritual gifts had already been demonstrated sufficiently, the Elder said sternly to one woman, “Don't test him any further.”

A nun who was devoted with her whole soul to the Elder Amvrosy and who did nothing, even in her everyday life, without his blessing, once asked if she might visit a certain lady living in the Optina guest house who had invited her. The Elder said, “No, it's not necessary. Don't go.” A long time passed; another woman living there also invited her over. Remembering that the Elder had not blessed her the first time, she made up her mind not to go to this one either, but being in the habit of asking about everything, she called Fr. Joseph out and asked him to tell the Elder that so-and-so had invited her over. Fr. Joseph, without hearing her out to the end, said, “Well, why not, go on, go on, go see her.” The nun was greatly troubled by this answer. First of all, she had not asked Fr. Joseph, but had only requested that he refer the question to the Elder, and secondly, his answer was completely opposed to the Elder Amvrosy's opinion. Fr. Joseph went down to the lower hut.

While the nun was in confusion awaiting his return, the Elder suddenly came out for a general blessing. Seeing the nun's troubled face, the Elder asked, “Why are you standing like that?”

Not wishing to explain the reason for her confusion in front of everyone, the nun said, “N. invited me to come over.”

“Well, why not, go on, go on, go see her,” answered the Elder, laughing. Word for word the same answer! Then the nun noticed that the Elder, with a smile on his face, was intently looking at someone. It turned out that Fr. Joseph was standing behind her and was also smiling ... The nun was startled by this occurrence, and from that time on she believed that in Fr. Joseph there lived the same spirit of grace that was in his teacher.

Thus the Elder gradually prepared him for eldership, teaching him both by word and by his own example. He loved him and trusted him in everything, calling him his right hand; and in the course of thirty years, right until the time when he moved to Shamordino,\footnote{He reposed there on October 10, 1891.} he never parted from him. One should have seen with what joy and what love the emaciated Elder's face lit up when one of his spiritual children told him about Fr. Joseph--how well he spoke, how he had calmed someone, how humble and how good he was. Once, in order to avoid untimely renown, the Elder said, “Wait a bit, be quiet about this matter for the time being,” although on another occasion he reminded someone, “Remember? Fr. Joseph told you the truth,” for his loving, fatherly heart found such comfort and delight, for in this he saw the dawn of a new luminary. Although Fr. Joseph did not perceive it himself, the Elder endeavored to reveal him little by little and to show the spiritual treasures of this chosen one of God.

Once, two nuns brought an icon of the Mother of God, which one of them had painted, to show the Elder. Afterward, they confessed that they had agreed together to tempt the Elder and, having shown him the icon, asked, “What do you think, Batiushka, does this look like the Heavenly Queen?”

The Elder answered in a serious tone, “We should ask Fr. Joseph," and he called him in. Fr. Joseph came in and the Elder said to him, "Now tell them whether this face looks like the Heavenly Queen’s.” Fr. Joseph smiled with his angelic smile and humbly lowered his eyes. The Elder dismissed him.

Another time, a nun, while speaking with the Elder, asked what the Mother of God was like in the last years of her life. The Elder said to her,“You go to Fr. Joseph and ask him what the Mother of God was like when she was sixty years old.” Of course, Fr. Joseph, in his humility, said that he knew nothing. With these actions, the Elder undoubtedly wanted to show that Fr. Joseph had been deemed worthy of Heavenly visions.

Once, one of those who used to go to Fr. Joseph with the Elder's blessing and who saw his weak health anxiously said to Fr. Amvrosy, “If Fr. Joseph becomes an elder, very likely he will be one like you, sickly and weak.”

“Well, so what if he is. It's nothing. He'll be humble because of it,” the Elder answered and then added, “All my infirmities have passed over to Fr. Joseph.”

These words, on the one hand, displayed the profound humility of the Elder Amvrosy, who considered himself only an infirm vessel, and on the other, they served to confirm that Fr. Joseph would be the future bearer of the grace of eldership.

There lived at the Optina Hermitage an aged, clairvoyant Elder, Fr. Pachomy, a fool for Christ's sake. He loved Fr. Joseph very much. Even when he was still a simple monk, without fail every time he met him, Fr. Pachomy would ask him for a blessing. “Fr. Pachomy, I'm not a hieromonk,” Fr. Joseph would say, smiling at him.

“I'm surprised,” Fr. Pachomy would answer, “Fr. Joseph is the same as Fr. Amvrosy.”

\label{lady-before}A certain handmaid of God, a fool for Christ's sake, came to visit Fr. Amvrosy, and seeing Fr. Joseph, she said to him, “A certain Elder had two cell-mates, and one of them remained in his place.”

Thus almost from the very beginning, Divine Providence foretold the eldership to which he was destined by God. It may be that in his private conversations the Elder Amvrosy spoke about this more clearly and more explicitly, but Fr. Joseph, in his humility, never revealed it. In general, even after the repose of the Elder, he astounded and attracted everyone with his utter humility; he never attributed any significance to himself. Thus he once wrote to someone, “What am I without the Elder? Zero and nothing more.” And with this he drew on the letter a huge zero as an illustration.

As was mentioned above, Fr. Joseph was very poor in health, and because he lived constantly with the Elder who, on account of his own illness, never went anywhere in the winter or even in cold weather out of fear that he would catch a cold, Fr. Joseph also became unaccustomed to the fresh air and was often a prey to colds. One should have seen with what patience he put on warm clothing every time he went out to the porch or the backyard, as happened often in his obedience as cell-attendant.

He ate very little food. Once, someone who was amazed by such abstinence asked him if this had been difficult for him to accomplish or if it had been given to him by nature. He answered with the following words, “If a person does not force himself, then even though he eat up all the food in Egypt and drink up all the water of the Nile, his stomach will still say: I'm hungry."\trans{Cf. \textit{The Ladder of Divine Ascent}, Step 15, par. 27.}

Fr. Joseph conducted himself as an equal of the brethren; he did not become familiar with anyone and he never went to any of the brethren, but only to church, unless the Elder sent him somewhere. When he travelled with the Elder to the cottage, the only recreation he would allow himself was fishing in the pond, but even in this innocent pleasure it was more his love for seclusion that was in evidence. His chief occupation was the observation of inner quiet--the desert was in his own heart, where the quiet light of prayer shone, where blessed humility dwelled, and whither he withdrew, "fleeing from faintheartedness and from tempest.''\footnote{Cf. Ps. 54:7-8.}

Consequently, in his directions later in life, he also cautioned those who came to him, especially the impatient ones, against an outer seclusion which lacked an inner seclusion; for instead of benefit, it often brings harm. Thereupon, he would often call to mind the monk who out of impatience left the community and settled in the desert without having first overcome his passions. Despite the fact that no one tempted him, even there he could not withhold himself from anger and he would get annoyed even with inanimate objects. In the end, he returned to the monastery after having, in his vexation, smashed a pitcher which was repeatedly tipping over.\trans{\textit{Apophthegmata Patrum}, Life of Saint Pachomius, 69.}

Nevertheless, he would encourage those who were disposed to the inner life and who avoided idle talk, and he would speak to such people primarily about patience, self-condemnation, humility, magnanimity, and meekness--in a word, of those virtues in which he himself abounded.

He never made himself conspicuous in anything; he went about his affairs quietly and humbly. He was the Elder's true assistant, but conducted himself as though he were not in such a high position. His manner was natural and spiritually simple. Possessing profound, inner humility, he had no need for humbleness of speech, which often displays only inner emptiness and has a disagreeable effect on others. On the contrary, inner humility is impossible to describe in words, but it makes itself felt and can even influence a sinful person.

Fr. Joseph's love for the Elder Amvrosy was just as quiet and reverent as his whole life; he did not like to and could not put into words his own feelings, but his affection was so profound that he was ready to give his life for him.

Once the skete brotherhood was frightened by the arrival of an unknown man who waved a pistol and shouted, “I'm going to Fr. Amvrosy!” Since no one dared approach him out of fear that he would shoot, the man headed for the hut. Someone had warned Fr. Joseph. He was not in the least disturbed, but with a peaceful countenance, and probably with a secret prayer, he went out to this stranger and meekly asked him what he wanted.

“I need to see Fr. Amvrosy,” he answered, brandishing his pistol.

Fr. Joseph, gazing cooly into his eyes, made the sign of the Cross over him; the stranger immediately dropped his hand, and one of those present succeeded in grabbing the pistol. The person turned out to be insane, but the incident showed what selfless love the Elder Amvrosy's disciple had for him.

Another time, when a woman wanted to involve Fr. Joseph in a matter concerning money, which would have been unpleasant for Fr. Amvrosy, she threatened Fr. Joseph. In reply he quietly said to her, “What of it! I'm even ready to go to prison for the Elder.”

The first Abbess of Shamordino, Mother Sofia, was renowned for her intelligence and devotion to the Elder, and he, in turn, entrusted much to her and told her many things. She often said, “Batiushka really loves his Fr. Joseph; and there's good reason for it.”

Indeed there was--he was a true disciple who never in act, word, or thought contradicted the Elder in any matter whatsoever, be it important or insignificant. His obedience, or better yet, his whole monastic life, confirmed before the eyes of everyone the justice of the answer which one Athonite Elder gave to those who questioned why there are no Elders now. "Because,” he answered, “there aren't any true disciples left.”

All the great Elders, ancient as well as contemporary, were in their own time true disciples. By completely cutting off their will and reason, by being obedient with their whole soul, by sincere and constant self-condemnation, they attained that blessed humility which both enlightened their hearts with grace and illumined their mind with the light of the mind of Christ,\trans{See Phil. 2:5.} whereby they were enabled to become directors and teachers of others and could understand every temptation and spiritual disorder. They became experienced in everything and possessed the gift of discernment. The holy Father John of the Ladder said, “Do not search for a clairvoyant elder, but rather for a humble and experienced one.”\trans{Cf. \textit{The Ladder of Divine Ascent}, Step 4, par. 120.} Clairvoyance, though characteristic of almost all those enlightened by the Spirit of God, does not alone serve as a distinctive sign of holiness and spiritual attainment. On the contrary, clairvoyance alone without the other virtues, and especially without humility and pure love, is not only unworthy of praise, but even harmful and fatal; it is not a sign of enlightenment of the soul, but rather of self-delusion, or what is otherwise called deception.\trans{\textit{Prelest} in Slavonic.} Not everyone who foretells the future is a saint. In the times of the apostles, there was a maiden who prophesied; nonetheless, the apostles thought it necessary to drive the spirit of prophecy out of her.\footnote{Acts 16:16-18.} The holy Apostle Paul clearly says in his Epistle to the Corinthians that it is possible to be a righteous man outwardly, and to prophesy and to know all mysteries and to work wonders; but if such a one does not have the requisite spirit of the love of Christ as well, then everything else loses all strength and meaning.\footnote{I Cor. 13:2.} In the history of monasticism, are there not many examples of great strugglers and wonderworkers who, not having been guarded by humility, deluded themselves with their own righteousness and therefore perished?

Someone asked the Elder Amvrosy about this subject: “Does it ever happen that one meets a person who correctly relates the past, foretells the future and it comes true, and yet, at the same time, it is difficult to consider him a saint, because of the other aspects of his life?” The Elder gave this explanation: “One ought not believe all blessed 'fools' and the like, even though their words come true, since not everything foretold comes from God. The enemy, being a spirit, knows the past and therefore can point out diverse instances, and sometimes, in order to mislead, even certain sins. Although the future is not revealed to him, yet since he is a spirit, he also can find out something of the future by certain of our expressions, conversations, and in general by certain signs, and he can make suggestions to the foreteller. In other respects,” added Batiushka, “the distinctive property of demonic prophecies is that they are always gloomy, repulsive, and they always promise misfortunes and always bring only confusion to the soul. If one of the servants of God prophecies by inspiration of the Holy Spirit, then, although they sometimes presage grief, this prophecy is accompanied by a peaceful state of the soul, full of repentance and contrition."\trans{See \emph{Life of Saint Anthony the Great}, § 22-36, Nicene and Post-Nicene Fathers, Vol. IV, pp. 202-6.}

Almost that same question was subsequently put to the Elder Joseph. “If one meets a person, obviously clairvoyant, yet at the same time one feels that in him something is not right, how can one ascertain if his clairvoyance is from God?”

“One can recognize such people by their humility,” answered the Elder, “because the enemy can give clairvoyance to a person, but he never gives humility--that burns him."