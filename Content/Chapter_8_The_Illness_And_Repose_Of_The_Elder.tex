\chapter{The Illness And Repose Of The Elder}
\hugered{F}ROM 1905 ON, the Elder began to fall ill frequently, and he obviously became much weaker physically, but in spirit he was just as alert and lucid as ever. His strength failed so much that he could no longer fulfill his duty as head of the skete, and after a serious and dangerous illness during May of that year, he requested that he be allowed to retire. Then he was also compelled to cease being a spiritual father because this especially exhausted him. “It is hardest for me to hear confessions,” said the Elder, “because they cannot be put off. Even though you are dying, you must hear confessions.” Persons devoted to the Elder, of course, received this retirement with great sorrow, but since they loved the Elder and wished to keep him, they submitted to this necessity without complaint. However, the spiritual bond of the children with their father was not broken. As before, the brotherhood went to him with their griefs and perplexities; and likewise, the nuns of the convents also brought their ailing souls to this experienced spiritual physician and opened to him the secrets of their hearts, all their thoughts, temptations, their wounds of sin.

The loving Elder did not refuse anyone. He patiently listened to everything and meekly guided everyone to accept with patience all that came one's way. Often one of his angelic smiles was enough to dispel despondency and hardness of heart without leaving a trace.

In the face of all the griefs and unpleasantnesses of life, it is always easier to live near such grace-filled people who so generously communicate to others the abundance of their inner light and warmth. No matter how heavy or gloomy the soul, no matter in what difficult circumstances a man may find himself, no matter how one's proud heart storms and seethes, if one has an anchor of hope in the person of a grace-filled Elder, then all the waves of internal and external temptations will be broken against this rock. The soul becomes light and bright, and one is amazed at why one should have been so upset. The angry heart is humbled and the crushed soul is set at peace. What happens then to such a one is accurately described by Russian poetry:

\begin{adjustwidth}{1.5cm}{1.5cm}
\center
\vspace*{.5cm}
Then the trouble in my soul is scattered,\\Then the wrinkles on my brow dispelled,\\And in the heavens I behold the Lord,\\And on the earth again find joy untold.
\vspace*{.5cm}
\end{adjustwidth}

Yes, it is precisely thus when the light of grace which illuminates the soul of a righteous man touches the sinful man wrapped in gloom, and having torn asunder the veils of conceit and egoism, opens before him a true and correct vantage point from which he sees life, himself, and everything around him.

That is why all those monasteries where eldership blossoms forth are so valuable and enjoy special respect; but alas, to our great sorrow, they are growing scarce in present-day monasticism.

The year 1911 came; day followed day. Everywhere life continued on its course. The Elder was weak and felt ill, but did not abandon his struggle. In the month of February, an event in the Shamordino Convent strongly affected his health. Abbess Catherine,\footnote{She had been sick since May, but then began to get well; and when everyone considered her to be on the path to recovery, she suddenly reposed from hemorrhaging.} the Superior of this convent, suddenly reposed. Extraordinarily intelligent and cultured, possessing an exalted and noble character, she was distinguished by a true monastic spirit. A devoted and close disciple of the Elder Amvrosy, she had the same fullness of faith in his successor. The Elder Joseph had been completely at peace concerning the convent because he knew that it was in capable and good hands. Therefore, her death was a great sorrow for him. This loss, together with the immediate increase of questions, worries, and associated business affairs completely exhausted the Elder so that he took to his bed. In the middle of the Great Fast, he became strong enough to receive people again, and in Holy Week, he was especially joyous and happy. Noticing his unusually joyful condition, those around concluded that he had had some kind of grace-filled vision.

During the first two days of Pascha, when the Elder received the whole brotherhood and the visitors, he was lively, cheerful, and felt well. On the second day of the feast, there was always a festive service in the skete which the Superior of the monastery celebrated. After Liturgy, all that had served came to congratulate the Elder and sing to him “Many Years." The Elder usually would request that “Eternal Memory” be sung for Elder Amvrosy. This time, by mistake, the hierodeacon commemorated “Elder Joseph” in place of “Elder Amvrosy."

On the third day, the eleventh of April, he came down with fever and nausea, and the reception of guests was stopped. The Elder often came down with such illnesses, so no one considered it of any special significance. But when on the following days the sickness did not abate and even became worse, all became alarmed.

“What if this illness were to be his death, and the hierodeacon's mistake turned out to be prophetic?” many thought. His temperature rose and he became so weak and exhausted that his condition became critical. He lay almost without moving and with eyes closed; only his lips whispered the prayer unceasingly. The various questions which the cell-attendants asked him seemed to return him to earth from another world.

The monks, full of devotion and love for their Elder, continually came to visit and to look upon the departing Abba. With sorrow and heavy hearts, they went up to him, received a blessing, and expressed their grief. The meek Elder looked at them with his usual love and angelic smile, and with a quiet and hardly audible voice he answered their questions and tried to illumine their grieved souls with a last ray of greeting. Many of them could not bear it and went away weeping. In their minds arose the image of this humble, meek, and righteous Elder who for so many years had borne their infirmities and who had guided them, encouraged them, and warmed them with his love. Now they felt that they were spiritual orphans, since they were being deprived forever of a spiritually experienced and unerring guide. Before their eyes, that lamp which had been a bright beacon to them in the midst of the raging sea of life was being extinguished.

The illness did not abate, and in Thomas Week, Archimandrite Xenophon insisted that a doctor be sent for. According to the doctor's diagnosis, the Elder had acute malaria. In his opinion, owing to the weakness of the Elder's heart, there was no hope for recovery. This diagnosis cut the last threads of hope, and everywhere the sad news was spread that Batiushka Joseph was near death.

In every place where this news was spread, \textit{molebens} were chanted to the Heavenly Physician and to the Fervent Intercessor for his healing. Especially fervent, tearful prayers were sent up from the convents. As for the Shamordino Hermitage, there's nothing to be said--they had felt the approaching danger for a long time; now the sisters walked in the gloom of night.

On the twentieth of April, the skete's wonderworking icon of the Mother of God of the Sign was brought into the Elder's cell at the wish of the skete head and of the brotherhood, and a \textit{moleben} was served. On the same day, the icon of the Mother of God of Kazan and the rassa of Saint Seraphim were brought from the monastery. During the \textit{moleben}, the Elder prayed quietly; in spite of the length of the two \textit{molebens} served one after the other, he was not wearied, but rather he seemed to be enlived and joyful. As from earliest youth, so to the end of his toilsome life, prayer was for him the best strengthening and comfort. All those in the hut prayed as well, but in the depths of their souls all came to the realization that this life which was precious to all was to be extinguished.

Two days later, the Elder wished to bid farewell and to ask forgiveness for the last time from the skete brotherhood. After them, the brotherhood of the monastery came. The Superior, who deeply honored the Elder, visited him daily. “Not well . . ." said the dying one, looking at him.

“No, Batiushka, live a little more yet, we still need you,” answered the worried Superior.

“May God's will be done,” answered the humble Elder, but to those monks close to him in spirit he would say, “I am dying.” Thus he clearly foresaw his approaching departure from this life, and he prepared for this hour quietly and joyfully, by immersing himself completely in prayer and pious reflection.

When the whole brotherhood had asked forgiveness of the Elder, he blessed the Shamordino sisters to come and do likewise. From the break of dawn, they hurried to Optina in groups of fifty in order to prostrate themselves before their dear Batiushka, to receive his last blessing, and to ask forgiveness. Having waited their turn, they did not go to the guest house, but hurried back in order to give place to others. For several days between Shamordino and Optina, there stretched an unceasing line of weeping sisters. Now they were left complete orphans; at home they had no mother, and their loving guardian and father was leaving them forever. A completely new life was about to begin for the Shamordino Convent, for from its foundation it had rested untroubled like a small child in the mighty hands of the Optina Elders. Not one Superior, not one sister made any arrangement without an Elder. Grief, joy, confusion, and temptation--all were brought to him. In the convent, the word of the Elder was holy; if someone objected to anything, it was enough to say, “Thus did the Elder bless," and all was accepted with faith, because everyone knew from experience that to infringe upon the blessing and decision of an Elder was very dangerous. Even the most restless elements in the convent submitted themselves without a word, since no one dared, indeed, could not, resist the strength of the humility of the Elder Joseph. But now, behold, complete orphanhood. That which was dear to the inhabitants of this convent, that by which they lived spiritually, that which enlivened them--all this was departing from them. A new grave was already being prepared to cover their last treasure. “Set your affection on things above, not on things on the earth''\footnote{Col. 3:2.} for “your conversation is in the Heavens”\footnote{Cf. Phil. 3:20.} was all that one could say to comfort the sisters who were becoming orphans.

The love shown by the dying Elder for the Shamordino sisters was very moving. Although he was exhausted by illness and the unceasing reception of visitors, when those around him suggested that he forbid the ever-increasing number of sisters from coming to him, the Elder answered, “Let everyone come; I will bless everyone. God will help. Otherwise many will be left weeping.” The cell-attendants once held back some sisters for a few minutes in the reception room in order to give the invalid a little breathing space. The Elder opened his eyes and asked, “Surely, this is not all from Shamordino. Isn't there anyone else? Call them in.” For whole days the sisters came, barely able to hold back their weeping. The Elder looked with love upon his orphans, blessed each one with an exhausted hand, and gave them each a small icon. Sometimes his lips whispered, “God bless,'' or “God forgive.” Wishing to comfort them before his departure, he allowed the Shamordino photographers to take pictures, and despite his extreme weakness, he tried to set himself up a little better so that his face could be seen more clearly.

When all the Shamordino sisters had come, the Elder then ordered them to write to the Belev Convent and to say that his condition was very bad. The abbess allowed all to go who wished to, and along the Belev road a similar line of sisters stretched out. For a long time, the Belev Convent of the Exaltation of the Cross had been devoted to the Optina Elders, from the time of the Elder Leonid. The Elders who followed in succession accepted into their fatherly care the Belev Convent which was distinguished by its unswerving devotion, love, and obedience to each Elder. Therefore, many tears were shed by these faithful disciples of the departing Elder, who bade them farewell with his gentle glance.

Visitors also began coming to Optina from other monasteries, and many of Fr. Joseph's spiritual children from the world came as well. The Elder himself made arrangements to telegraph A. Y. Perlovaya, the benefactress of Shamordino, that she might come. Within a few days, the Elder requested that she return home. She wanted to remain near the Elder, but, contrary to his rule of not sending anyone away when a feast-day was approaching, he insisted that she leave on Saturday. While returning to Moscow, Mrs. Perlovaya suddenly felt a pain in her eye. It turned out to be glaucoma, and it was necessary for her to have an operation immediately. If she had remained at Optina one more day, her sight would have been irrevocably lost.

Having spent a few days there, the Elder's nephew, who sincerely loved and honored him, decided to leave since his business demanded it. “May I come to see you again, Batiushka?” he asked.

“Yes, but only in May,” answered the Elder.

Those who could not come themselves wrote to him their anxieties and for the last time asked forgiveness and absolution.

When all of his own people and the visitors had been satisfied, the Elder became peaceful. He obviously began to cut himself off from everything worldly and was constantly immersed in prayer. Once during the night, the Elder, who had been lying motionless, suddenly lifted himself up and repeated several times, "Mother Sofia! We have to pray for Mother Sofia!” The next day in Shamordino, by the grave of the first Superior, the schema-nun Sofia, a \textit{pannikhida}\footnote{See Glossary, p. \pageref{pannikhida}.} was served. After this, the Elder became a little better, but in a few days he had to undergo the last difficult trial. It was that drop which filled his cup of earthly griefs and served as the final cleansing of that sinfulness common to all human nature. Thus, his soul was completely prepared to stand before the face of the righteous Judge, Who is terrible to the evil, but longed-for by the pure and humble servants of God.

From the twenty-eighth on, the Elder Joseph completely stopped taking food,\trans{This was 12 days before his repose, on the 9th.} nourishing himself only with the Heavenly Bread of the Body and Blood of Christ. Right until his death, his mind was clear and he was in full consciousness; he surprised everyone with his memory and his attention to unfinished business and future arrangements. He gave the clearest answers to all business questions, dictated replies to the letters which he received, and two days before his death he even signed them with his own hand. Daily, all who had just arrived, as well as some persons near and devoted to the Elder, were allowed to go in for his blessing. For the most part, the Elder lay with closed eyes. Many, thinking that he was unconscious, expressed in a whisper their grief that they had not received his blessing. Then the Elder would open his failing eyes and would lift up his weakened hand for a blessing. This love, this sensitivity of soul, touched and moved one to tears. One of his spiritual daughters, gazing at the dying one, confessed all her sins mentally. When she finished, she said to herself, “Beloved father! If you hear, then look at me.” At that moment the Elder suddenly opened his eyes, lifted them upward, and then lowered them.

Several days before his death, the Elder began to moan greatly. When asked what was hurting, he answered, “It's strange; nothing hurts me, but I'm becoming weak.”

His strength failed, and when they aired out his cell, the cell-attendants had to carry him into the reception room in their arms. The physician came again, and after confirming his first opinion, he expressed surprise that the Elder had fought for so long against the ailment.

On the morning of the eighth of May, he became a little better, but at midday he suddenly became weak. At about four o'clock, a doctor from Moscow, who had come by chance, examined him and said that his pulse was becoming weak and that he would live two days at the most. Having seen the prepared photographic apparatus, and having learned that Batiushka had permitted the Shamordino sisters to take another picture of him, the doctor quietly said to the cell-attendant, “Do not trouble the Elder with it; he doesn't have much longer to live."

But after the doctor left, the Elder called the cell-attendant and said with a barely audible voice, “Call the photographer quickly; it will get dark soon.” In this incident one can see the depth of his love which moved him to sacrifice himself for others to his very last breath.

Afterward, everyone came in to receive a blessing. The Elder looked at everyone affectionately and said quietly, “Go to the Vigil.”

The evening of the eighth was full of anxiety. Batiushka did not sleep and dozed off only in the morning. At eight o'clock, a hieromonk came with the Holy Gifts. The sick one could no longer expectorate the phlegm which was choking him. When he was asked whether he could receive Holy Communion, he answered, “I can,” and received the Blood only. After Communion, he was offered some tea, to which he said, “What do I need tea for now?” and after this he added, “Pray that the Lord will receive my soul in peace.”

Then his devoted and beloved cell-attendant said, “Batiushka, what do you mean? We cannot pray for your death; we all want you to live. We will pray that the Lord will lighten your sufferings.”

Batiushka, having fallen silent for a little, said, “Well, let it be as is pleasing to God.”

At three o'clock in the afternoon, his temperature began to rise quickly and his pulse strengthened. It was the last surge of life before the end. At four o'clock, Batiushka received everyone for a blessing for the last time. At six o'clock, everyone came to listen to the Canon at the Departing of the Soul. The Elder was fully conscious; they laid a shroud with holy relics on his chest. After the reading of the prayers, the monks surrounding the Elder approached to bid farewell; after them, the nuns and all his close spiritual children approached. The Elder lay with closed eyes although wheezing could be heard in his chest. Then everyone was asked to leave the women's part of the hut and it was closed until morning.

With inexpressible pain in their hearts, the Elder's devoted women disciples left the dear hut forever. Although they were not able to be present to see him at his last moments, they awaited his end at the entrance in the forest.

Meanwhile, the Elder's fever grew worse, his temperature went up to 103° F., his pulse to 130. At ten o'clock, his temperature began to fall rapidly. The Elder lay on his back with his arms folded on his chest. His eyes were closed and his face was alight with such an unearthly light that all those present were astonished; peace and deep calm were imprinted on him. His breathing became shallower; his lips, moving almost unnoticeably, gave witness that the eternal worker of prayer would finish only when his lips were closed in death. At ten forty-five, the Elder gave up his last breath. His pure and righteous soul peacefully parted from his much-toiling body and soared to the Heavenly mansions; his angelic smile irradiated and rested upon his noble countenance.

Thus reposed the ever-memorable Elder of Optina Hermitage, Hiero-schemamonk Joseph, the successor of the great Elders Lev, Makary, and Amvrosy.

At two o'clock in the morning, first in the skete, then in the monastery, a bell announced that the Elder had reposed. “Batiushka Joseph has reposed” rang in the hearts of all. He had completed the path of his sojourn; finished was the course which during his whole life, but especially his last years, had been covered with thorns. His time of trials and temptations had passed, and he was released from the captivity which had wearied his great spirit. The way of the cross was finished, and the time of recompense had come when in all the righteousness of his life, he would shine brighter than the sun. “His spirit went unto God, who had given it.''\footnote{Eccles. 12:7.}

Now lettest Thou Thy servant depart, O Master, for already his eyes see Thy salvation, which Thou hast prepared for those awaiting with patience the fulfillment of Thine immutable promises.

After the preparation of the body, there immediately began uninterrupted \textit{pannikhidas} throughout the night. The first \textit{pannikhida} was served by the skete Superior Hegumen Barsanufy, then all the skete and monastery hieromonks, the spiritual sons of the deceased, served in turn.

That night, several of the monks, not yet knowing that the Elder had already reposed, saw him in a dream appearing bright, shining, and joyful. On the following days, he also appeared to many. To the question, “What, Batiushka? Have you really died?” he answered, “No, I did not die, but on the contrary, now I am completely well.”

At six o'clock in the morning, the body of the deceased Elder was placed in a simple, monastic coffin lined with black cloth and was carried out to the skete church. For the last time, he was leaving the hut in which he had lived an even fifty years. Here he passed his whole life; here he was reborn in spirit and received spiritual baptism from the great Elder Amvrosy; here he traversed the road of the struggle with temptations and passions; here he matured spiritually and reached the perfection of manhood; here he witnessed the great, grace-filled gifts of his teacher and his prayerful sighs, from which his countenance often shone with a Heavenly light; here he learned love and compassion for people; and here, finally, he was often made worthy of grace-filled illumination, and with the power of God he accomplished wondrous and awesome deeds. He comes once more into his skete church where he served and prayed with the brethren, only to leave the skete at last in order to lie at the feet of his great teacher and to await the trumpet of the Archangel.

At eleven o'clock, when the late Liturgy had finished in the monastery, the whole brotherhood with the Superior at its head went in procession to the skete. The Superior went into the skete church by way of the Holy Gates,\trans{It would seem that the Holy Gates were situated at the entrance to the monastery. See also p. \pageref{holy-gates}.} and all the clergy together served a solemn \textit{pannikhida} at the coffin of the deceased. Before the beginning of the \textit{pannikhida}, the Archimandrite ordered that all the nuns be allowed to enter the skete. After the \textit{pannikhida}, the coffin was carried by the hands of the devoted disciples out of the skete church and in procession to the monastery to the chanting of “Holy God” and the ringing of the bells, with church banners, holy icons, and an assembly of the clergy. The dense Optina pine forest sighed, as though it were bidding farewell in its own way.

The coffin was temporarily set in the Church of St. Mary of Egypt.\trans{The main church of Optina Monastery.} After the completion of the \textit{litya}\footnote{See Glossary, p. \pageref{litya}.} served by all the clergy together, the uninterrupted serving of \textit{pannikhidas} began. As one \textit{pannikhida} ended, a different priest would come out of the altar to begin another. The flow of people was enormous. Archimandrites, abbots, hieromonks from neighboring monasteries, and a multitude of the white clergy came;\trans{I.e., the married clergy.} all hastened to pay their respects and to pray for the repose of the luminous soul of the esteemed Elder. The senior nuns of Shamordino and of Belev, as well as many others, even some from very distant convents, came to Optina. Lay people, abandoning their affairs, also hastened thither. This coffin, in which there could barely be discerned the dry, emaciated, almost child-like body of him whose strong spirit enlivened and saved so many hundreds of souls, drew all to itself for the last time.

The Shamordino Convent sent their two choirs: one of nuns and the other of children from the orphanage; the latter, after having chanted several \textit{pannikhidas}, returned home on the same day, but the choir of nuns remained until the day of the burial. The day before the burial, the choir of chanters from the Belev Convent also came, and all the \textit{pannikhidas}, which began at three o'clock in the morning, were sung alternately by the choirs of the two convents.

Here it is necessary to point out that on behalf of the highly esteemed Father Abbot and the elder brethren, all was done for the Shamordino sisters that could serve for their consolation; the convent will forever preserve a thankful memory regarding them for the attention and for the brotherly sympathy which they showed during the illness and repose of the Elder.

On Wednesday the eleventh before the all-night vigil, with mournful tolling of the bells, the coffin of the deceased was solemnly carried into the church dedicated to the Entry of the Virgin into the Temple. During the transfer, the choir of the Belev nuns chanted. As the procession entered the church, it was met by the Shamordino choir chanting “Holy God,” whereupon all the clergy together chanted a \textit{pannikhida}. Thus the orphans surrounded their father all the time, and they bestowed on him the last tokens of their love and devotion.

At six-thirty, there was a solemn Vigil for the reposed served by all the clergy, and the compunctionate prayers soothed the pangs of sorrow.

On the twelfth of May, there was served the solemn funeral of the holy Elder Hiero-schemamonk Joseph, who had reposed as one of the saints. The Liturgy and burial were presided over by the Superior, Archimandrite Xenophon. With him there served the Superior of the Borov-Pafnutiev Monastery, Archimandrite Benedict, the Superior of the Troitsky-Lyutikov Monastery, Archimandrite Gerasim, the former Superior of the Pokrov-Dobry Monastery, Archimandrite Agapit, the skete Superior Father Hegumen Barsanufy, and a multitude of protopresbyters, hieromonks, and priests from the towns of Belev, Kozelsk, and elsewhere.

During the priests' communion, the funeral oration, which had been compiled by several persons, was read by the proctor,\trans{\textit{Blagochinny.} This corresponds to the Greek \textit{periodeutis}, and was in monasteries a position of superintendence in disciplinary matters.} Fr. Theodot, the most faithful and devoted disciple of the Elder. This oration\footnote{Published in a separate booklet.} expressed so faithfully the general feelings, and drew such a simple but warm picture of the deceased, that to listen to it without tears was impossible. The speaker himself wept.

Finally the time of farewell and separation came. The hand of the deceased was warm and soft as that of a living person. Nor was there any deathly coldness about him, but it was as pleasant and warm near him as it had been when he was alive. After the funeral service, the coffin was carried in procession to the place of his eternal rest. The way from the church to the grave was spread with greenery and flowers. The sun poured bright rays into the monastery square which was completely filled with people, and played on the golden brocade (an offering of the Shamordino Convent), on the vestments, icons, and crosses. The large Optina bell tolled mournfully; its intermittent knell, interspersed with the doleful peal of the small bells, expressed the general grief. The last \textit{litya} was performed at the grave, prepared at the feet of the Elder Amvrosy, and the coffin was lowered into the burial vault. All of the clergy, together with all of the monks and nuns, threw in a handful of earth and rendered dust to dust. By this time, the bells were ringing joyfully, bringing to mind eternal joy, eternal blessedness!

A memorial dinner was offered to the guests in the Superior's rooms; dinners were served by the monastery for everyone else in all the guest-houses. No one left without having prayed from his heart for the newly reposed.

A cross was planted in the small mound which arose over the fresh grave. In it shone a perpetual lamp which reminded one of that light and warmth which he, who was now covered by this burial mound, had spread around himself.

This lamp not only recalls the past, but bears witness that the reposed Elder, even after his death, did not cease to warm and enlighten those who fervently honored him after his death. Though so little time has elapsed since his death, cases of his grace-filled help and healing are already to be seen:

\begin{enumerate}
\item On the ninth day, a certain demonized woman was completely healed at his grave. She had long had an illness which defied diagnosis and which did not respond to any treatment. Chancing to be at Optina when the Elder Joseph reposed, she venerated the hand of the deceased Elder in the church and immediately began to scream. Afterward, she would by no means approach the coffin of the Elder peacefully, but would pull away and shout, “I'm afraid, I'm afraid of him!” Finally, the peasant women accompanying her decided to take her up to the grave by force. The sick woman pulled away and furiously shouted, but they forcibly bowed her and placed her on the grave. She suddenly became calm, and having lain a little while, she got up completely well. Spending a rather long time at Optina, she prepared for and received Holy Communion. The attacks of sickness did not return.

\item The nun V.P. of the E. Convent came down with an ailment of the throat. In addition, some kind of sore developed on her tongue so that she could neither speak nor eat. Suffering much, she did not know what to do. At this time, she received a letter from the Optina Hermitage from a hieromonk whom she knew. He remembered her love for the Elder and her grief on his account, so he sent in the letter a few flowers from Fr. Joseph's grave for her consolation. With firm faith in the healing power of prayer, she got up joyfully, went to the portrait of the Elder, and with tears began to ask for his prayerful help. Then, making three prostrations before the icons, she took one flower and swallowed it. Awaking in the morning on the next day, she felt no pain, but only a slight irritation. By the evening of the same day, there remained no trace of the illness.

\item A woman from the town of D. came to the Elder Joseph in the month of May to ask him to pray for her son who was about to take examinations; neither he nor his teachers had hopes for a satisfactory outcome. She arrived only minutes before the death of the Elder, so she could say nothing to him. Not losing faith, she mentally turned to Batiushka and said, “If you have boldness, pray that my son will do well on the exam.” Having returned home, she soon received news that, beyond all expectation, the examinations had gone very well.

\item Owing to his obedience, one hieromonk was unable to come to church when the funeral services were held. He was especially grieved that he had not been present to see the face of the deceased Elder when it was uncovered at the request of those who revered him. Shortly thereafter, he dreamed that he saw the Elder, lying in the coffin, lift the cloth from his face and say, “You grieve that you did not see me--here, behold!” Then the Elder said a few more words which concerned the hieromonk personally and which, he assured us, brought him to fear and compunction. “From these words,” he said, “I saw what great boldness the Elder has even now. What will he be like after the fortieth day?”

\item The night when the Elder reposed, an impoverished nun of the Belev Convent, who had subsisted only by the assistance of the Elder, was very much troubled over how she would live now. In a dream, she saw Batiushka come to her, radiant and joyful. He said to her, “Don't grieve. Here, Batiushka Amvrosy is sending you twenty-five rubles for your needs.” On awakening and discovering that Batiushka Joseph had reposed during the night, she thought that not only Batiushka Amvrosy, but Batiushka Joseph as well would never send it to her now. As it was, she had no hope of receiving anything from anywhere. But how surprised she was when she received a bank note for twenty-five rubles. Soon the same benefactress sent her still another twenty-five rubles, for she remembered that once Fr. Joseph had asked her to help this poor nun.

\item Two nuns of the Belev Convent, devoted disciples of the Elder, were hurrying to Optina for the fortieth day memorial services. Because of their poverty, they came on foot. The weaker of the two became very tired and asked her travelling companion to stop for a rest, but fearing that they would be late for the all-night Vigil, the latter did not want to sit down. Finally, having lost all her strength, she convinced her companion and they sat down on a half-rotted tree. Suddenly, they heard something fall near them and her companion shouted, “Get up quickly! There are snakes here!” Thinking that she was only trying to frighten her in order to force her to go on, the nun at first did not believe, but then with horror she saw a terrible serpent. Underneath this tree there was a nest of snakes, and only the prayers of the deceased Elder, to whose memorial they were rushing with such fervor, saved them from the deadly danger.

\item Mother O. of the Belev Convent relates: “Being poor, I had to earn my own keep at the convent and therefore every work day was precious to me. But out of love and faith in the Elder, I did not begrudge the time required, but occupied myself with writing down instances of miraculous help from Batiushka Joseph, even as I had been charged. And Batiushka did not leave me in debt, but soon rewarded me. To my utmost surprise, a nun, from whom I had not expected to receive anything, suddenly brought me some money and said, “Here, this is for your labors and love for the Elder.” At this time, I had come down with a severe tooth ailment and no matter how much medicine I tried, nothing would help. As soon as I began to labor for the glorification of the Elder, however, the thought flashed through my mind (I believe that it was the inspiration of Batiushka), to take a mixture for disorders of the nerves. Immediately, I felt relief. From that time, by the prayers of the Elder, my teeth have not hurt any more.”

\item After the solemn services in the church, throughout the whole of the fortieth day, the \textit{pannikhidas} at the grave of the Elder did not cease. A nun who was standing near imagined herself at the skete hut. She set before her mind the image of the radiant, meek, gentle Elder as though he were alive. She stood before him and poured out her soul. The Elder arose, took a \textit{prosphora} from the table, gave it to her and affectionately said, “This is for you.” These priceless moments would never be repeated; Batiushka would never give \textit{prosphora} any more. At that moment, they began to chant, “With the Saints grant rest”; the hieromonk, who was serving, hurriedly took a \textit{prosphora} out of his pocket. Turning to a nun who was standing next to him, he asked her to give it to the nun, naming her, who had been so grieved that she would never again receive a \textit{prosphora} from Batiushka. How did it come into the hieromonk's head to send a \textit{prosphora} during the \textit{pannikhida} to her specifically, when she was standing considerably further away than others?”
\end{enumerate}

Although the Elder Joseph had left his children in body, he had not left them in spirit.